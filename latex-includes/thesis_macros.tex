% for header page
\newcommand{\vs}{\vspace{20pt}}

% for header page
\newcommand{\pt}[1]{\MakeUppercase{
    \large{}
    \centering{}
    #1 \\
    \vs{}
  }}

% allow custom lexer for pygmentize
% https://github.com/gpoore/minted/issues/176#issuecomment-695344998
\usepackage{minted}
\usepackage{regexpatch}

\makeatletter
\newcommand{\minted@def@optcl@novalue}[2]{%
  \define@key{minted@opt@g}{#1}[]{%
    \minted@addto@optlistcl{\minted@optlistcl@g}{#2}%
    \@namedef{minted@opt@g:#1}{#2}}%
  \define@key{minted@opt@g@i}{#1}[]{%
    \minted@addto@optlistcl{\minted@optlistcl@g@i}{#2}%
    \@namedef{minted@opt@g@i:#1}{#2}}%
  \define@key{minted@opt@lang}{#1}[]{%
    \minted@addto@optlistcl@lang{minted@optlistcl@lang\minted@lang}{#2}%
    \@namedef{minted@opt@lang\minted@lang:#1}{#2}}%
  \define@key{minted@opt@lang@i}{#1}[]{%
    \minted@addto@optlistcl@lang{%
      minted@optlistcl@lang\minted@lang @i}{#2}%
    \@namedef{minted@opt@lang\minted@lang @i:#1}{#2}}%
  \define@key{minted@opt@cmd}{#1}[]{%
    \minted@addto@optlistcl{\minted@optlistcl@cmd}{#2}%
    \@namedef{minted@opt@cmd:#1}{#2}}%
}

% new minted option "custom" for adding command line option "-x"
\minted@def@optcl@novalue{custom}{-x}

% new minted option "formatter=<formatter>" for specifying pygments formatter
\minted@def@opt{formatter}

% apply "-f <formatter>"
\newcommand\minted@formatter{%
  \minted@get@opt{formatter}{latex}\space
}

% Note: may require fix to regexpatch:
% https://tex.stackexchange.com/questions/578518/regexpatchcmd-failing-with-latex2e-2020-10-01-patch-level-4-use-cs-replacemen#comment1469403_578518
\xpatchcmd*\minted@checkstyle
  {-f latex }
  {-f \minted@formatter}
  {}{\fail}
\xpatchcmd*\minted@pygmentize
  {-f latex }
  {-f \minted@formatter}
  {}{\fail}

  \makeatother
  % end custom lexer for pygmatize

% Need VerbatimEnvironment:
% https://tex.stackexchange.com/a/400115
\newenvironment{hminted}
{\VerbatimEnvironment
\begin{minted}[escapeinside=//,frame=single,custom]{../latex-includes/hazel_lexer.py:HazelLexer}}
{\end{minted}}

% hazel inline; don't really know what's going on here
% https://tex.stackexchange.com/a/181393
\newcommand{\hmintinlinet}{
  \mintinline[escapeinside=//,frame=single,custom]{../latex-includes/hazel_lexer.py:HazelLexer}
}
\newcommand{\hmintinline}[1]{\expandafter\hmintinlinet\expandafter{#1}}

% \newcommand{\ab}{expanded stuff}
% \newcommand{\pythoninline}{\mintinline[]{python}}

% hazel inputminted
\newcommand{\inputhminted}[1]{
\inputminted[escapeinside=//,frame=single,custom]{../latex-includes/hazel_lexer.py:HazelLexer}{lstings/#1.hz}
}

% for presentation
\newcommand{\inputhpminted}[1]{
\inputminted[escapeinside=//,frame=single,custom]{../latex-includes/hazel_lexer.py:HazelLexer}{thesis/lstings/#1.hz}
}


% same as above but with no frame
\newcommand{\inputhnfminted}[1]{\inputminted[escapeinside=//,vspace=-1em,custom]{../latex-includes/hazel_lexer.py:HazelLexer}{lstings/#1.hz}}

\newcommand{\inputhnfpminted}[1]{\inputminted[escapeinside=//,custom]{../latex-includes/hazel_lexer.py:HazelLexer}{thesis/lstings/#1.hz}}

% ocaml inputminted
\newcommand{\inputominted}[1]{\inputminted[escapeinside=//,frame=single]{ocaml}{lstings/#1.ml}}


% extra math operators
% especially for math operators that are introduced in this paper, want to
% declare them here so they are easily changeable
\DeclareMathOperator{\fix}{fix}       % fixpoint
\newcommand{\env}{\sigma}             % environment
\newcommand{\pp}{\Uparrow}            % post-process
\newcommand{\pplc}{\pp_{[]}}          % post-process lambda conversion (substitution) operator
\newcommand{\pplco}{\pp_{[],1}}       % post-process lc part 1
\newcommand{\pplct}{\pp_{[],2}}       % post-process lc part 2
\newcommand{\ppn}{\pp_{i}}            % post-process hole closure numbering
\newcommand{\ppnd}{\pp_{i,d}}         % post-process hole closure numbering (results only)
\newcommand{\ppns}{\pp_{i,\env}}      % post-process hole closure numbering (sigmas only)
\newcommand{\pplcl}{PPI$_{[]}$}       % post-process lambda within evaluation boundary conversion label
\newcommand{\pplclo}{PPO$_{[]}$}      % post-process lambda outside evaluation boundary conversion label
\newcommand{\ppnl}{PP$_{i}$}       % post-process hole closure numbering label
\newcommand{\ppndl}{PP$_{i,d}$}       % post-process hole closure numbering label
\newcommand{\ppnsl}{PP$_{i,\env}$}     % post-process hole closure numbering label
\newcommand{\pth}{p}                  % path
\newcommand{\hci}{H}                  % hole instance/closure info
\DeclareMathOperator{\hid}{hid}       % hole instance id generator

% for use in Hazel listings
\newcommand{\rar}{$\rightarrow$}
\newcommand{\Rar}{$\Rightarrow$}
\newcommand{\lbd}{$\lambda$}
\newcommand{\heh}[1]{$\hehole^{#1}$}

% for missing references
\newcommand{\todo}[1]{\textbf{[TODO: #1]}}
\newcommand{\todoref}[1]{\textbf{[TODO: need reference(s): #1]}}

% lambda calculi
\newcommand{\ulc}{$\Lambda$}
\newcommand{\stlc}{$\Lambda_\to$}
\newcommand{\gtlc}{$\Lambda_\to^?$}
\newcommand{\gtclc}{$\Lambda_\to^{\langle\tau\rangle}$}

\newcommand{\gtlch}{*}

% structural diffing
\newcommand{\nodiff}[2]{#1\trianglerighteq #2}            % no diff
\newcommand{\nfdiff}[2]{#1\triangleright #2}              % non-fill diff
\newcommand{\fdiff}[4]{#1\blacktriangleright_{#4}^{#3}#2} % fill diff
\newcommand{\sdiff}[4]{#1\ntrianglerighteq_{#4}^{#3}#2}   % some diff
\newcommand{\adiff}[4]{#1\triangleright_{#4}^{#3}#2}      % any diff

\newcommand{\formsame}[2]{#1\sim #2}      % same expression forms
\newcommand{\formdiff}[2]{#1\nsim #2}      % different expression forms

\newcommand{\far}[3]{\llbracket{}#1/#2\rrbracket{}#3} % fill and resume
\newcommand{\dresult}{d_{result}}
\newcommand{\ufill}{u_{fill}}
\newcommand{\dfill}{d_{fill}}

\newcommand{\fenv}{\rho} % fill env

\newcommand{\unev}{\textsf{ uneval}} % uneval judgment

\usepackage{graphicx}
\usepackage{amsmath}
\usepackage{amssymb}
\usepackage{stmaryrd}
\usepackage{setspace}
\usepackage{mdframed}
\usepackage{subcaption}
\usepackage{placeins}
\usepackage[utf8]{inputenc}
\usepackage{latexsym}
\usepackage{csquotes}

% https://www.overleaf.com/learn/latex/Theorems_and_proofs
\usepackage{amsthm}
\usepackage{thmtools}

% https://tex.stackexchange.com/a/72691
\usepackage[hang, flushmargin]{footmisc}
\usepackage{hyperref}

% must be loaded after hyperref
\usepackage{cleveref}

% per Prof. Fontaine
\Crefname{theorem}{Metatheorem}{Metatheorems}

% tikz diagrams
\usepackage{tikz}
\usetikzlibrary{positioning}

% code listings
\usepackage{minted}

% https://tex.stackexchange.com/a/35547
\usepackage{etoolbox}
\AtBeginEnvironment{minted}{\singlespacing}

% https://tex.stackexchange.com/q/627249
\usepackage{fancyvrb}
\makeatletter
% new fancyvrb options "above space" and "below vspace"
\define@key{FV}{above vspace}[\topsep]{\def\FancyVerbAboveVspace{#1}}
\define@key{FV}{below vspace}[\topsep]{\def\FancyVerbBelowVspace{#1}}
% redefine existing option "vspace"
\define@key{FV}{vspace}[\topsep]{%
  \def\FancyVerbAboveVspace{#1}\def\FancyVerbBelowVspace{#1}}
% init
\fvset{vspace}
% patch fancyvrb internals
\xpatchcmd\FV@ListVSpace
  {\@topsepadd=\FancyVerbVspace}
  {\@topsepadd=\FancyVerbAboveVspace}
  {}{\PatchFailed}
\xpatchcmd\FV@EndList
  {\@endparenv}
  {\@topsepadd\dimexpr\@topsepadd-\FancyVerbAboveVspace
                                 +\FancyVerbBelowVspace\relax
   \@endparenv}
  {}{\PatchFailed}
\minted@def@optfv{vspace}
\minted@def@optfv{above vspace}
\minted@def@optfv{below vspace}
\makeatother

%%% Local Variables:
%%% mode: latex
%%% TeX-master: "../thesis/main"
%%% End:
