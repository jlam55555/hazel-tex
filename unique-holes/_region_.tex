\message{ !name(main.tex)}\documentclass{article}

% see readme
\def\OPTIONConf{1}%
\usepackage{joshuadunfield}
\usepackage{llproof}
\input{../latex-includes/macros.tex}

\usepackage{amsmath}
\usepackage{amssymb}
\usepackage{stmaryrd}
\usepackage{hyperref}
\usepackage{minted}

\newtheorem{theorem}{Theorem}

\title{Hole instance renumbering for unique holes}
\author{Jonathan Lam}
\date{2022/01/13}

\begin{document}

\message{ !name(main.tex) !offset(4) }

Adding to the performance issue is the fact that \mintinline{reasonml}|Program.get_result| is called extraneously many times. The program result can be stored in the \mintinline{reason}|Model.t|, and the program does not have to be re-evaluated after every action: it should only be re-evaluated after appropriate edit actions and can be memoized with respect to the \mintinline{reason}|UHExp| program expression.

\section{Existing hole instance numbering scheme}
\message{ !name(main.tex) !offset(11) }

\end{document}

%%% Local Variables:
%%% mode: latex
%%% TeX-master: t
%%% End:
