\documentclass{article}

\usepackage{amsmath}
\usepackage{minted}
\usepackage{geometry}

\DeclareMathOperator{\fix}{fix}

\title{Mutual recursion}
\author{Jonathan Lam}
\date{2022/02/18}

\begin{document}
\maketitle{}

I want to understand mutual recursion for myself, so here it is. Mutual recursion using a pair and fixpoint form.

\begin{minted}{ocaml}
let (a, b) = (fun x -> b x, fun x -> a x)
\end{minted}

\begin{align}
  (a,b) &= \fix a.\fix b.(\lambda x.b\ x,\lambda x.a\ x) \\
        &= \fix b.(\lambda x.b\ x,
          \lambda x.(\fix a.\fix b.(\lambda x.b\ x,\lambda x.a\ x))\ x) \\
        &= (\lambda x.(\fix b.(\lambda x.b\ x,\lambda x.(\fix a.\fix b.(\lambda x.b\ x,\lambda x.a\ x))\ x))\ x,
          \lambda x.(\fix a.\fix b.(\lambda x.b\ x,\lambda x.a\ x))\ x)
\end{align}
or
\begin{align}
  a &= \lambda x.(\fix b.(\lambda x.b\ x,\lambda x.(\fix a.\fix b.(\lambda x.b\ x,\lambda x.a\ x))\ x))\ x \\
  b &= \lambda x.(\fix a.\fix b.(\lambda x.b\ x,\lambda x.a\ x))\ x
\end{align}

\end{document}

%%% Local Variables:
%%% mode: latex
%%% TeX-master: t
%%% End:
