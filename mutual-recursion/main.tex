\documentclass{article}

\usepackage{amsmath}
\usepackage{minted}
% \usepackage{geometry}

\DeclareMathOperator{\fix}{fix}
\DeclareMathOperator{\self}{self}
\DeclareMathOperator{\ife}{if}
\DeclareMathOperator{\true}{T}
\DeclareMathOperator{\false}{F}

\title{Mutual recursion}
\author{Jonathan Lam}
\date{2022/02/18}

\begin{document}
\maketitle{}

\noindent{}
I want to understand mutual recursion for myself, so here it is. Mutual recursion using a pair and fixpoint form. The part I was missing was having the projection out of the tuple. The fixpoint form is described in the following sections of Harper's PFPL: Chapter 10 (Plotkin's PCF) and section 11.3 (Primitive and Mutual Recursion).

\begin{minted}{ocaml}
let rec even x = if x = 0 then true  else odd  (x - 1)
and     odd  x = if x = 0 then false else even (x - 1)
\end{minted}
This translates to the tuple of mutually-dependent functions:
\begin{align}
  e' &\equiv\lambda x.\ife x=0.\{\true|\self.1\ (x-1)\} \\
  o' &\equiv\lambda x.\ife x=0.\{\false|\self.0\ (x-1)\} \\
  \fix_{eo} &=\fix\self.(e',o') \\
  (e,o) &\leftarrow\fix_{eo} \\
     &\Downarrow(\lambda x.\ife x=0.\{\true|\fix_{eo}.1\ (x-1)\},
       \lambda x.\ife x=0.\{\false|\fix_{eo}.0\ (x-1)\}) \\
  e &\leftarrow\lambda x.\ife x=0.\{\true|\fix_{eo}.1\ (x-1)\} \\
  o &\leftarrow\lambda x.\ife x=0.\{\false|\fix_{eo}.0\ (x-1)\}
\end{align}
Now, let us consider the evaluation of \mintinline{ocaml}|even 2|:
\begin{gather}
  e\ 2 \\
  (\lambda x.\ife x=0.\{\true|\fix_{eo}.1\ (x-1)\})\ 2 \\
  \ife 2=0.\{\true|\fix_{eo}.1\ (2-1)\} \\
  \fix_{eo}.1\ 1 \\
  (\lambda x.\ife x=0.\{\true|\fix_{eo}.1\ (x-1)\},
  \lambda x.\ife x=0.\{\false|\fix_{eo}.0\ (x-1)\}).1\ 1 \\
  (\lambda x.\ife x=0.\{\false|\fix_{eo}.0\ (x-1)\})\ 1 \\
  \ife 1=0.\{\false|\fix_{eo}.0\ (1-1)\} \\
  \fix_{eo}.0\ 0 \\
  (\lambda x.\ife x=0.\{\true|\fix_{eo}.1\ (x-1)\},
  \lambda x.\ife x=0.\{\false|\fix_{eo}.0\ (x-1)\}).0\ 0 \\
  (\lambda x.\ife x=0.\{\true|\fix_{eo}.1\ (x-1)\})\ 0 \\
  \ife 0=0.\{\true|\fix_{eo}.1\ (0-1)\} \\
  \true
\end{gather}

% \begin{align}
%   (a,b) &= \fix a.\fix b.(\lambda x.b\ x,\lambda x.a\ x) \\
%         &= \fix b.(\lambda x.b\ x,
%           \lambda x.(\fix a.\fix b.(\lambda x.b\ x,\lambda x.a\ x))\ x) \\
%         &= (\lambda x.(\fix b.(\lambda x.b\ x,\lambda x.(\fix a.\fix b.(\lambda x.b\ x,\lambda x.a\ x))\ x))\ x,
%           \lambda x.(\fix a.\fix b.(\lambda x.b\ x,\lambda x.a\ x))\ x)
% \end{align}
% or
% \begin{align}
%   a &= \lambda x.(\fix b.(\lambda x.b\ x,\lambda x.(\fix a.\fix b.(\lambda x.b\ x,\lambda x.a\ x))\ x))\ x \\
%   b &= \lambda x.(\fix a.\fix b.(\lambda x.b\ x,\lambda x.a\ x))\ x
% \end{align}

\end{document}

%%% Local Variables:
%%% mode: latex
%%% TeX-master: t
%%% End:
