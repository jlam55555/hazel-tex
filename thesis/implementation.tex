\section{Implementation and optimization of FAR}
\label{sec:far_impl}

\subsection{Evaluation with environments}
\label{sec:eval_with_envs}

In the case of Hazel (which does not prioritize speed of evaluation in its implementation, and is not a compiled language), evaluation with (reified) environments offers an additional (performance) benefit over the substitution model: the ability to easily identify (and thus memoize) operations over environments. This is useful for the optimizations described later in this paper.

The implementation of evaluation in Hazel differs from a typical interpreter implementation of evaluation with environments in three regards: we need to account for hole environments; environments are uniquely identified by an identifier for memoization (in turn for optimization); and any closures in the evaluation result should be converted back into plain $\lambda$ abstractions, for reasons that will be discussed later TODOREF.

% TODO: describe why the result from substitution is better than the result from environments

Omar et al. \cite{conf/popl/HazelnutLive19} describes evaluation with the substitution model using a little-step semantics with an evaluation context $\mathcal{E}$, reproduced in TODOREF.

% TODO: reproduce diagram below

TODOREF is an analogous small-step description of the substitution model, also using the little-step semantics.

% TODO: introduce little-step formalization

The Hazel implementation follows a big-step evaluation model, so a big-step formalization is also displayed in TODOREF.

% TODO: introduce big-step formalization

% TODO: describe formalization

% TODO: need justification that the two are equivalent

\subsection{Restructuring hole instance numbering}
\label{sec:restructuring_hole_numbering}

\begin{figure}
  \centering
  \begin{hminted}
let a = |$\hehole_1$| in
let b = |$\hehole_2$| in
let c = |$\hehole_3$| in
let d = |$\hehole_4$| in
let e = |$\hehole_5$| in
let f = |$\hehole_6$| in
let g = |$\hehole_7$| in
|$\hehole_8$|
  \end{hminted}
  \caption{A problematic example for hole renumbering}
  \label{fig:hole_renumbering_problem}
\end{figure}

\subsection{Implementing FAR}
\label{sec:implementing_far}

\subsection{Memoization of recent actions} 
\label{sec:memoization_actions}

\subsection{UI changes for notebook-like editing}
\label{sec:notebook_ui}

%%% Local Variables:
%%% mode: latex
%%% TeX-master: "main"
%%% End:
