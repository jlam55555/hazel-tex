\chapter{Implementation of fill-and-resume}
\label{sec:far_impl}

\section{Motivation}
\label{sec:far-motivation}

\begin{figure}
  \centering
  \inputhminted{far_motivation}
  \caption{A sample program with an expensive calculation stored in a hole's environment}
  \label{fig:far-motivation}
\end{figure}

Consider the program shown in \Cref{fig:far-motivation}. In this program, the calculation of $x=\text{fib }30$ is arbitrarily chosen to represent a computationally-expensive operation. The result of this program is \[
  [f\leftarrow [\varnothing]\lambda x.\{\dots\},x\leftarrow 832040]\hehole^1
\]
Now, if we want to ``fill'' hole 1 with the expression $x+2$, then it would seem extremely wasteful to have to re-compute the value of $x$. After all, the computation remains exactly the same, and the only part that we are changing uses the result of the previous computation. Moreover, we realize that the end result stores the computed value of $x$ in the hole's closure's environment. Rather than filling the expression $x+2$ in the hole in the original program, we may instead fill the hole in the evaluated program result, and ``resume'' evaluation.
\begin{gather*}
  [f\leftarrow [\varnothing]\lambda x.\{\dots\},x\leftarrow 832040](x+2) \\
  832040+2 \\
  832042
\end{gather*}
Here we observe that the generalized closures surrounding holes and other stopped evaluations allow us to capture the environment for future computation.

This is the \textit{fill-and-resume} (\textit{FAR}) operation that is described in Hazelnut Live \cite{conf/popl/HazelnutLive19}. It is described in terms of a substitution-based evaluation semantics, and with respect to its theoretic foundations in contextual modal type theory (CMTT), but does not specify any details with regard to its implementation, such as the extraction of the expression and hole for the fill operation. Hazelnut Live also describes the practical memoization problem with the suggestion to ``cache more than one recent edit state to take full advantage of hole filling'' -- we present a structural diffing\footnote{``Diffing'' taken to mean the action of performing a (structural) diff operation between two edit states.} algorithm that easily allows us to fill a hole from an arbitrary past edit state.

\section{The FAR process}
\label{sec:far-process}

The fill-and-resume process can be broken into the following sequence.

\begin{enumerate}
\item Obtain a previous edit state with which to fill from the model.
\item Determine whether a fill operation is appropriate. If it is not, perform regular evaluation of the program, and do not continue to the following steps.
\item If the fill operation is valid, then obtain the fill parameters (the internal expression to fill, and the hole number from the previous edit state with which to fill).
\item Pre-process the evaluation result to prepare for (re-)evaluation.
\item Re-evaluate.
\item Pos-process the evaluation result for display purposes.
\item Update the model with the evaluation results.
\end{enumerate}

These steps will be described in greater detail in the following sections.

\subsection{Entrypoint to the FAR algorithm}
\label{sec:far-entrypoint}

\todo{memoization alongside Program.get\_program}

\todo{include a diagram of the updated evaluation call graph}

\subsection{Detecting the fill parameters via structural diff}
\label{sec:obtaining-fill-parameters}

Following the notation from \cite{conf/popl/HazelnutLive19}, the fill operation $\far{\dfill}{\ufill}{\dresult}$ indicates the fill of hole $\ufill$ with expression $\dfill$ in the expression $\dresult$. $\dresult$ is the past program result with which to fill. We need a method to determine the fill parameters $\ufill$ and $\dfill$.

\subsubsection{A na\"ive algorithm for detecting fill parameters}
\label{sec:far-detect-naive}

One way to approach the problem of obtaining the hole number and filled expression is at action time. When constructing an expression, we can check if the cursor lies in a hole (either directly in an empty hole, or if there is an ancestor non-empty hole). When deleting an expression, we can check if the cursor lies in a non-empty hole. However, this method is somewhat short-sighted. What happens if we wish to make multiple edits, e.g., fill a hole with the number 12? Then there are two actions, and the second action is not a hole fill action.
\begin{gather*}
  \hehole^1 \\
  1 \\
  12
\end{gather*}
We may remedy this specific case by grouping together consecutive construction actions\footnote{Grouping together of actions is already performed to some level by the undo history, for visual purposes.}. However, we may also consider more complicated edit sequences that involve movement and deletion actions, potentially outside of the hole. Consider the following edit sequence.
\begin{gather*}
  2+3*\hehole^1 \\
  2+\hehole^2*\hehole^1\\
  2+\hehole^1\\
  2+5 \\
  2+(5) \\
  2+\hehole^1*(5) \\
  2+3*(5) \\
  2+3*(5+\hehole^1)
\end{gather*}
The final edit state in this sequence is actually a valid fill of the first edit state of the sequence\footnote{There are actually multiple valid fill operations here. Another valid fill operation occurs between the third edit state and all of the following edit states.}, such that $u=1$ and $d=5+\hehole^1$. However, an algorithm to trace the edit actions to determine that this is a valid fill may be difficult, since there is an incomprehensible mix of construct, delete, and movement edit actions. Even worse, the edits actually go outside the original hole, which likely makes the algorithm intractable. Thus, we wish for a more robust solution that is independent of the edit sequence between two states.

\subsubsection{Structural diffing between two edit states}
\label{sec:far-detect-structural-diff}

Instead of observing the edit action, we may instead attempt to find the root of the difference between any two edit states, and determine if that is the the difference gives valid fill parameters. This has the benefit of being a relatively simple algorithm, while overcoming the limitation of the previous method because it is path-independent.

The structural diff algorithm takes two expressions as input and returns one of three diff judgments\footnote{The following notations for diffing are chosen somewhat arbitrarily. The triangle seems appropriate because it has a variant with an equals bar ($\nodiff{}{}$), as well as a ``no fill'' ($\nfdiff{}{}{}{}$) and ``fill'' variant ($\fdiff{}{}{}{}$). The triangle is also horizontally asymmetric, which mirrors the fact that the diff relation is asymmetric.}. $\nodiff{d_1}{d_2}$ indicates \textit{no diff} between $d_1$ and $d_2$. $\nfdiff{d_1}{d_2}$ indicates a \textit{non-fill diff} from $d_1$ to $d_2$. $\fdiff{d_1}{d_2}{u}{d}$ indicates a \textit{fill diff} of hole $u$ with expression $d$ from $d_1$ to $d_2$.

We also define two shorthand operators for notational convenience. $\sdiff{d_1}{d_2}{u}{d}$ indicates \textit{some (non-empty) diff} from $d_1$ to $d_2$, which may or may not be a fill difference. $\adiff{d_1}{d_2}{u}{d}$ indicates \textit{any diff} (potentially no diff). Both notations are used to avoid writing multiple similar rules, where the only change in the rules is diff judgment type.

\todo{put the following in a figure, and move to another file}

\begin{singlespace}
  \judgbox{\sdiff{d_1}{d_2}{u}{d}}{Some (non-empty) diff between $d_1$ and $d_2$.}
  \begin{mathpar}
    \Infer{SDiffNFDiffSome}{
      \nfdiff{d_1}{d_2}
    }{\sdiff{d_1}{d_2}{\varnothing}{\varnothing}}
    \and
    \Infer{SDiffFDiffSome}{
      \fdiff{d_1}{d_2}{u}{d}
    }{\sdiff{d_1}{d_2}{u}{d}} \\
    \and
    \Infer{SDiffSomeNFDiff}{
      \sdiff{d_1}{d_2}{\varnothing}{\varnothing}
    }{\nfdiff{d_1}{d_2}}
    \and
    \Infer{SDiffSomeFDiff}{
      \sdiff{d_1}{d_2}{u}{d}
    }{\fdiff{d_1}{d_2}{u}{d}}
  \end{mathpar}

  \judgbox{\adiff{d_1}{d_2}{u}{d}}{Any (possibly-empty) diff between $d_1$ and $d_2$.}
  \begin{mathpar}
    \Infer{ADiffNoDiffAny}{
      \nodiff{d_1}{d_2}
    }{\adiff{d_1}{d_2}{\cdot}{\varnothing}}
    \and
    \Infer{ADiffNFDiffAny}{
      \nfdiff{d_1}{d_2}
    }{\adiff{d_1}{d_2}{\varnothing}{\varnothing}}
    \and
    \Infer{ADiffFDiffAny}{
      \fdiff{d_1}{d_2}{u}{d}
    }{\adiff{d_1}{d_2}{u}{d}}
    \and
    \Infer{ADiffAnyNoDiff}{
      \adiff{d_1}{d_2}{\cdot}{\varnothing}
    }{\nodiff{d_1}{d_2}}
    \and
    \Infer{ADiffAnyNFDiff}{
      \sdiff{d_1}{d_2}{\varnothing}{\varnothing}
    }{\nfdiff{d_1}{d_2}}
    \and
    \Infer{ADiffAnyFDiff}{
      \sdiff{d_1}{d_2}{u}{d}
    }{\fdiff{d_1}{d_2}{u}{d}}
  \end{mathpar}
\end{singlespace}

We break up the diff judgments into cases. First, we may consider the case of two holes of different \textit{expression forms}\footnote{We use the term \textit{expression form} or \textit{expression variant} to indicate the variant types of \mintinline{ocaml}|DHExp.t|. For example, empty holes and constants of the base type are different expression forms. Empty holes and non-empty holes are also different forms per the grammar.}.

For convenience, we define the judgment $\formsame{d_1}{d_2}$ to mean that $d_1$ and $d_2$ are of the same expression form\footnote{The relation $\sim$ is already defined to mean type consistency when applied to types. This interpretation applies when the relation is applied to internal expressions.}. We may more concretely express this judgment by the following rules.

\begin{singlespace}
  \begin{mathpar}
    \Infer{FEqConst}{}{\formsame{c_1}{c_2}}
    \and
    \Infer{FEqVar}{}{\formsame{x_1}{x_2}}
    \and
    \Infer{FEqLam}{}{\formsame{\lambda x.d_1}{\lambda x.d_2}}
    \and
    \Infer{FEqAp}{}{\formsame{e_1\ e_2}{e_1'\ e_2'}}
    \and
    \Infer{FEqAsc}{}{\formsame{e:\tau}{e':\tau'}}
    \and
    \Infer{FEqEHole}{}{\formsame{\hehole^u}{\hehole^{u'}}}
    \and
    \Infer{FEqNEHole}{}{\formsame{\hhole{d}^u}{\hhole{d'}^{u'}}}
  \end{mathpar}
\end{singlespace}

If the two expressions have different forms, then the current node is necessarily the diff root. It is a fill diff iff the left expression is a hole.

\begin{singlespace}
  \begin{mathpar}
    \Infer{DFNEqEHole}{
      \formdiff{\hehole^u}{d_2}
    }{\fdiff{\hehole^u}{d_2}{u}{d_2}}
    \and
    \Infer{DFNEqNEHole}{
      \formdiff{\hhole{d}^u}{d_2}
    }{\fdiff{\hhole{d}^u}{d_2}{u}{d_2}}
    \and
    \Infer{DFNEqNonHole}{
      d_1\ne\hehole^u \\
      d_1\ne\hhole{d}^u \\
      \formdiff{d_1}{d_2}
    }{\nfdiff{d_1}{d_2}}
    \and
  \end{mathpar}
\end{singlespace}

If the two expressions have the same form, then we need to check the root node and its subexpression(s). For expressions with no subexpressions, there is a non fill diff iff the expressions differ.

\begin{singlespace}
  \begin{mathpar}
    \Infer{DFEqConstNEq}{c_1\ne c_2}{\nfdiff{c_1}{c_2}}
    \and
    \Infer{DFEqConstEq}{}{\nodiff{c}{c}}
    \and
    \Infer{DFEqVarNEq}{x_1\ne x_2}{\nfdiff{x_1}{x_2}}
    \and
    \Infer{DFEqVarEq}{}{\nodiff{x}{x}}
  \end{mathpar}
\end{singlespace}

For expressions with a single subexpression, we first check if there are any differences, ignoring the subexpression. If there is a difference, then the current node is the non fill diff root. Otherwise, we pass through the diff from the child node.

\begin{singlespace}
  \begin{mathpar}
    \Infer{DFEqLamNEq$_1$}{
      x_1\ne x_2
    }{\nfdiff{\lambda x_1:\tau_1.d_1}{\lambda x_2:\tau_2.d_2}}
    \and
    \Infer{DFEqLamNEq$_2$}{
      \tau_1\ne\tau_2
    }{\nfdiff{\lambda x:\tau_1.d_1}{\lambda x:\tau_2.d_2}}
    \and
    \Infer{DFEqLamEq}{
      \adiff{d_1}{d_2}{u}{d}
    }{\adiff{\lambda x:\tau.d_1}{\lambda x:\tau.d_2}{u}{d}}
    \and
    \Infer{DFEqAscNEq}{
      \tau_1\ne\tau_2
    }{\nfdiff{d_1:\tau_1}{d_2:\tau_2}}
    \and
    \Infer{DFEqAscEq}{
      \adiff{d_1}{d_2}{u}{d}
    }{\adiff{d_1:\tau}{d_2:\tau}{u}{d}}
  \end{mathpar}
\end{singlespace}

The last case to check for non-hole expressions are expressions with more than one subexpression. In this case, we first check if there exist any differences outside the subexpressions, which would result in a non fill diff rooted at the current node. Otherwise, if there are no subexpression diffs, then the result is no diff. If more than one subexpression has a diff, then the diff is a non fill diff rooted at the current node. The last case is when exactly one child has a diff, which would be passed through. This is illustrated below with the binary function application expression form.

\begin{singlespace}
  \begin{mathpar}
    \Infer{DFEqApEq$_1$}{
      \nodiff{d_1}{d_1'} \\
      \nodiff{d_2}{d_2'} \\
    }{\nodiff{d_1\ d_2}{d_1'\ d_2}}
    \and
    \Infer{DFEqApEq$_2$}{
      \sdiff{d_1}{d_1'}{u}{d} \\
      \sdiff{d_2}{d_2'}{u'}{d'} \\
    }{\nodiff{d_1\ d_2}{d_1'\ d_2}}
    \and
    \Infer{DFEqApEq$_3$}{
      \nodiff{d_1}{d_1'} \\
      \sdiff{d_2}{d_2'}{u}{d} \\
    }{\sdiff{d_1\ d_2}{d_1'\ d_2}{u}{d}}
    \and
    \Infer{DFEqApEq$_4$}{
      \sdiff{d_1}{d_1'}{u}{d} \\
      \nodiff{d_2}{d_2'} \\
    }{\sdiff{d_1\ d_2}{d_1'\ d_2}{u}{d}}
  \end{mathpar}
\end{singlespace}

In the minimal $\lambda$-calculus grammars specified for Hazel, the only expression form of plural subexpression arity is function application, but the following description extends to higher numbers of subexpressions (such as the case for \mintinline{ocaml}|case| expressions with arbitrary numbers of rules.

The last case to consider is the comparison of two hole expressions of the same form. The empty hole case is very similar to the nullary subexpression case. The non-empty hole case is very similar to the unary subexpression case, except for a special rule that propagates non-fill diffs upwards to be a fill diff rooted in the current hole. This allows for diffs that are not rooted directly in a hole to be filled in their nearest non-empty hole parent node.

\begin{singlespace}
  \begin{mathpar}
    \Infer{DFEqEHoleNEq}{
      u\ne u'
    }{\fdiff{\hehole^u}{\hehole^{u'}}{u}{\hehole^{u'}}}
    \and
    \Infer{DFEqEHoleEq}{
    }{\nodiff{\hehole^u}{\hehole^u}}
    \and
    \Infer{DFEqNEHoleNEq}{
      u\ne u'
    }{\fdiff{\hhole{d}^u}{\hhole{d'}^{u'}}{u}{\hhole{d'}^{u'}}}
    \and
    \Infer{DFEqNEHoleEq$_1$}{
      \nodiff{d}{d'}
    }{\nodiff{\hhole{d}^u}{\hhole{d'}^u}}
    \and
    \Infer{DFEqNEHoleEq$_2$}{
      \fdiff{d}{d'}{u'}{d''}
    }{\fdiff{\hhole{d}^u}{\hhole{d'}^u}{u'}{d''}}
    \and
    \Infer{DFEqNEHoleEqProp}{
      \nfdiff{d}{d'}
    }{\fdiff{\hhole{d}^u}{\hhole{d'}^u}{u}{\hhole{d'}^u}}
  \end{mathpar}
\end{singlespace}

\todo{metatheorems regarding the diff algorithm}

The diff algorithm begins by performing the structural diff between the elaborated program state of a past edit state $d_{old}$ and the current edit state $d_{cur}$. A FAR operation is only valid if the diff judgment is a fill diff $\fdiff{d_{old}}{d_{cur}}{u}{d}$, which gives us the FAR parameters $u$ and $d$.

\subsubsection{Performance tradeoffs of the two detection algorithms}
\label{sec:far-detect-compare}

In the most na\"ive form, the previous algorithm's efficiency is $O(\log E)$, where $E$ is the number of expression nodes in the program, if we assume that the depth of an expression node is logarithmic with respect to the total number of expression nodes. That algorithm only has to traverse up the ancestors to decide whether the edit lies in a hole.

The structural diff algorithm presented in this section is $O(E)$, since it traverses each node (once) until it finds a difference. However, if one travels backwards multiple edit states, then the cost is $O(SE)$, where $S$ is the number of edit states compared using this algorithm. While this is much more expensive than the previous algorithm, we assume that the program size is relatively small, causing delay only on the order of milliseconds. However, it may be able to find a valid fill-and-resume in many more cases than the previous algorithm, potentially saving a much longer repeated evaluation time.

\subsubsection{Choosing the edit state to fill from}
\label{sec:far-past-edit-states}

Past edit states are stored in an undo history in Hazel, which allows the user to quickly return to previous edit states. The structure of the undo history is complicated and Hazel-specific, and thus not described here. We note that the result of evaluation is not stored alongside the evaluation result, but the evaluation function itself is memoized, so retrieving a previous edit state and re-evaluating the program is typically not expensive.

\todo{note why that memoization won't work well for us anymore}

There are a number of possible design decisions when searching for a valid hole fill. Firstly, one must decide the maximum number of edit states to search: should it be a fixed number of edit states, or should it be given a fixed time budget? Is it best to cache edit states that recently led to a fill operation (\`a la LRU cache)? Is the most recent edit state that leads to a valid fill usually the best candidate, or even a good candidate? Would it be best to allow for user-configurable settings, or perhaps even for the user to manually select the previous edit from which to fill?

\todoref{need to create a future work section for this}

\subsection{Pre-processing the evaluation result for re-evaluation}
\label{sec:far-preprocessing}

Before beginning the re-evaluation, we would like to substitute all instances of the hole in the previous evaluation result $\dresult$ with the substituted expression. Each time a the hole $\ufill$ is encountered, it is replaced with the fill expression $\dfill$. This way, we can simply reinvoke the evaluation function on the past program result and expect it to resume the evaluation.

There are a few nuances here that should be addressed. First of all, we acknowledge another benefit of the generalized closure variant. If the environment was still baked into the hole, and the hole was replaced with an expression, we would need an additional mechanism to remember the hole's environment. This becomes more messy when the hole doesn't lie directly in a hole closure (i.e., if the hole lies outside the evaluation boundary). Closures are simply recursed through in this pre-processing step, and their environments will still be available even when the hole gets replaced with the fill expression.

We note that the pre-processing acts on the un-post-processed previous evaluation result $\dresult$. This is because the internal expression directly from evaluation and the result after post-processing have different properties or invariants. We do not want to invalidate the properties that are expected to be upheld during evaluation, such as the fact that the body of any $\lambda$-abstraction lies outside the evaluation boundary (whereas post-processing will modify function bodies).

Another issue to tackle is the problem that closures were previously considered to be final values, and would not be re-evaluated. Technically, we may re-evaluate closures; since the evaluation function is idempotent, this will not yield the incorrect result, but it is needlessly inefficient. However, we will need to re-evaluate closures during FAR re-evaluation, since the fill expression will necessarily lie within some closure.

\todo{write the previous statement as a metatheorem}

The (re-)evaluation of a closure is given by the rule EClosure.

\begin{singlespace}
  \begin{mathpar}
    \Infer{EClosure}{
      \env\vdash d\Downarrow d'
    }{\env'\vdash[\env]d\Downarrow d'}
  \end{mathpar}
\end{singlespace}

For efficiency reasons, we will only want to re-evaluate all closures in the result exactly once. To do this, we set a flag for the closure that indicates that it should be re-evaluated. This preprocessing step will recurse through $\dresult$ and set the flag to true for all closures. All closures that result from an evaluation judgment will have the flag set to false. The above closure evaluation rule is only for closures with the re-eval flag set to true. Closures with the re-eval flag set to false will act as values and evaluate to themselves, which is the same as the original evaluation behavior described in \Cref{sec:closures_to_lambdas}. We denote closures with the re-eval flag set to false using the established notation for closures $[\env]d$\footnote{This allows previous discussions of closures to remain valid, since they take the interpretation that the re-eval flag is set to false.}, and denote closures with the re-eval flag set to true by $\llbracket\env\rrbracket d$\footnote{Using the double-square bracket notation also reinforces the fact that re-evaluation is tied with fill-and-resume, which also uses double-square brackets.}.

Finally, we may consider the issue of multiple instances of the same hole closure\footnote{Hole closure refers to the connotation from \Cref{sec:instance-illustration}: instances of hole $\ufill$ that share the same (physical) environment.}. If we substitute the hole, then we lose the information about the holes instance, and thus cannot memoize the evaluations of the same hole instance by environment number. A solution to this is to introduce another \mintinline{ocaml}|DHExp.t| variant \mintinline{ocaml}|FillExp(HoleClosureId.t, DHExp.t)| that indicates the hole closure number as well as the expression to fill. This will be denoted using a hole with a subscript $\hhole{d}_i$. The preprocessing expression will fill a hole $\ufill$ with this expression rather than the expression $\dfill$ directly. During evaluation, this data structure will facilitate the memoization of hole closures.

\subsection{Modifications to evaluation to allow for re-evaluation}
\label{sec:re-eval}

Re-evaluation during fill-and-resume is mostly the same as regular evaluation, but now we need to keep in mind the considerations from pre-processing the previous program result.

First of all, closures will have been marked for re-evaluation by the pre-processing step. This means that the closure environment will first be recursively re-evaluated, following by the closure body. This ensures that the entire program result is fully evaluated\footnote{Note that there is now an ``inversion'' of evaluation order, in that we cannot expect the environment to be fully evaluated before it is encountered in a closure. In an ordinary evaluation, we would expect all the bindings in the environment to have been evaluated before they have been stored in the environment.}. This is similar to the EClosure rule, except that now we add the consideration of the re-eval flag to avoid re-evaluating closures more than once, and also recurse evaluation through variable bindings in the bound environment. We introduce a (re-)evaluation judgment for environments, which simply maps the evaluate operation over the bindings of an environment. Closures with the re-eval flag set to false will have the regular evaluation rule.

\begin{singlespace}
  \begin{mathpar}
    \Infer{EEClosure}{
    }{\env'\vdash[\env]d\Downarrow[\env]d}
    \and
    \Infer{EREClosure}{
      \env\Downarrow\env'' \\
      \env''\vdash d\Downarrow d'
    }{\env'\vdash\llbracket\env\rrbracket d\Downarrow d'}
    \and
    \Infer{EREEnv}{
      \env\Downarrow\env' \\
      \varnothing\vdash d\Downarrow d'
    }{\env,x\leftarrow d\Downarrow\env',x\leftarrow d'}
    \and
    \Infer{EREEnvNull}{
    }{\varnothing\Downarrow\varnothing}
  \end{mathpar}
\end{singlespace}

\todo{metatheorem about how all closures from $\dresult$ will re-evaluated (exactly once), assuming that evaluation terminates}

The other difference that we have to deal with are memoizing the evaluation of the filled hole expressions. Per the discussion of the pre-processing step, all filled expressions will exist in a wrapper that indicates the hole number. We may simply memoize the results by that hole number. This requires us to thread some state throughout our evaluation\footnote{Luckily, threading state through evaluation is useful for other purposes as well, such as keeping track of the current \mintinline{ocaml}|EvalEnvId.t|, and keeping track of evaluation state. These are grouped together under one data structure, \mintinline{ocaml}|EvalState.t|, for ease of implementation.}.

\begin{singlespace}
  \begin{mathpar}
    \Infer{EFillMemoNew}{
    }{\env,(\fenv,i\leftarrow d)\vdash\hhole{d'}_i\Downarrow d\dashv\fenv}
    \and
    \Infer{EFillMemoNew}{
      (i\leftarrow d)\not\in\fenv \\
      \env,\fenv\vdash d'\Downarrow d''\dashv\fenv'
    }{\env,\fenv\vdash\hhole{d'}_i\Downarrow d''\dashv\fenv',i\leftarrow d''} \\
    \and
    \Infer{EFillOther}{
      d\ne\hhole{d}_i \\
      \env\vdash d\Downarrow d'
    }{\env,\fenv\vdash d\Downarrow d'\dashv\fenv}
  \end{mathpar}
\end{singlespace}

In these judgments, we introduce a new \textit{fill memoization context} $\fenv:i\mapsto d$, a mapping of hole closure numbers to expressions\footnote{The symbol $\fenv$ was chosen arbitrarily. It is simply the Greek letter before $\env$.}. A new \textit{fill-memoized evaluation judgment} $\env^-,\fenv^-\vdash d^-\Downarrow d'^+\dashv\fenv'^+$ describes the evaluation with memoization of hole fills. When a hole closure number is encountered for the first time, the expression is evaluated normally and added to $\fenv$; otherwise, the evaluated result of the fill is simply looked up from $\fenv$. Note that memoization requires $\fenv$ to be threaded throughout the evaluation, i.e., it is treated both as an input and output of the evaluation judgment.

This fill-memoized evaluation judgment subsumes the normal evaluation judgment. For all non-hole-fill expressions, it performs the normal evaluation judgment and returns the fill memoization context unchanged.

\subsection{Post-processing resumed evaluation}
\label{sec:far-postprocessing}

The postprocessing algorithm remains unchanged from before. Note that an evaluated program result should never include either of the new forms $\hhole{d}_i$ or $\llbracket\env\rrbracket d$; these should all have been encountered and evaluated out. In other words, the evaluation result from re-evaluation during a FAR should be indistinguishable from the evaluation result from a regular evaluation, and thus the postprocessing process is unchanged.

\todo{metatheorems to back up the above}

\section{FAR examples}
\label{sec:far-examples}

\todo{the previous example}

\begin{figure}
  \centering
  \inputhminted{far_introduce_type_error}
  \caption{FAR introducing a static type error}
  \label{fig:far-introduce-type-error}
\end{figure}

\begin{figure}
  \centering
  \inputhminted{far_remove_type_error}
  \caption{FAR removing a static type error}
  \label{fig:far-remove-type-error}
\end{figure}

\begin{figure}
  \centering
  \inputhminted{far_fill_in_hole_env}
  \caption{FAR filling hole in a hole environment}
  \label{fig:far-fill-hole-in-hole-env}
\end{figure}

\begin{figure}
  \centering
  \inputhminted{far_hole_closure_memoization}
  \caption{FAR hole closure memoization}
  \label{fig:far-hole-closure-memoization}
\end{figure}

\subsection{Noteworthy non-examples}
\label{sec:far-nonexamples}

\subsubsection{Dynamic type errors}
\label{sec:far-dynamic-type-errors}

One may wonder if dynamic type errors (cast failure) have any nice relation to fill-and-resume, but it turns out that they are not treated much different than other non-hole expressions. We may not fill a cast failure directly, because it is not in a hole. We may only remove a cast failure if it lies in a non-empty hole that is filled.

We may introduce new cast failures by filling an expression that is assigned to type hole (something of dynamic type). However, this is no different than introducing any well-typed expression and performing normal evaluation.

Thus the only interesting cases of introducing or remoing holes lies in the case of static type errors (non-empty holes) as described previously.

\subsubsection{Infix operators}
\label{sec:far-infix-operators}

Filling holes in an infix operator sequence may not result in a hole fill due to infix operator precedences, despite the initial appearance. Consider the example shown in \Cref{fig:far-infix-operator-fill}. Say we construct a binary plus operator in the hole. One might expect the fill operation to be $\far{\hehole^2+\hehole^1}{1}{\dresult}$. However, we need to be careful: the multiplication operator has a higher precedence than the addition operator, causing the AST to have a different structure outside of where hole 1 was in the original edit state. It may be more clear if we write the latter edit state with its implicit parentheses: \hmintinline{(1 * /\heh2/) + (/\heh1/ * 2)}.

If, however, hole 1 in the original edit state was a parenthesized expression, and the sum expression shape was constructed inside the parenthesized expression, then the AST wouldn't change (as a parenthesized expression has higher precedence than the outer multiplication operation) and it would be a valid fill operation.

\begin{figure}
  \centering
  \begin{singlespace}
    \makebox[\textwidth][c]{
      \begin{subfigure}[b]{0.4\paperwidth}
        \inputhminted{far_infix_operators}
        \caption{Previous edit state}
      \end{subfigure}
      \qquad
      \begin{subfigure}[b]{0.4\paperwidth}
        \inputhminted{far_infix_operators_filled}
        \caption{Filled edit state}
      \end{subfigure}
    }
  \end{singlespace}
  \caption{FAR infix operator fill}
  \label{fig:far-infix-operator-fill}
\end{figure}

\section{Tracking evaluation state}
\label{sec:eval-state}

It may be useful to thread some state throughout the evaluation process. This means that changes to this state by an earlier invocation to evaluate will affect later evaluations.

The concept of keeping state has already appeared in several contexts. For hole postprocessing, the memoized hole closure info keeps track of seen environments (\Cref{sec:renumbering_memoization}). evaluation, we keep track of the next unique environment identifier (\Cref{sec:env-purity}) and memoize the evaluation of filled expressions (\Cref{sec:re-eval}).

This becomes relevant again (with respect to evaluation) because resuming evaluation requires us to store the state after evaluation, and restore the state after evaluation. Adding more state variables also makes the implementation messier by increasing the number of variables to pass to and return from each of the numerous calls to \mintinline{ocaml}|Evaluator.evaluate|.

The nice solution to this is to group together all evaluation state under a single data structure, \mintinline{ocaml}|EvalState.t|. This one state variable is passed around to all calls to evaluate, and is stored to and restored from \mintinline{ocaml}|Result.t|, which stores all information about a program's evaluated result (including the evaluated expression and the hole closure information). Adding state then becomes trivial, because we only have to modify this data structure rather than all of the calls to \mintinline{ocaml}|Evaluator.evaluate|.

For sake of brevity, we will not update the judgments to include this state.

\subsection{Step counting}
\label{sec:step-counting}

With the \mintinline{ocaml}|EvalState.t| data structure, it is now trivial to keep track of evaluation statistics. Sample statistics might include step counting and number of times a particular evaluation trace has been paused and resumed.

Step counting may be implemented simply by incrementing the count on each call to \mintinline{ocaml}|Evaluator.evaluate|. This may be useful to stop long-running executions after a particular number of (possibly user-specified) execution steps in order to prevent program crashes. It is also useful for our purposes to track how efficient fill-and-resume is: during a fill operation, one may observe the difference in the number of steps before and after the resumed evaluation, and compare it to the case if the evaluation had begun from scratch.

\section{Differences from the substitution model}
\label{sec:far-sub}

The FAR operation is much simpler when evaluating with substitution rather than with environments. With substitution, we eliminate the need for pre- and post-processing steps, and dealing with re-evaluating closures. However, this comes at the cost of not being able to memoize repeated instances of the same hole closure. The original description of FAR from \cite{conf/popl/HazelnutLive19} is reproduced below. The equals (=) operator is used to indicate the FAR operation here; it is meant to symbolize the mix of substitution and evaluation that is presented here. The hole closure notation from \cite{conf/popl/HazelnutLive19}, in which the hole closure is represented using a subscript, is also used here, since the substitution model doesn't have the concept of separate closures.

\todo{move this to a figure later}

\begin{singlespace}
  \judgbox{\far{u}{d}{d'}=d''}{$d'''$ is obtained by filling hole $u$ with $d$ in $d'$}

  \begin{mathpar}
    \Infer{FillSubConst}{}{\far{u}{d}{c}=c}
    \and
    \Infer{FillSubVar}{}{\far{u}{d}{x}=x}
    \and
    \Infer{FillSubLam}{
    }{\far{u}{d}{\lambda x:\tau.d'}=\lambda x:\tau.\far{u}{d}{d'}}
    \and
    \Infer{FillSubAp}{}{\far{u}{d}{d_1\ d_2}=(\far{u}{d}{d_1})\ (\far{u}{d}{d_2})}
    \and
    \Infer{FillSubEHole$_1$}{}{\far{u}{d}{\hehole_\env^u}=[\far{u}{d}{\env}]d}
    \and
    \Infer{FillSubEHole$_2$}{
      u\ne u'
    }{\far{u}{d}{\hehole_\env^{u'}}=\hehole_\env^{u'}}
    \and
    \Infer{FillSubNEHole$_1$}{
    }{\far{u}{d}{\hhole{d'}_\env^u}=[\far{u}{d}{\env}]d}
    \and
    \Infer{FillSubNEHole$_2$}{
      u\ne u'
    }{\far{u}{d}{\hhole{d'}_\env^{u'}}=\hhole{\far{u}{d}{d'}}_{\far{u}{d}{\env}}^{u'}}
    \and
    \Infer{FillSubCast}{
    }{\far{u}{d}{d'\langle\tau\Rightarrow\tau'\rangle}=(\far{u}{d}{d'})\langle\tau\Rightarrow\tau'\rangle}
    \and
    \Infer{FillSubFailedCast}{
    }{\far{u}{d}{d'\langle\tau\Rightarrow\tehole\not\Rightarrow\tau'\rangle}=(\far{u}{d}{d'})\langle\tau\Rightarrow\tehole\not\Rightarrow\tau'\rangle}
  \end{mathpar}

  \judgbox{\far{u}{d}{\env}=\env'}{$\env'$ is obtained by filling hole $u$ with $d$ in $\env$}

  \begin{mathpar}
    \Infer{FillSubEnvNull}{}{\far{u}{d}{\varnothing}=\varnothing}
    \and
    \Infer{FillSubEnv}{
      \env'=\far{u}{d}{\env} \\
      d''=\far{u}{d}{d'}
    }{\far{u}{d}{\env,x\leftarrow d'}=\env',x\leftarrow d''}
  \end{mathpar}
\end{singlespace}

What remains the same between the environment model and the substitution model is the need for an algorithm to detect the fill parameters, for which the structural diff algorithm presented in \Cref{sec:far-detect-structural-diff} is still applicable.

\section{FAR for notebook-style editing}
\label{sec:notebook-ui}

\todo{improvements: reproducibility}

\todo{limitations: limited cases for re-evaluation}

\section{Improvements to FAR}
\label{sec:far-improvements}

\todo{move this to future work section}

\todo{need general cleanup of FAR implementation, not very complete}

\todo{optimizing how many edit states to go back: tradeoffs between storage (how many edit states to store) and speed (how much execution time to detect a FAR, how expensive is FAR depending on which edit state; currently the most recent edit state is used, something like a LRU; may want to cache recent edit states that spawned a FAR or group edit states like in the undo history currently)}

\todo{automatic vs manual far detection}

\section{Metatheorems governing FAR}
\label{sec:far-meta}

\todo{filling (from original paper)}

\todo{commutativity}

\todo{all closures will be evaluated exactly once -- this allows holes that have not yet been evaluated in the result to be successfully re-evaluated (assuming program terminates) and maintains the invariant that all closures will be evaluated at the end of evaluation}

\section{Generalized views on non-empty holes and FAR}
\label{sec:generalized-holes-far}

The work performed for FAR leads us to the following nice generalizations of some of the concepts we've encountered through this work.

\todo{perhaps move this to the discussion section, along with generalized holes from Ch4?}

\subsection{Generalized non-empty holes}
\label{sec:generalized-neholes}

\todo{every partially-complete program can be represented as part of a non-empty hole. In particular, top-level program can be considered to be in a non-empty hole to be filled. Also works nicely as a general parent (in HoleClosureInfo) for all holes in a non-complete program}

\subsection{Generalized FAR}
\label{sec:generalized-far}

\todo{every evaluation can be rewritten as a FAR operation, since every program originates from a hole as an initial state, or using the above version where the top-level expression exists in a non-empty hole}

%%% Local Variables:
%%% mode: latex
%%% TeX-master: "main"
%%% End:
