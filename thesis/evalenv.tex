\chapter{Implementing the environment model of evaluation}
\label{sec:env_model_evaluation}

\section{Hazel-specific implementation}
\label{sec:eval_with_envs}

The implementation of evaluation in Hazel differs from a typical interpreter implementation of evaluation with environments in three regards. First, we need to account for hole environments. Secondly, environments are uniquely identified by an identifier for memoization (in turn for optimization). Lastly, any closures in the evaluation result should be converted back into $\lambda$-abstractions, so that the result matches the result from evaluation with substitution.

\subsection{Evaluation rules}
\label{sec:evalenv-rules}

The evaluation model threads a run-time environment $\env$\footnote{The symbol $\env$ was chosen to represent the environment as it was used to represent hole environments in \cite{conf/popl/HazelnutLive19}. The relationship between these two environments will be discussed in \Cref{sec:holeenv_evalenv_connection}.} throughout the evaluation process for variable lookups. This replaces the variable substitution pass when evaluating with the substitution model. An environment is conceptually a mapping $\env:x\mapsto d$, although it will later be augmented to be more amenable to memoization.

The evaluation rules for evaluation using substitution are described in \Cref{sec:hazel-dynamics} and shown in \Cref{chap:reproduced-hazelnut-live-rules}. We present the updates to the rules necessary for evaluation with environments in \Cref{fig:big-step-formal}. Many of the rules are unchanged and are not repeated, especially the evaluation of casts (EFinal, EApCast, ECastId, ECastSucceed, ECastFail, EGround, EExpand). We also note a change to the set of value judgments: function closures $[\env]\lambda x:\tau.d$ are now considered values, and functions $\lambda x:\tau.d$ are not.

\begin{figure}
  \centering
  \begin{mdframed}
    \begin{singlespace}
      \judgbox{E\vdash d\Downarrow d'}{Internal expression $d$ evaluates to $d'$ given environment $E$}

\begin{mathpar}
  \Infer{EvalB-Final}{E\vdash d\textsf{ final}}{E\vdash d\Downarrow d}
  \and
  \Infer{EvalB-Var}{}{E,x\leftarrow d\vdash x\Downarrow d}
  \and
  \Infer{EvalB-Lam}{
    [\hhole{d''}_{\sigma}/\hhole{d''}_{\sigma::E}][\hehole_{\sigma}/\hehole_{\sigma::E}]d=d'
  }{E\vdash(\lambda x:\tau.d)\Downarrow [E](\lambda x:\tau.d')}
  \and
  \Infer{EvalB-Ap$_1$}{
    E\vdash d_1\Downarrow d_1' \\
    d_1'\ne([E']\lambda x:\tau.d) \\
    E\vdash d_2\Downarrow d_2'
  }{E\vdash d_1(d_2)\Downarrow d_1'(d_2')}
  \and
  \Infer{EvalB-Ap$_2$}{
    E\vdash d_1\Downarrow ([E']\lambda x:\tau.d_1') \\
    E\vdash d_2\Downarrow d_2' \\
    E::E',x\leftarrow d_2'\vdash d_1'\Downarrow d
  }{E\vdash d_1(d_2)\Downarrow d}
  \and
  \Infer{EvalB-Let}{
    E\vdash d_2\Downarrow d_2' \\
    E,x\leftarrow d_2'\vdash d_1\Downarrow d
  }{E\vdash\texttt{let }x=d_2\texttt{ in }d_1\Downarrow d}
  \and
  \Infer{EvalB-Op$_1$}{
    E\vdash d_1\Downarrow d_1' \\
    E\vdash d_2\Downarrow d_2' \\
    (d_1\ne\hnum{n_1}\lor d_2\ne\hnum{n_2})
  }{E\vdash d_1+d_2\Downarrow d_1'+d_2'}  
  \and
  \Infer{EvalB-EHole}{}{E\vdash\hehole_{\sigma}\Downarrow\hehole_{\sigma::E}}
  \and
  \Infer{EvalB-NEHole}{
    E\vdash d\Downarrow d'
  }{E\vdash\hhole{d}_{\sigma}\Downarrow\hhole{d'}_{\sigma::E}}
  \and
  \Infer{EvalB-Op$_2$}{
    E\vdash d_1\Downarrow\hnum{n_1} \\
    E\vdash d_2\Downarrow\hnum{n_2}
  }{E\vdash d_1+d_2\Downarrow\hnum{n_1+n_2}}
\end{mathpar}

%%% Local Variables:
%%% mode: latex
%%% TeX-master: "main"
%%% End:

    \end{singlespace}
  \end{mdframed}
  \caption{Updates to evaluation rules for the environment model of evaluation}
  \label{fig:big-step-formal}
\end{figure}

As expected, a $\lambda$-abstraction binds its lexical environment, forming a function closure (ELam). Variables are not eagerly substituted, but rather looked up in the environment when encountered (EVar). On function application, the expression in function position is now expected to evaluate to a function closure. The closure's environment is extended with the binding of the expression in argument position and the body of the function is evaluated (EAp).

There are many additional forms added to Hazel on top of the base Hazelnut grammar, such as pairs, \mintinline{ocaml}|let| expressions, and \mintinline{ocaml}|case| expressions. The extension of the core rules should be straightforward and these extra expression forms will not be described here. There are two exceptions: a description of recursive $\lambda$-abstractions (the fixpoint form) is described in \Cref{sec:rec_impl}, and the issue of failed pattern-matching in \mintinline{ocaml}|let| and \mintinline{ocaml}|case| expressions is described in \Cref{sec:failed_pattern_match}.

\subsection{Evaluation of holes}
\label{sec:holeenv_evalenv_connection}

In the substitution model, there is no evaluation rule for empty holes; they are final. For non-empty holes, evaluation simply recurses into the subexpression. Notably, the hole environment is not evaluated or updated when the hole is reached; rather, it is filled in by substitution when a variable in scope is bound.

In the environment model, we do not have eager substitution passes. Thus, we update the environment when evaluation reaches a hole, simply by setting the hole environment to the current evaluation (lexical) environment. To facilitate this, we originally give holes no environment (denoted $\hehole_\cdot^u$ and $\hhole{d}_\cdot^u$), rather than the identity environment $\text{id}(\Gamma)$. The latter is used for substitution, but is not needed anymore.

Note that, in the updated interpretation, free variables simply are missing from a hole environment. In the substitution case, free variables exist in the environment as the identity substitution $x\leftarrow x$.

We are currently using the subscript notation for hole environments from Hazelnut Live. When discussing the evaluation boundary in \Cref{sec:generalized-closures-eval-boundary}, we will see that it will be more convenient separate the environment from a hole using the notation for generalized closures.

\subsection{Evaluation of recursive functions}
\label{sec:rec_impl}

To handle recursion in a strongly-typed language, we require self-reference. Hazel uses the fixpoint operator from System PCF. The static and dynamic semantics of the fixpoint operator are described in \Cref{sec:stlc}. In Hazel, the fixpoint does not exist in the external language, but is inserted automatically by the elaboration process when a $\lambda$-abstraction is bound to a variable\footnote{The current implementation of Hazel only allows for recursion in a limited number of cases. The elaboration process only inserts a fixpoint operator for a type-annotated $\lambda$-abstraction. This does not allow for mutual recursion, which could be implemented with fixpoint operators applied to pairs of functions. This may change in future versions of Hazel.}.

We wish to introduce self-reference when evaluating with environments. Perhaps the simplest way is to use memory references (pointers), in which environments may recursively refer to themselves. We eschew this solution because we wish to keep the purity of the Hazel implementation. We discuss purity of implementation in \Cref{sec:env-purity}.

We explore two pure solutions. The first is to eliminate the fixpoint by using recursive data structures in OCaml, which simplifies evaluation but creates slight differences in the postprocessed result. The latter is to adapt the fixpoint to the environment model of evaluation. Either method is viable; however, the latter is chosen because it is closer to the original implementation and we do not have the postprocessing issue.

The updated evaluation rules for fixpoint evaluation are shown in \Cref{fig:fixpoint-rules}. The evaluation of a fixpoint performs a side-effect on the body expression, which is expected to evaluate to a function closure. Namely, we add a self-reference to the closure environment (EFix). When a variable that stores a fixpoint is looked up, then the fixpoint is unwound (EUnwind). If a variable is not a fixpoint, the regular EVar rule applies.

This understanding is consistent with the understanding of fixpoints when using substitution. Fixpoints are unwound when they are encountered, but when evaluating with environments this may occur during a variable lookup.

\begin{figure}
  \centering
  \begin{mdframed}
    \begin{singlespace}
      \begin{mathpar}
  \Infer{EFix}{
    \env\vdash d\Downarrow [\env']d'
  }{
    \env\vdash\fix f:\tau.d\Downarrow [\env,f\leftarrow\fix f:\tau.[\env']d']d'
  } \\
  \and
  \Infer{EVar}{d\ne\fix f:\tau.d'}{\env,x\leftarrow d\vdash x\Downarrow d}
  \and
  \Infer{EUnwind}{
    \env\vdash\fix f:\tau.d\Downarrow d'
  }{\env,x\leftarrow\fix f:\tau.d\vdash x\Downarrow d'}
\end{mathpar}

%%% Local Variables:
%%% mode: latex
%%% TeX-master: "main"
%%% End:

    \end{singlespace}
  \end{mdframed}
  \caption{Evaluation of fixpoints with the environment model}
  \label{fig:fixpoint-rules}
\end{figure}

We may avoid the fixpoint form by using mutually-recursive data structures, so that a closure may contain an environment which contains itself as a binding. This is easy to implement in a language with pointers or mutable references. Mutually-recursive data in OCaml is somewhat tricky in the general case, as it requires statically-constructive forms\footnote{\href{https://ocaml.org/manual/letrecvalues.html}{\S10.1: Recursive definitions of values} of the OCaml reference describes this in greater detail. Simply put, this prevents recursive variables from being defined as arguments to functions, instead only allowing recursive forms to be arguments to data constructors.}. In the more general case of mutual recursion, this would likely make implementation very tricky, and it would be more practical to use impure \mintinline{ocaml}|ref|s to achieve self-reference. However, for the simple case of a non-mutually-recursive function, we may statically construct the mutual recursion using the rule shown in \Cref{fig:rec-closures-let}.

\begin{figure}
  \centering
  \begin{mdframed}
    \begin{singlespace}
      \begin{mathpar}
        \Infer{ERecClosure}{
          \env'=\env,f\leftarrow d_1' \\
          d_1'=[\env']\lambda x.d_1 \\
          \env'\vdash d_2\Downarrow d
        }{
          \env\vdash\texttt{let }f=\lambda x.d_1\texttt{ in }d_2\Downarrow d
        }
      \end{mathpar}
    \end{singlespace}
  \end{mdframed}
  \caption{Evaluation rule for recursion using self-recursive data structures}
  \label{fig:rec-closures-let}
\end{figure}

Using the recursive environment in closures simplifies evaluation and may lead to a slight uptick in performance, due to the elimination of unwinding steps for fixpoints. However, it complicates the display of recursive functions in the context inspector and structural equality checking, due to infinite recursion. The first problem is re-introducing the \mintinline{ocaml}|FixF| form during postprocessing (\Cref{sec:generalized-closures-eval-boundary}) by detecting recursive environments and converting them to \mintinline{ocaml}|FixF| expressions. The second problem is solved by the fast equality checker for memoized environments described in \Cref{sec:fast-equals}, which is useful even for non-recursive environments. We may also say that using recursive data structures without mutable \mintinline{ocaml}|ref|s is limited by the language limitations, necessitating workarounds even for the simply-recursive case, and potentially much more complicated workarounds for the mutual recursion case.

There is a nuance that may cause the postprocessed\footnote{Postprocessing will be discussed in \Cref{sec:postprocessing-substitution}.} result to slightly differ from that using the fixpoint form. To illustrate this, consider the simple program in \Cref{fig:rec-closures-issue}. The result will be a closure of hole 1 with the identifiers \mintinline{ocaml}|x| and \mintinline{ocaml}|f| in scope. When evaluating using the fixpoint expression, the binding for \mintinline{ocaml}|f| will be the expression $(\fix f.[\varnothing]\lambda x.\hehole^1)$, and the binding for \mintinline{ocaml}|x| is $([f\leftarrow \fix f.[\varnothing]\lambda x.\hehole^1]\lambda x.\hehole^1)$. \mintinline{ocaml}|f| is bound to the closure in the EFix rule, and \mintinline{ocaml}|x| is bound during EAp to the evaluated value of \mintinline{ocaml}|f|.

\begin{listing}
  \centering
  \inputhminted{missing_fixf}
  \caption{Illustrating the problem with postprocessing with recursive closures}
  \label{fig:rec-closures-issue}
\end{listing}

However, when evaluating with a recursive data structure, both \mintinline{ocaml}|x| and \mintinline{ocaml}|f| refer to the same value $d=([f\leftarrow d]\lambda x.\hehole^1)$. It is impossible to discern the two and decide where to begin the ``start of the recursion'', i.e., to determine that \mintinline{ocaml}|f| should be a fixpoint expression and \mintinline{ocaml}|x| should be a $\lambda$-abstraction, at least without significant additional extra effort. Thus to remove the recursion, we may arbitrarily decide that the outermost recursive form should be a $\lambda$-abstraction and set the recursive binding in its environment to be a fixpoint expression, which will successfully remove the recursion but mistakenly change some expressions that would be fixpoint forms to $\lambda$-expressions. This distinction may not be critical, but it will at least confuse the user. This justifies our use of the fixpoint form for evaluating recursive functions.

\section{The evaluation boundary and general closures}
\label{sec:generalized-closures-eval-boundary}

Evaluation with the environment model lazily substitutes variables. Evaluation steps that require the environment (e.g., evaluation of holes and variables) are only performed when evaluation reaches the expression of interest. Evaluation with the substitution model eagerly substitutes using a separate substitution pass. In the evaluation result or in the Hazel context inspector, the user may examine expressions in the internal language. The user expects to see fully substituted values, and closures should not appear directly to the user. For example, bound variables in a function body (other than the function argument) should show the captured expression.

In other words, any unevaluated expression must be ``caught up'' to the substituted equivalent after evaluation. This requires that the environment be stored alongside the unevaluated expression, and that a postprocessing step should be taken to perform the substitution and discard the stored environment. Note that this is essentially performing a substitution pass after evaluation, but is preferred over substitution during evaluation because it is only performed on the evaluation result (rather than all the intermediate expressions during evaluation). This is called \textit{substitution postprocessing}, and will be discussed in \Cref{sec:postprocessing-substitution}.

We first define the \textit{evaluation boundary} to be the conceptual distinction between expressions for which evaluation has reached (``inside'' the boundary), and for those that remain unevaluated (``outside'' the boundary). This definition will be useful for describing the postprocessing algorithm.

\subsection{Evaluation of failed pattern matching using generalized closures}
\label{sec:failed_pattern_match}

There are two cases where an expression in the evaluation result may lie outside the evaluation boundary\footnote{A third case will appear in \Cref{sec:far_impl} when we discuss the fill-and-resume optimization.}. The first is in the body of a $\lambda$-abstraction. A $\lambda$-abstraction evaluates to a closure, and thus captures an environment with it. The second case is that of an unmatched \mintinline{ocaml}|let| or \mintinline{ocaml}|case| expression (in which the scrutinee matches none of the rules), for which the body expression(s) will remain unevaluated in the result without an associated environment\footnote{There is a third place where pattern-matching may fail: the pattern of an applied $\lambda$-abstraction may not match its argument. However, this is not an issue since functions are already captured in a closure.}. Pattern-matching is not part of Hazelnut Live or in this paper because pattern-matching is not a primary concern of either of these works. However, it is a practical concern with Hazel that arises from the introduction of evaluation with environments.

We solve this by introducing (lexical) \textit{generalized closures}, the product of an arbitrary expression and its lexical environment. Traditionally, the term ``closure'' refers to \textit{function closures}, which are the product of a $\lambda$-abstraction with its lexical environment. Hazelnut Live introduces \textit{hole closures}, which are the product of hole environments with their lexical environments, and are fundamental to the Hazel live environment. Hole environments allow a user to inspect a hole's environment in the context inspector, and enable the fill-and-resume optimization described in \Cref{sec:far_impl}. We propose generalizing the term ``closures'' to the definition stated above. Conceptually, all generalized closures represent a partial or stopped evaluation (using the environment model), as well as the state (the environment) that may be used to resume the evaluation. Similar to the evaluation of function closures, closures are final (boxed) values and evaluate to themselves.

The application of generalized closures to the problem of unevaluated \mintinline{ocaml}|let| or \mintinline{ocaml}|case| bodies is straightforward: if there is a failed pattern match, wrap the entire expression in a (generalized) closure with the current lexical environment. Then, the postprocessing can successfully perform the substitution.

\subsection{Generalization of existing hole types}
\label{sec:generalized-closures-datatypes}

Consider the abbreviated definition of the internal expression variant type in \Cref{fig:generalized-closures-datatypes}. In \Cref{fig:ungeneralized-datatypes} the previous implementation is shown (when evaluating using the substitution model), augmented with a type for function closures. In this version, each expression variant that requires an environment has the environment hardcoded into the variant. In \Cref{fig:generalized-datatypes} the proposed version with generalized closures is shown. The \mintinline{ocaml}|Lam|, \mintinline{ocaml}|Let|, and \mintinline{ocaml}|Case| variants are unchanged. Importantly, the environments are removed from the hole types and a new generalized \mintinline{ocaml}|Closure| is introduced. In this model, a hole, $\lambda$-abstraction, unmatched \mintinline{ocaml}|let|, or unmatched \mintinline{ocaml}|case| expression is wrapped in the \mintinline{ocaml}|Closure| variant when evaluated.

\begin{figure}
  \centering
  \begin{singlespace}
    \makebox[\textwidth][c]{
      \begin{subfigure}[b]{0.4\paperwidth}
        \inputominted{ungeneralized_closures}
        \caption{Non-generalized closures}
        \label[figure]{fig:ungeneralized-datatypes}
      \end{subfigure}
      \qquad
      \begin{subfigure}[b]{0.4\paperwidth}
        \inputominted{generalized_closures}
        \caption{Generalized closures}
        \label[figure]{fig:generalized-datatypes}
      \end{subfigure}
    }
  \end{singlespace}
  \caption{Comparison of internal expression datatype definitions (in module \mintinline{ocaml}|DHExp|) for non-generalized and generalized closures.}
  \label{fig:generalized-closures-datatypes}
\end{figure}

The notation used to express a function closure may be extended to all generalized closure types. In particular, the environment for a hole changes from the initial notation used in Hazelnut Live to a notation similar to function closures, shown in \Cref{fig:generalized-closure-notation}.

\begin{figure}
  \centering
  \begin{singlespace}
    \begin{align}
      \label{eq:generalized-closure-notation}
      &[\env]\lambda x.d\tag{function closure} \\
      &[\env]\hhole{d}^u\tag{hole closure} \\
      &[\env](\texttt{let }x=d_1\texttt{ in }d_2)\tag{closure around \texttt{let}} \\
      &[\env](\texttt{case }x\texttt{ of }\text{rules})\tag{closure around \texttt{case}}
    \end{align}
  \end{singlespace}
  \caption{Revised notation for generalized closure}
  \label{fig:generalized-closure-notation}
\end{figure}

This implementation of closures is an improvement in three ways. Firstly, it simplifies the variant types by factoring out the environment, separating the ``core'' expression from the environment coupled with it. Secondly, it allows for a more intuitive understanding of holes in the environment model of evaluation. This solves the question of what environment to initialize a hole with when it is created during the elaboration phase: a hole is simply initialized without a hole environment, much as a function closure is initially without an environment (a plain syntactical $\lambda$ abstraction). This removes the need for the awkward notation $\hhole{d}_\cdot^u$ introduced earlier to indicate a hole that has not yet been assigned an environment. Lastly, generalized closures play an important role in the fill-and-resume operation, in which (unevaluated) closures can contain arbitrary subexpressions and allow ``resuming'' evaluation in the stored environment.

Note that while the generalized closures for the body expressions of $\lambda$ abstractions, unmatched \mintinline{ocaml}|let| expressions, and unmatched \mintinline{ocaml}|case| expressions represent expressions outside of the evaluation boundary, the expressions within non-empty holes (which also are bound to a hole closure) lie within the evaluation boundary. This shows the two goals that generalized closures achieve; to encapsulate a stopped expression (which is used during postprocessing to perform substitution), and to encapsulate an expression to be filled for the fill-and-resume operation.

\subsection{Formalizing the evaluation boundary}
\label{sec:eval-boundary-metatheorem}

We may characterize the evaluation boundary with two theorems on the evaluated result. First, we need to define three auxiliary judgments.

The $d^-\unev$ unevaluated judgment shown in \Cref{fig:uneval} indicates that an expression form directly contains an unevaluated expression, or may contain an unevaluated expression at some point in the future. In Hazelnut Live, function bodies are the only unevaluated expressions. We note also that hole expressions are also considered $\unev$; this is because if they are filled in fill-and-resume, they will contain an unevaluated expression. If we consider Hazel's more complete syntax, then additional axioms should be added for unmatched \mintinline{ocaml}|let| and \mintinline{ocaml}|case| statements. UENotFinal provides an intuitive characteristic of the $\unev$ judgment: unevaluated expressions are not final (evaluated). It is trivial to prove UENotFinal from the three axioms and from the $\textsf{final}$ judgment.

\begin{figure}
  \centering
  \begin{mdframed}
    \begin{singlespace}
      \judgbox{d\unev}{$d$ contains an unevaluated subexpression not in a closure}

\begin{mathpar}
  \Infer{UELam}{}{\lambda x:\tau.d\unev}
  \and
  \Infer{UEEHole}{}{\hehole^u\unev}
  \and
  \Infer{UENEHole}{}{\hhole{d}^u\unev}
  \and
  \Infer{UENotFinal}{d\unev}{d\textsf{ not final}}
\end{mathpar}

%%% Local Variables:
%%% mode: latex
%%% TeX-master: "main"
%%% End:

    \end{singlespace}
  \end{mdframed}
  \caption{Unevaluated judgment}
  \label{fig:uneval}
\end{figure}

Two subexpression judgments are shown in \Cref{fig:subexpression}. $d^-\subseteq d'^-$ indicates that a $d$ is a subexpression of $d'$, not recursing into hole environments. $d\in d'$ also indicates a subexpression, but allows recursing into closure environments. This distinction is important because all environments lie inside the evaluation boundary.

\begin{figure}
  \centering
  \begin{mdframed}
    \begin{singlespace}
      \judgbox{d\subseteq d'}{$d$ is a subexpression of $d'$}
\begin{mathpar}
  \Infer{SEId}{}{d\subseteq d}
  \and
  \Infer{SELam}{
    d\subseteq d'
  }{d\subseteq \lambda x:\tau.d'}
  \and
  \Infer{SEAp$_1$}{
    d\subseteq d_1
  }{d\subseteq d_1\ d_2}
  \and
  \Infer{SEAp$_2$}{
    d\subseteq d_2
  }{d\subseteq d_1\ d_2}
  \and
  \Infer{SENEHole}{
    d\subseteq d'
  }{d\subseteq\hhole{d'}^u}
  \and
  \Infer{SEClosure}{
    d\subseteq d'
  }{d\subseteq[\env]d'}
\end{mathpar}

\judgbox{d\in d'}{$d$ exists in $d'$}
\begin{mathpar}
  \Infer{SEDirect}{
    d\subseteq d'
  }{d\in d'}
  \and
  \Infer{SEEnv}{
    d\in\env
  }{d\in [\env]d'}
\end{mathpar}

\judgbox{d\in\env}{$d$ exists in $\env$}
\begin{mathpar}
  \Infer{SEEnv$_1$}{
    d\in\env
  }{d\in\env,x\leftarrow d'}
  \and
  \Infer{SEEnv$_2$}{
    d\in d'
  }{d\in\env,x\leftarrow d'}
\end{mathpar}

%%% Local Variables:
%%% mode: latex
%%% TeX-master: "main"
%%% End:

    \end{singlespace}
  \end{mdframed}
  \caption{Subexpression judgment}
  \label{fig:subexpression}
\end{figure}

\begin{theorem}[Evaluation boundary]
  If $\varnothing\vdash d\Downarrow d'$ and $d''\in d'$ and $d''\textsf{ uneval}$, then $d''\subseteq[\env]d'''\in d'$.
  \label{thm:eval-boundary-1}
\end{theorem}

\Cref{thm:eval-boundary-1} states that all unevaluated subexpressions lie within a closure in the evaluated result. This theorem ensures that the postprocessing substitution and fill-and-resume will have the necessary information to succeed. The justification is straightforward by switching on the evaluation rules. The evaluation rules for $\lambda$-abstractions and hole expressions wrap the expression in a closure.

\begin{theorem}[Singular evaluation boundary]
  If $\varnothing\vdash d\Downarrow d'$ and $d''\in d'$ and $[\env]d''\in d'$ and $[\env]d''\subset[\env']d'''$, then $d'''=\hhole{d''''}^u$.
  \label{thm:eval-boundary-2}
\end{theorem}

\Cref{thm:eval-boundary-2} states that there are no nested closures in an expression (not recursing into closure environments). This means that unevaluated expression are always separated from evaluated expressions by a single closure, justifying the term ``evaluation boundary'' rather than ``evaluation boundaries.'' The only case when closures may be nested are in the case of non-empty holes, in which the hole expression actually lies within the evaluation boundary\footnote{We similarly need to consider the scrutinee of unmatched \mintinline{ocaml}|let| and \mintinline{ocaml}|case| statements, which lies within the evaluation boundary.}. We justify this theorem in the same manner by using induction on the evaluation rules, keeping track of nested closures and recognizing that the outer closure(s) must be around non-empty holes.

\subsection{Alternative strategies for evaluation past the \\ evaluation boundary}
\label{sec:alt_strat_unevaluated}

Without generalized closures, unevaluated expressions (body expressions of $\lambda$-abstractions, unmatched \mintinline{ocaml}|let| expressions, and unmatched \mintinline{ocaml}|case| expressions) may be filled by a modified form of evaluation, which is only different in that a failed lookup (due to in-scope but yet-unbound variables) will leave the variable unchanged\footnote{Ordinarily, a lookup on a \mintinline{ocaml}|BoundVar| (a variable which is in scope) should never fail during evaluation, and thus throws an exception during evaluation.}. However, this is essentially the same as substitution, and is expensive to do during evaluation. Also, while this speculative execution would be reasonable for \mintinline{ocaml}|let| expressions, it would be highly undesirable for \mintinline{ocaml}|case| expression, where it is easy to imagine an example where speculative execution of all branches leads to infinite recursion.

Another way to eliminate the case of unmatched expressions is to introduce an exhaustiveness checker to Hazel; then, we can guarantee (at run-time) that a pattern will never fail to match. This would also require changing the semantics of pattern holes, which always fail to match; the behavior may be changed so that pattern holes always match, but do not introduce new bindings. Since the focus of this work is not on patterns, these ideas were not explored and are left for future work in the Hazel project.

\subsection{Pattern matching for closures}
\label{sec:generalized-closures-matching}

Pattern matching is not the primary focus of this work, but it warrants a brief discussion here. Since we introduce a new \mintinline{ocaml}|DHExp.t| variant, we also need to implement all the methods that switch on a \mintinline{ocaml}|DHExp.t|, such as pattern matching.

Pattern matching is implemented in the function \mintinline{ocaml}|Evaluator.matches|, which has type \mintinline{ocaml}|(DHPat.t, DHExp.t) => Evaluator.match_result|. If pattern matching succeeds, then an environment containing the matched binding(s) will be returned. Otherwise, pattern matching may be indeterminate (if either the pattern or bound expression is indeterminate), or it may fail. Note that the expression passed to \mintinline{ocaml}|Evaluator.matches| is already evaluated.

Closures are a unique variant of \mintinline{ocaml}|DHExp.t| in that they are a container type, whose contained expression determines its behavior during pattern matching. An evaluated closure\footnote{An evaluated closure is one for which the \mintinline{ocaml}|re_eval| flag introduced in \Cref{sec:far-preprocessing} is false. Thus far, all closures we have encountered are evaluated.} may only contain one of four types of expressions: $\lambda$-abstractions, holes, unmatched \mintinline{ocaml}|let| expressions, or unmatched \mintinline{ocaml}|case| expressions. The former is a boxed value and should match against variables only, and otherwise fail. The latter three are indeterminate and should match against variables and return an indeterminate match otherwise.

Another way to understand this behavior is to consider the updated \textsf{final} judgment once closures have been taken into account. These updated final judgments are shown in \Cref{fig:update-final-judgment}. Function closures are values (VFunClosure). Other general closures are indeterminate (IClosure). The IClosure rule subsumes the old IEHole and INEHole rules in \Cref{sec:hazelnut-live-final-judgment}. FClosure states that all closures are final expressions\footnote{See the previous comment. Re-evaluatable closures will no longer be final.}. The derivation of FClosure from VFunClosure and IClosure is trivial.

\begin{figure}
  \centering
  \begin{mdframed}
    \begin{singlespace}
      \begin{mathpar}
  \Infer{VFunClosure}{
  }{[\env]\lambda x:\tau.d\textsf{ val}}
  \and
  \Infer{IClosure}{
    d\ne\lambda x:\tau.d
  }{[\env]d\textsf{ indet}}
  \and
  \Infer{FClosure}{
  }{[\env]d\textsf{ final}}
\end{mathpar}

%%% Local Variables:
%%% mode: latex
%%% TeX-master: "main"
%%% End:

    \end{singlespace}
  \end{mdframed}
  \caption{Updates to final judgments with generalized closures}
  \label{fig:update-final-judgment}
\end{figure}

\section{The postprocessing substitution algorithm ($\pplc$)}
\label{sec:postprocessing-substitution}

The substitution postprocessing process aims to perform substitution on expressions that lie outside the evaluation boundary in the evaluation result (an internal expression). The algorithm works in two stages: first inside the evaluation boundary, and then proceeding outside the boundary when a closure is encountered.

The symbol chosen to denote postprocessing is $\pplc$\footnote{The choice of symbol is somewhat arbitrary, but we may read it as ``reverting'' some expressions generated by and useful for evaluation (i.e., closures) to a more context-inspector-friendly form, which is in some sense the opposite of evaluation ($\Downarrow$). The bracket subscript indicates that this post-processing step is intended to remove closure expressions.}. The two stages of this algorithm will be denoted $\pplco$ and $\pplct$, respectively.

\subsection{Substitution within the evaluation boundary ($\pplco$)}
\label{sec:postprocessing-subst-inside}

When inside the evaluation boundary, all (bound) variables have been looked up and all hole environments assigned, so we do not need to perform any substitution. The main point of this step is to recurse through the expression until a closure is found, at which point we enter the second stage and perform substitution.

For expressions without subexpressions, the expression is returned unchanged; there is nothing to do. For other non-closure expression types, $\pplco$ recurses through any subexpressions.

For closure types, we first need to recusively apply $\pplco$ to all bindings in the closure environment. For non-empty holes, the body is inside the evaluation boundary and thus $\pplco$ is applied. For other expressions inside a closure, the body expression is outside the evaluation boundary, and thus $\pplct$ is applied to the body expression, using the closure environment. The closure is then removed.

A $\lambda$-abstraction, \mintinline{ocaml}|let| expression, \mintinline{ocaml}|case| expression, or hole outside of a closure, or a bound variable that has not been looked up, will never exist outside of a closure within the evaluation boundary, so these cases need not be handled.

Note that in the implementation with recursive data structures used to represent environments as described in \Cref{sec:rec_impl}, an additional step must be taken before recursing into function closures. Recursive function bindings must be detected and converted to \mintinline{ocaml}|FixF| expressions to prevent infinite recursion.

\subsection{Substitution outside the evaluation boundary ($\pplct$)}
\label{sec:postprocessing-subst-outside}

When outside the evaluation boundary (and inside a closure), we need to substitute bound variables\footnote{The wording is a little tricky here, since there are the \mintinline{ocaml}|BoundVar| and \mintinline{ocaml}|FreeVar| internal expression variants, which refer to variables which are in scope or not in scope. However, we may only substitute variables which are in-scope (\mintinline{ocaml}|BoundVar|) and bound; some instances may not yet be bound.} and assign an environment to holes.

Bound variables are looked up in the environment; this lookup may fail if the variable does not exist in the environment, in which case the variable is left unchanged. For other primary expressions, the expression is left unchanged. When a hole is encountered, its environment is the closure environment\footnote{There is nothing to do at this point for hole closures. The hole closure numbering step will assign a closure identifier to the hole as described in the second postprocessing algorithm in \Cref{sec:two-stage-renumber}.}. A closure will never exist outside the evaluation boundary in the evaluation result (by \Cref{thm:eval-boundary-2}).

Note that the $\pplco$ algorithm only takes an internal expression $d$ as its input, whereas the $\pplct$ algorithm takes an internal expression $d$ and a (closure) environment $\env$ as inputs.

% TODO: example programs:
% - lambda and fix forms
% - holes inside and outside boundary
% - recursion through hole environments

\begin{figure}
  \centering
  \begin{mdframed}
    \begin{singlespace}
      \judgbox{\env\vdash d\pplc d'}{$d$ postprocesses ($\lambda$-conversion) to $d'$ outside the evaluation boundary}

\begin{mathpar}
  \Infer{\pplclo-Value}{
    d\textsf{ value} \\
    d\ne\lambda x.d
  }{d\pplc d}
  \and
  \Infer{\pplclo-Var}{}{\env,x\leftarrow d\vdash x\pplc d}
  \and
  \Infer{\pplclo-Fix}{
    \env\vdash d\pplc d'
  }{\env\vdash\fix f.d\pplc\fix f.d'}
  \and
  \Infer{\pplclo-Lam}{
    \env\vdash d\pplc d'
  }{\env\vdash\lambda x.d\pplc\lambda x.d'}
  \and
  \Infer{\pplclo-Ap}{
    \env\vdash d_1\pplc d_1' \\
    \env\vdash d_2\pplc d_2'
  }{\env\vdash d_1(d_2)\pplc d_1'(d_2')}
  \and
  \Infer{\pplclo-Op}{
    \env\vdash d_1\pplc d_1' \\
    \env\vdash d_2\pplc d_2'
  }{\env\vdash d_1+d_2\pplc d_1'+d_2'}
  \and
  \Infer{\pplclo-EHole}{
  }{\env\vdash\hehole_\varnothing^u\pplc\hehole_\env^u}
  \and
  \Infer{\pplclo-NEHole}{
    \env\vdash d\pplc d'
  }{\env\vdash\hhole{d}_\varnothing^u\pplc\hhole{d'}_\env^u}
\end{mathpar}

\judgbox{d\pplc d'}{$d$ postprocesses ($\lambda$-conversion) to $d'$ within the evaluation boundary}

\begin{mathpar}
  \Infer{\pplcl-Value}{
    d\textsf{ value} \\
    d\ne\fix f.d \\
    d\ne [\env]\lambda x.d
  }{d\pplc d}\\
  \and
  \Infer{\pplcl-Fix}{
    \env\vdash d\pplc d'\\
    \env,f\leftarrow (\fix f.\lambda x.d')\vdash d'\pplc d''
  }{\fix f.([\env]\lambda x.d)\pplc \lambda x.d''}
  \and
  \Infer{\pplcl-Closure}{
    \env\vdash d\pplc d'
  }{[\env]\lambda x.d\pplc\lambda x.d'}
  \and
  \Infer{\pplcl-Ap}{
    d_1\pplc d_1' \\
    d_2\pplc d_2'
  }{d_1(d_2)\pplc d_1'(d_2')}
  \and
  \Infer{\pplcl-Op}{
    d_1\pplc d_1' \\
    d_2\pplc d_2'
  }{d_1+d_2\pplc d_1'+d_2'}
  \and
  \Infer{\pplcl-EHole}{
    \env'=\{(x\leftarrow d'):(x\leftarrow d)\in\env,d\pplc d'\}
  }{\hehole_\env^u\pplc\hehole_{\env'}^u}
  \and
  \Infer{\pplcl-NEHole}{
    d\pplc d' \\
    \env'=\{(x\leftarrow d'):(x\leftarrow d)\in\env,d\pplc d'\}
  }{\hhole{d}_\env^u\pplc\hhole{d'}_{\env'}^u}
\end{mathpar}

TODO: closure needs to go recursive

%%% Local Variables:
%%% mode: latex
%%% TeX-master: "main"
%%% End:

    \end{singlespace}
  \end{mdframed}
  \caption{Substitution postprocessing}
  \label{fig:big-step-inside-formal}
\end{figure}

We may try to characterize the result of the substitution process slightly more formally. \Cref{thm:substitution-postprocessing} describes how the substitution postprocessing algorithm removes closures in the result. Notably, this states that a closure exists in the result if and only if the closure's expression is a hole. This is consistent with what we expect in Hazelnut Live, where closures did not exist outside of hole closures. We can justify this by performing induction on the postprocessing rules; it is clear that all closures are eliminated using the \pplcl{}Closure rule, and closures are only introduced using the \pplclo{}EHole and \pplclo{}NEHole rules.

\Cref{thm:eval-correctness} states that the postprocessed result of evaluation with environments is the same as the result of evaluation with substitution, as presented in Hazelnut Live. We provide an intuitive justification for this. First, we check that the evaluation rules for the environment model are correct, and this is easy due the similarity to the evaluation rules for evaluation with substitution and evaluation with environments. The difference in the result lies only outside the evaluation boundary: holes and variables may not be correctly bound outside the environment boundary. We then perform induction using the postprocessing rules to ensure that variables and hole environments are properly looked up outside the evaluation boundary. \Cref{thm:eval-boundary-1} states that the postprocessing algorithm will have the necessary environments to perform the substitution pass. \Cref{thm:substitution-postprocessing} affirms that there will be no stray closures remaining in the result except hole closures, as non-hole closures did not exist in evaluation with substitution.

\begin{theorem}[Substitution postprocessing closures]
  If $\env\vdash d\Downarrow d'$ and $d'\pplc d''$, then:
  \begin{enumerate}
  \item If $[\env]d'''\in d''$, then $d'''=\hehole^u$ or $d'''=\hhole{d}^u$.
  \item If $d'''=\hehole^u$ or $d'''=\hhole{d}^u$, then $d'''\subset[\env]d'''$.
  \end{enumerate}
  \label{thm:substitution-postprocessing}
\end{theorem}

\begin{theorem}[Evaluation with environments correctness]
  Let $\Downarrow_s$ be the evaluation semantics described by Hazelnut Live \cite{conf/popl/HazelnutLive19}. Then if $\env\vdash d\Downarrow d'$ and $d'\pplc d''$, and if $d\Downarrow_s d'''$, then $d''=d'''$.
  \label{thm:eval-correctness}
\end{theorem}

\subsection{Post-processing memoization}
\label{sec:memoization}

There is repeated postprocessing if the same closure environment is encountered multiple times in the evaluation result. If we can identify and look up environments, then we can memoize their postprocessing.

\subsubsection{Modifications to the environment datatype}
\label{sec:memoization-evalenv}

Memoization of environments requires a unique key for each environment. The existing environment type \mintinline{ocaml}|Environment.t| is a map $\env:x\mapsto d$. We introduce a new environment type \mintinline{ocaml}|EvalEnv.t|\footnote{This is the name in the current implementation (due to this environment type being specialized for evaluation), but perhaps a better name is \mintinline{ocaml}|MemoEnv.t|.} that is the product of an identifier and the variable map $\env:(\text{id}_\env,x\mapsto d)$, in which $\text{id}_\env$ indicates a unique environment identifier.

To ensure that there is a bijection between environment identifiers and environments, a new unique identifier must be generated each time an environment is extended. An instance of \mintinline{ocaml}|EvalEnvIdGen.t| is used to generate a new unique identifier, and is required as an additional argument to functions in the \mintinline{ocaml}|EvalEnv| module that modify the environment\footnote{
  In the same manner as \mintinline{ocaml}|MetaVarGen.t|, \mintinline{ocaml}|EvalEnvId.t| is implemented as type \mintinline{ocaml}|int| and \mintinline{ocaml}|EvalEnvIdGen.t| is implemented as a simple counter. To keep the implementation pure, the instance of \mintinline{ocaml}|EvalEnvIdGen.t| needs to be threaded through all calls of \mintinline{ocaml}|Evaluator.evaluate| to avoid a global mutable state, and is discussed in \Cref{sec:env-purity}.
}.

Note that while physical identity may be used to distinguish between different environments, it is difficult to use for efficient lookups due to the abstraction of pointers in a high-level language like OCaml or Javascript. We may think of numeric identifiers (in general) as high-level pointers. We may state this property of environment identifiers as \Cref{thm:env-id}, which allows us to use environment identifiers as a key for environments.

\begin{theorem}[Use of $\text{id}_\env$ as an identifier]
  The mapping $i_\env:\sigma\mapsto\text{id}_\env$ that maps an environment (identified up to physical equality) to its assigned environment identifier is a bijection.
  \label{thm:env-id}
\end{theorem}

We justify this by the construction of environment identifiers. $\env_i\ne\env_j$ implies that there is a series of modified environments $\{\env_i,\env_{i+1},\dots,\env_{j-1},\env_j\}$ (without loss of generality, assume $\env_i$ is an earlier environment than $\env_j$). By construction, each element of the set $\{i_\env(\env_i),i_\env(\env_{i+1}),\dots,i_\env(\env_j)\}$ is unique. Thus $i_\env(\env_1)\ne i_\env(\env_2)$.

\subsubsection{Modifications to the post-processing rules}
\label{sec:memoization-postprocessing}

During substitution postprocessing ($\pplc$), a mapping $\text{id}_\env\mapsto\env$ stores the set of substituted (postprocessed) environments. Upon encountering a closure in the evaluation result, it is looked up in this map. If it is found, the stored result is used. If it is not found, the environment is recursively substituted by applying $\pplco$ to each binding.

\section{Implementation considerations}
\label{sec:evalenv_impl_considerations}

This section details various design decisions and tradeoffs of the current implementation; some parts of this may require an understanding of the hole closure numbering postprocessing step described in \Cref{sec:renumbering}.

\subsection{Data structures}
\label{sec:data-structures}

As is common in functional programming, the most common data structures used are (linked) lists and maps (binary search trees). The standard library modules \mintinline{ocaml}|List| and \mintinline{ocaml}|Map| are used for these. In particular, the original implementation uses linked-lists for the implementation of environments, and we have not modified this decision. In Hazel, the hole closure storage data structures \mintinline{ocaml}|HoleClosureInfo_.t| and \mintinline{ocaml}|HoleClosureInfo.t| use a combination of maps and lists.

The only major change to the data structures is the switch from using linked lists (\mintinline{ocaml}|VarMap.t|) as the backing store for environments to using a binary search tree representation (\mintinline{ocaml}|VarBstMap.t|). This improves performance of operations on large environments.

Hashtables were not used at all in the implementation; their effect on performance is unknown and is reserved for future work. While they allow for amortized $O(1)$ operations, they are stateful and thus difficult to copy, and do not allow for the structural sharing memory optimization. Since immutable data structures are efficiently copied each time they are modified, the costs of introducing hashtables will likely outweigh the costs.

\subsection{Additional constraints due to hole closure numbering}
\label{sec:difficulties-hole-numbering}

\Cref{sec:hole-numbering-algorithm} introduces another postprocessing algorithm, which may be combnied with substitution postprocessing. The introduction of hole closure parents in \Cref{sec:closure-parents} makes closure memoization more difficult for environments in non-hole closures. In particular, adding a new parent to a hole requires that the hole postprocessing (the hole closure numbering operation) be re-run on a hole. Memoizing the hole prevents a hole closure in an environment from being assigned multiple closure parents. To get around this, we propose a modified memoization routine in \Cref{sec:unification-postprocessing} that only postprocesses environments when a hole or variable is reached, rather than when a closure is reached in postprocessing.

\subsection{Storing evaluation results versus internal expressions}
\label{sec:result-vs-dhexp}

The evaluation takes as input an internal expression and returns the evaluated internal expression along with a \textsf{final} judgment (either \mintinline{ocaml}|BoxedValue| or \mintinline{ocaml}|Indet|).

The decision should be made whether to store this \textsf{final} judgment in the environment\footnote{In other words, we need to decide whether \mintinline{ocaml}|EvalEnv.t| should be a mapping from variables to \mintinline{ocaml}|EvalEnv.result| (including \textsf{final} judgment) or from variables to \mintinline{ocaml}|DHExp.t|.}. Storing the judgment allows us to simply use the stored value directly during evaluation, but requires much boxing and unboxing in other cases (e.g., during postprocessing). On the other hand, not storing the judgment is cleaner when used outside of evaluation, but requires recalculation of the \textsf{final} judgment during evaluation upon lookup\footnote{Recalculating the \textsf{final} judgment means re-evaluating the expression upon variable lookup, since the \mintinline{ocaml}|Evaluator.evaluate| function currently performs the evaluation and \textsf{final} judgments. This should not be an expensive operation since the value should already be final and cannot make any evaluation steps, but still may require several calls to evaluate.}. The decision is somewhat arbitrary but may have small effects on the evaluation performance and elegance of implementation.

%%% Local Variables:
%%% mode: latex
%%% TeX-master: "main"
%%% End:
