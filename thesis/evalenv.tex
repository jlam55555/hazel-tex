\section{Implementing the environment model of evaluation}
\label{sec:env_model_evaluation}

\subsection{Hazel-specific implementation}
\label{sec:eval_with_envs}

In the case of Hazel (which does not prioritize speed of evaluation in its implementation, and is not a compiled language), evaluation with (reified) environments offers an additional (performance) benefit over the substitution model: the ability to easily identify (and thus memoize) operations over environments. This is useful for the optimizations described later in this paper.

The implementation of evaluation in Hazel differs from a typical interpreter implementation of evaluation with environments in three regards: we need to account for hole environments; environments are uniquely identified by an identifier for memoization (in turn for optimization); and any closures in the evaluation result should be converted back into plain $\lambda$ abstractions.

\subsubsection{Evaluation rules}
\label{sec:evalenv-rules}

% TODO: walk through simple example

Omar et al. \cite{conf/popl/HazelnutLive19} describes evaluation with the substitution model using a little-step semantics with an evaluation context $\mathcal{E}$. 

The Hazel implementation follows a big-step model for evaluation, which is simpler, more performant, and does not require the evaluation context. Thus it is more convenient to follow a big-step semantics as shown in \Cref{fig:big-step-formal}. An equivalent small-step semantics is described in \Cref{sec:small-step-evalenv} but will not be discussed further.

The evaluation model threads a run-time environment $\env$\footnote{The symbol $\env$ was chosen to represent the environment as it was used to represent hole environments in \cite{conf/popl/HazelnutLive19}. The relationship between these two environments will be discussed in \Cref{sec:holeenv_evalenv_connection}.} throughout the evaluation process. An environment is conceptually a mapping $\env:x\mapsto d$, although it will later be augmented to be more amenable to memoization.

% TODO: closure conversion
% TODO: move discussion of closure conversion to background section

% TODO: introduce big-step formalization
\begin{figure}
  \centering
  \begin{mdframed}
    \begin{singlespace}
      \judgbox{E\vdash d\Downarrow d'}{Internal expression $d$ evaluates to $d'$ given environment $E$}

\begin{mathpar}
  \Infer{EvalB-Final}{E\vdash d\textsf{ final}}{E\vdash d\Downarrow d}
  \and
  \Infer{EvalB-Var}{}{E,x\leftarrow d\vdash x\Downarrow d}
  \and
  \Infer{EvalB-Lam}{
    [\hhole{d''}_{\sigma}/\hhole{d''}_{\sigma::E}][\hehole_{\sigma}/\hehole_{\sigma::E}]d=d'
  }{E\vdash(\lambda x:\tau.d)\Downarrow [E](\lambda x:\tau.d')}
  \and
  \Infer{EvalB-Ap$_1$}{
    E\vdash d_1\Downarrow d_1' \\
    d_1'\ne([E']\lambda x:\tau.d) \\
    E\vdash d_2\Downarrow d_2'
  }{E\vdash d_1(d_2)\Downarrow d_1'(d_2')}
  \and
  \Infer{EvalB-Ap$_2$}{
    E\vdash d_1\Downarrow ([E']\lambda x:\tau.d_1') \\
    E\vdash d_2\Downarrow d_2' \\
    E::E',x\leftarrow d_2'\vdash d_1'\Downarrow d
  }{E\vdash d_1(d_2)\Downarrow d}
  \and
  \Infer{EvalB-Let}{
    E\vdash d_2\Downarrow d_2' \\
    E,x\leftarrow d_2'\vdash d_1\Downarrow d
  }{E\vdash\texttt{let }x=d_2\texttt{ in }d_1\Downarrow d}
  \and
  \Infer{EvalB-Op$_1$}{
    E\vdash d_1\Downarrow d_1' \\
    E\vdash d_2\Downarrow d_2' \\
    (d_1\ne\hnum{n_1}\lor d_2\ne\hnum{n_2})
  }{E\vdash d_1+d_2\Downarrow d_1'+d_2'}  
  \and
  \Infer{EvalB-EHole}{}{E\vdash\hehole_{\sigma}\Downarrow\hehole_{\sigma::E}}
  \and
  \Infer{EvalB-NEHole}{
    E\vdash d\Downarrow d'
  }{E\vdash\hhole{d}_{\sigma}\Downarrow\hhole{d'}_{\sigma::E}}
  \and
  \Infer{EvalB-Op$_2$}{
    E\vdash d_1\Downarrow\hnum{n_1} \\
    E\vdash d_2\Downarrow\hnum{n_2}
  }{E\vdash d_1+d_2\Downarrow\hnum{n_1+n_2}}
\end{mathpar}

%%% Local Variables:
%%% mode: latex
%%% TeX-master: "main"
%%% End:

    \end{singlespace}
  \end{mdframed}
  \caption{Big-step semantics for the environment model of evaluation}
  \label{fig:big-step-formal}
\end{figure}

% TODO: describe formalization
% TODO: describe the implementation

\subsubsection{Connection to hole environments}
\label{sec:holeenv_evalenv_connection}

\subsubsection{Differences to the substitution model of evaluation}
\label{sec:differences_subst}

\subsection{The evaluation boundary and post-processing}
\label{sec:closures_to_lambdas}


% TODO: describe why the result from substitution is better than the result from environments

% TODO: example programs:
% - lambda and fix forms
% - holes inside and outside boundary
% - recursion through hole environments

% two-stage approach

\begin{figure}
  \centering
  \begin{mdframed}
    \begin{singlespace}
      \judgbox{\env\vdash d\pplc d'}{$d$ postprocesses ($\lambda$-conversion) to $d'$ outside the evaluation boundary}

\begin{mathpar}
  \Infer{\pplclo-Value}{
    d\textsf{ value} \\
    d\ne\lambda x.d
  }{d\pplc d}
  \and
  \Infer{\pplclo-Var}{}{\env,x\leftarrow d\vdash x\pplc d}
  \and
  \Infer{\pplclo-Fix}{
    \env\vdash d\pplc d'
  }{\env\vdash\fix f.d\pplc\fix f.d'}
  \and
  \Infer{\pplclo-Lam}{
    \env\vdash d\pplc d'
  }{\env\vdash\lambda x.d\pplc\lambda x.d'}
  \and
  \Infer{\pplclo-Ap}{
    \env\vdash d_1\pplc d_1' \\
    \env\vdash d_2\pplc d_2'
  }{\env\vdash d_1(d_2)\pplc d_1'(d_2')}
  \and
  \Infer{\pplclo-Op}{
    \env\vdash d_1\pplc d_1' \\
    \env\vdash d_2\pplc d_2'
  }{\env\vdash d_1+d_2\pplc d_1'+d_2'}
  \and
  \Infer{\pplclo-EHole}{
  }{\env\vdash\hehole_\varnothing^u\pplc\hehole_\env^u}
  \and
  \Infer{\pplclo-NEHole}{
    \env\vdash d\pplc d'
  }{\env\vdash\hhole{d}_\varnothing^u\pplc\hhole{d'}_\env^u}
\end{mathpar}

\judgbox{d\pplc d'}{$d$ postprocesses ($\lambda$-conversion) to $d'$ within the evaluation boundary}

\begin{mathpar}
  \Infer{\pplcl-Value}{
    d\textsf{ value} \\
    d\ne\fix f.d \\
    d\ne [\env]\lambda x.d
  }{d\pplc d}\\
  \and
  \Infer{\pplcl-Fix}{
    \env\vdash d\pplc d'\\
    \env,f\leftarrow (\fix f.\lambda x.d')\vdash d'\pplc d''
  }{\fix f.([\env]\lambda x.d)\pplc \lambda x.d''}
  \and
  \Infer{\pplcl-Closure}{
    \env\vdash d\pplc d'
  }{[\env]\lambda x.d\pplc\lambda x.d'}
  \and
  \Infer{\pplcl-Ap}{
    d_1\pplc d_1' \\
    d_2\pplc d_2'
  }{d_1(d_2)\pplc d_1'(d_2')}
  \and
  \Infer{\pplcl-Op}{
    d_1\pplc d_1' \\
    d_2\pplc d_2'
  }{d_1+d_2\pplc d_1'+d_2'}
  \and
  \Infer{\pplcl-EHole}{
    \env'=\{(x\leftarrow d'):(x\leftarrow d)\in\env,d\pplc d'\}
  }{\hehole_\env^u\pplc\hehole_{\env'}^u}
  \and
  \Infer{\pplcl-NEHole}{
    d\pplc d' \\
    \env'=\{(x\leftarrow d'):(x\leftarrow d)\in\env,d\pplc d'\}
  }{\hhole{d}_\env^u\pplc\hhole{d'}_{\env'}^u}
\end{mathpar}

TODO: closure needs to go recursive

%%% Local Variables:
%%% mode: latex
%%% TeX-master: "main"
%%% End:

    \end{singlespace}
  \end{mdframed}
  \caption{Big-step semantics for $\lambda$-conversion post-processing}
  \label{fig:big-step-inside-formal}
\end{figure}

\subsection{A strict evaluation boundary}
\label{sec:strict_eval_boundary}

% TODO: example program: pattern doesn't match

\subsection{Post-processing memoization}
\label{sec:memoization}

\subsubsection{Modifications to the environment datatype}
\label{sec:memoization-evalenv}

\subsubsection{Modifications to the post-processing rules}
\label{sec:memoization-postprocessing}

\begin{figure}
  \centering
  \begin{mdframed}
    \begin{singlespace}
      \judgbox{TODO}{TODO}

\begin{mathpar}
  TODO
\end{mathpar}

%%% Local Variables:
%%% mode: latex
%%% TeX-master: "main"
%%% End:

    \end{singlespace}
  \end{mdframed}
  \caption{Big-step semantics modifications for environment memoization}
  \label{fig:big-step-memoization-rules}
\end{figure}

\subsection{Purity}
\label{sec:env-purity}

% threading around state: technically still pure, represented very similar to rules
% evalenv number generation is threaded around very similar to metavargen, but
% somewhat unwieldy and 

\subsubsection{Elegance and complexity of implementation}
\label{sec:elegance-and-complexity}

% purity is a technical term, elegance is not; unwieldiness
% is the additional complexity worth it?
% - nice benefits with memoization in specific cases (will see later)
% - general linear speedup even with the overhead and without benefits of memoization
% (e.g., see Fibonacci example)

%%% Local Variables:
%%% mode: latex
%%% TeX-master: "main"
%%% End:
