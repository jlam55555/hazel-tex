\section{Implementing the environment model of evaluation}
\label{sec:env_model_evaluation}

\subsection{Hazel-specific implementation}
\label{sec:eval_with_envs}

In the case of Hazel (which does not prioritize speed of evaluation in its implementation, and is not a compiled language), evaluation with (reified) environments offers an additional (performance) benefit over the substitution model: the ability to easily identify (and thus memoize) operations over environments. This is useful for the optimizations described later in this paper.

The implementation of evaluation in Hazel differs from a typical interpreter implementation of evaluation with environments in three regards: we need to account for hole environments; environments are uniquely identified by an identifier for memoization (in turn for optimization); and any closures in the evaluation result should be converted back into plain $\lambda$ abstractions, for reasons that will be discussed later TODOREF.

% TODO: describe why the result from substitution is better than the result from environments

\subsubsection{Evaluation rules}
\label{sec:evalenv-rules}

Omar et al. \cite{conf/popl/HazelnutLive19} describes evaluation with the substitution model using a little-step semantics with an evaluation context $\mathcal{E}$, reproduced in TODOREF.

% TODO: reproduce diagram below

\Cref{fig:small-step-formal} is an analogous small-step description of the substitution model, also using the little-step semantics.

% TODO: introduce little-step formalization
\begin{figure}
  \centering
  \begin{mdframed}
    \begin{singlespace}
      \judgbox{\env\vdash d\to d'}{Internal expression $d$ takes an instruction transition to $d'$ given environment $\env$}

\begin{mathpar}
  \Infer{EvalS-Var}{}{\env,x\leftarrow d\vdash x\to d}\\
  \and
  \Infer{EvalS-Lam}{}{\env\vdash(\lambda x:\tau.d)\to ([\env]\lambda x:\tau.d')}
  \and
  \Infer{EvalS-Fix}{\env\vdash d\to d'}{\env\vdash\fix f.d\to \fix f.d'}
  \and
  \Infer{EvalS-Ap$_1$}{
    \env\vdash d_2\textsf{ final} \\
    \env',x\leftarrow d_2\vdash d_1\to d_1'
  }{\env\vdash ([\env']\lambda x:\tau.d_1)(d_2)\to ([\env']\lambda x:\tau.d_1')(d_2)}
  \and
  \Infer{EvalS-Ap$_2$}{
    \env\vdash d_2\textsf{ final} \\
    \env',x\leftarrow d_2\vdash d_1\textsf{ final}
  }{\env\vdash([\env']\lambda x:\tau.d_1)(d_2)\to d_1}
  \and
  \Infer{EvalS-Ap$_3$}{
    \env\vdash d_2\textsf{ final} \\
    \env',f\leftarrow ([\env']\lambda x:\tau.d_1),x\leftarrow d_2\vdash d_1\to d_1'
  }{\env\vdash\fix.([\env']\lambda x:\tau.d_1)(d_2)\to\fix.([\env']\lambda x:\tau.d_1')(d_2)}
  \and
  \Infer{EvalS-Ap$_4$}{
    \env\vdash d_2\textsf{ final} \\
    \env',f\leftarrow([\env']\lambda x:\tau.d_1),x\leftarrow d_2\vdash d_1\textsf{ final}
  }{\env\vdash\fix.([\env']\lambda x:\tau.d_1)(d_2)\to d_1}
  \and
  \Infer{EvalS-Let$_1$}{
    \env\vdash d_2\textsf{ final} \\
    \env,x\leftarrow d_2\vdash d_1\to d_1'
  }{\env\vdash\texttt{let }x=d_2\texttt{ in }d_1\to\texttt{let }x=d_2\texttt{ in }d_1'}
  \and
  \Infer{EvalS-Let$_2$}{
    \env\vdash d_2\textsf{ final} \\
    \env,x\leftarrow d_2\vdash d_1\textsf{ final}
  }{\env\vdash\texttt{let }x=d_2\texttt{ in }d_1\to d_1}\\
  \and
  \Infer{EvalS-Op}{}{\env\vdash\hnum{n_1}+\hnum{n_2}\to\hnum{n_1+n_2}} \\
  \and
  \Infer{EvalS-EHole}{}{\env\vdash\hehole_{\varnothing}^u\to\hehole_{\env}^u}
  \and
  \Infer{EvalS-NEHole}{\env\vdash d\textsf{ final}}{\env\vdash\hhole{d}_{\varnothing}^u\to\hhole{d}_{\env}^u}
\end{mathpar}

%%% Local Variables:
%%% mode: latex
%%% TeX-master: "main"
%%% End:

    \end{singlespace}
  \end{mdframed}
  \caption{Small-step semantics for the environment model of evaluation}
  \label{fig:small-step-formal}
\end{figure}

The Hazel implementation follows a big-step evaluation model, so a big-step formalization is also displayed in \Cref{fig:big-step-formal}.

% TODO: introduce big-step formalization
\begin{figure}
  \centering
  \begin{mdframed}
    \begin{singlespace}
      \judgbox{E\vdash d\Downarrow d'}{Internal expression $d$ evaluates to $d'$ given environment $E$}

\begin{mathpar}
  \Infer{EvalB-Final}{E\vdash d\textsf{ final}}{E\vdash d\Downarrow d}
  \and
  \Infer{EvalB-Var}{}{E,x\leftarrow d\vdash x\Downarrow d}
  \and
  \Infer{EvalB-Lam}{
    [\hhole{d''}_{\sigma}/\hhole{d''}_{\sigma::E}][\hehole_{\sigma}/\hehole_{\sigma::E}]d=d'
  }{E\vdash(\lambda x:\tau.d)\Downarrow [E](\lambda x:\tau.d')}
  \and
  \Infer{EvalB-Ap$_1$}{
    E\vdash d_1\Downarrow d_1' \\
    d_1'\ne([E']\lambda x:\tau.d) \\
    E\vdash d_2\Downarrow d_2'
  }{E\vdash d_1(d_2)\Downarrow d_1'(d_2')}
  \and
  \Infer{EvalB-Ap$_2$}{
    E\vdash d_1\Downarrow ([E']\lambda x:\tau.d_1') \\
    E\vdash d_2\Downarrow d_2' \\
    E::E',x\leftarrow d_2'\vdash d_1'\Downarrow d
  }{E\vdash d_1(d_2)\Downarrow d}
  \and
  \Infer{EvalB-Let}{
    E\vdash d_2\Downarrow d_2' \\
    E,x\leftarrow d_2'\vdash d_1\Downarrow d
  }{E\vdash\texttt{let }x=d_2\texttt{ in }d_1\Downarrow d}
  \and
  \Infer{EvalB-Op$_1$}{
    E\vdash d_1\Downarrow d_1' \\
    E\vdash d_2\Downarrow d_2' \\
    (d_1\ne\hnum{n_1}\lor d_2\ne\hnum{n_2})
  }{E\vdash d_1+d_2\Downarrow d_1'+d_2'}  
  \and
  \Infer{EvalB-EHole}{}{E\vdash\hehole_{\sigma}\Downarrow\hehole_{\sigma::E}}
  \and
  \Infer{EvalB-NEHole}{
    E\vdash d\Downarrow d'
  }{E\vdash\hhole{d}_{\sigma}\Downarrow\hhole{d'}_{\sigma::E}}
  \and
  \Infer{EvalB-Op$_2$}{
    E\vdash d_1\Downarrow\hnum{n_1} \\
    E\vdash d_2\Downarrow\hnum{n_2}
  }{E\vdash d_1+d_2\Downarrow\hnum{n_1+n_2}}
\end{mathpar}

%%% Local Variables:
%%% mode: latex
%%% TeX-master: "main"
%%% End:

    \end{singlespace}
  \end{mdframed}
  \caption{Big-step semantics for the environment model of evaluation}
  \label{fig:big-step-formal}
\end{figure}

% TODO: describe formalization

% TODO: need justification that the two are equivalent

% TODO: describe the implementation

\subsection{The evaluation boundary and post-processing}
\label{sec:closures_to_lambdas}

% two-stage approach

\begin{figure}
  \centering
  \begin{mdframed}
    \begin{singlespace}
      \judgbox{\env\vdash d\pplc d'}{$d$ postprocesses ($\lambda$-conversion) to $d'$ outside the evaluation boundary}

\begin{mathpar}
  \Infer{\pplclo-Value}{
    d\textsf{ value} \\
    d\ne\lambda x.d
  }{d\pplc d}
  \and
  \Infer{\pplclo-Var}{}{\env,x\leftarrow d\vdash x\pplc d}
  \and
  \Infer{\pplclo-Fix}{
    \env\vdash d\pplc d'
  }{\env\vdash\fix f.d\pplc\fix f.d'}
  \and
  \Infer{\pplclo-Lam}{
    \env\vdash d\pplc d'
  }{\env\vdash\lambda x.d\pplc\lambda x.d'}
  \and
  \Infer{\pplclo-Ap}{
    \env\vdash d_1\pplc d_1' \\
    \env\vdash d_2\pplc d_2'
  }{\env\vdash d_1(d_2)\pplc d_1'(d_2')}
  \and
  \Infer{\pplclo-Op}{
    \env\vdash d_1\pplc d_1' \\
    \env\vdash d_2\pplc d_2'
  }{\env\vdash d_1+d_2\pplc d_1'+d_2'}
  \and
  \Infer{\pplclo-EHole}{
  }{\env\vdash\hehole_\varnothing^u\pplc\hehole_\env^u}
  \and
  \Infer{\pplclo-NEHole}{
    \env\vdash d\pplc d'
  }{\env\vdash\hhole{d}_\varnothing^u\pplc\hhole{d'}_\env^u}
\end{mathpar}

\judgbox{d\pplc d'}{$d$ postprocesses ($\lambda$-conversion) to $d'$ within the evaluation boundary}

\begin{mathpar}
  \Infer{\pplcl-Value}{
    d\textsf{ value} \\
    d\ne\fix f.d \\
    d\ne [\env]\lambda x.d
  }{d\pplc d}\\
  \and
  \Infer{\pplcl-Fix}{
    \env\vdash d\pplc d'\\
    \env,f\leftarrow (\fix f.\lambda x.d')\vdash d'\pplc d''
  }{\fix f.([\env]\lambda x.d)\pplc \lambda x.d''}
  \and
  \Infer{\pplcl-Closure}{
    \env\vdash d\pplc d'
  }{[\env]\lambda x.d\pplc\lambda x.d'}
  \and
  \Infer{\pplcl-Ap}{
    d_1\pplc d_1' \\
    d_2\pplc d_2'
  }{d_1(d_2)\pplc d_1'(d_2')}
  \and
  \Infer{\pplcl-Op}{
    d_1\pplc d_1' \\
    d_2\pplc d_2'
  }{d_1+d_2\pplc d_1'+d_2'}
  \and
  \Infer{\pplcl-EHole}{
    \env'=\{(x\leftarrow d'):(x\leftarrow d)\in\env,d\pplc d'\}
  }{\hehole_\env^u\pplc\hehole_{\env'}^u}
  \and
  \Infer{\pplcl-NEHole}{
    d\pplc d' \\
    \env'=\{(x\leftarrow d'):(x\leftarrow d)\in\env,d\pplc d'\}
  }{\hhole{d}_\env^u\pplc\hhole{d'}_{\env'}^u}
\end{mathpar}

TODO: closure needs to go recursive

%%% Local Variables:
%%% mode: latex
%%% TeX-master: "main"
%%% End:

    \end{singlespace}
  \end{mdframed}
  \caption{Big-step semantics for $\lambda$-conversion post-processing}
  \label{fig:big-step-inside-formal}
\end{figure}

\subsection{A strict evaluation boundary}
\label{sec:strict_eval_boundary}

\subsection{Post-processing memoization}
\label{sec:memoization}

\subsubsection{Modifications to the environment datatype}
\label{sec:memoization-evalenv}

\subsubsection{Modifications to the post-processing rules}
\label{sec:memoization-postprocessing}

\begin{figure}
  \centering
  \begin{mdframed}
    \begin{singlespace}
      \judgbox{TODO}{TODO}

\begin{mathpar}
  TODO
\end{mathpar}

%%% Local Variables:
%%% mode: latex
%%% TeX-master: "main"
%%% End:

    \end{singlespace}
  \end{mdframed}
  \caption{Big-step semantics modifications for environment memoization}
  \label{fig:big-step-memoization-rules}
\end{figure}

\subsection{Purity of implementation}
\label{sec:env_purity}

%%% Local Variables:
%%% mode: latex
%%% TeX-master: "main"
%%% End:
