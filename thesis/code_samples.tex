\chapter{Selected code samples}
\label{app:code-samples}

% macro for including a reasonml source file
\newcommand{\cb}[1]{
  \begin{singlespace}
    \inputminted[fontsize=\footnotesize,frame=single]{reasonml}{code/#1}
  \end{singlespace}
  \captionof{listing}{\texttt{#1}}
}

\section{Correspondence between theory and code}
\label{sec:correspondence}

\Cref{tab:correspondence-theory-code} is a list of symbols to the relevant ReasonML module(s). \Cref{tab:correspondence-algos-code} provides a list of algorithms or judgments to the relevant ReasonML module(s). If the source code for a listed module is present in this Appendix, then the section will be listed as well.

\begin{table}
  \centering
  \makebox[\textwidth][c]{
    \begin{tabular}{cll}
      \hline
      Symbol & Name & Module(s) \\
      \hline\hline
      $e$ & External expression & \texttt{UHExp} \\
      $\tau$ & Type & \texttt{HTyp} \\
      $\Gamma$ & Typing context & \texttt{Contexts} \\
      $d$ & Internal expression & \texttt{DHExp} (\Cref{sec:code-internal-language}) \\
      $\sigma$ & (Numbered) environment & \texttt{EvalEnv}, \texttt{EvalEnvId}, \texttt{VarBstMap} (\Cref{sec:code-evalenv}) \\
      $\hci$ & Hole closure tracking & \texttt{HoleClosureInfo\_}, \texttt{HoleClosureInfo} (\Cref{sec:code-hole-closures}) \\
      \adiff{d_1}{d_2}{u}{d} & Fill diff judgment & \texttt{DiffDHExp} (\Cref{sec:code-far}) \\
      \hline\hline
    \end{tabular}
  }
  \caption{Correspondence between symbols and code}
  \label{tab:correspondence-theory-code}
\end{table}

\begin{table}
  \centering
  \makebox[\textwidth][c]{
    \begin{tabular}{cll}
      \hline
      Algorithm & Name & Module(s) \\
      \hline\hline
      \env\vdash d\Downarrow d' & Evaluation & \texttt{Evaluator} (\Cref{sec:code-evaluator}) \\
      d\Uparrow d' & Postprocessing & \texttt{EvalPostprocess} (\Cref{sec:code-postprocessing}) \\
      \adiff{d_1}{d_2}{u}{d} & Structural diff & \texttt{DiffDHExp} (\Cref{sec:code-far}) \\
      \far{u}{d}{d'} & FAR & \texttt{FillAndResume} (\Cref{sec:code-far}) \\
      \hline\hline
    \end{tabular}
  }
  \caption{Correspondence between algorithms and code}
  \label{tab:correspondence-algos-code}
\end{table}

\section{Relevant code snippets}
\label{sec:relevant-code}

Relevant code snippets are shown below. Omitted sections are indicated with \mintinline{reasonml}|/* (...) */|. The source language is ReasonML. Each file represents a module. Files with the \texttt{.rei} file extension are interface files and only contain definitions (the module's public interface) and documentation in the form of comments. Files with the \texttt{.re} file extension contain the module's implementation. These code snippets are taken from the \texttt{fill-and-resume-backend} branch.

\subsection{Internal language}
\label{sec:code-internal-language}

\cb{DHExp.rei}

\cb{DHExp.re}

\subsection{Numbered environments}
\label{sec:code-evalenv}

\cb{VarBstMap.re}

\cb{EvalEnv.rei}

\cb{EvalEnv.re}

\cb{EvalEnvId.rei}

\cb{EvalEnvId.re}

\subsection{Evaluation}
\label{sec:code-evaluator}

\cb{Evaluator.rei}

\cb{Evaluator.re}

\subsection{Postprocessing}
\label{sec:code-postprocessing}

\cb{EvalPostprocess.rei}

\cb{EvalPostprocess.re}

\cb{Program.re}

\cb{Result.rei}

\cb{Result.re}

\subsection{Unique hole closures}
\label{sec:code-hole-closures}

\cb{MetaVar.rei}

\cb{MetaVar.re}

\cb{HoleClosureId.rei}

\cb{HoleClosureId.re}

\cb{HoleClosure.rei}

\cb{HoleClosure.re}

\begin{singlespace}
  \inputminted[fontsize=\footnotesize,frame=single]{reasonml}{code/HoleClosureInfo_.rei}
\end{singlespace}
\captionof{listing}{\texttt{HoleClosureInfo\_.rei}}

\begin{singlespace}
  \inputminted[fontsize=\footnotesize,frame=single]{reasonml}{code/HoleClosureInfo_.re}
\end{singlespace}
\captionof{listing}{\texttt{HoleClosureInfo\_.re}}

\cb{HoleClosureInfo.rei}

\cb{HoleClosureInfo.re}

\cb{HoleClosureParents.rei}

\cb{HoleClosureParents.re}

\subsection{FAR}
\label{sec:code-far}

\cb{FillAndResume.rei}

\cb{FillAndResume.re}

\cb{DiffDHExp.rei}

\cb{DiffDHExp.re}

\cb{Model.re}

\subsection{Evaluation state}
\label{sec:code-evalstate}

\cb{EvalState.rei}

\cb{EvalState.re}

\cb{EvalStats.rei}

\cb{EvalStats.re}

%%% Local Variables:
%%% mode: latex
%%% TeX-master: "main"
%%% End:
