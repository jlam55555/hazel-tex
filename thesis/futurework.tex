\chapter{Future work}
\label{sec:future_work}

\section{Improvements to FAR}
\label{sec:far-improvements}

\todo{need general cleanup of FAR implementation, not very complete}

\todo{optimizing how many edit states to go back: tradeoffs between storage (how many edit states to store) and speed (how much execution time to detect a FAR, how expensive is FAR depending on which edit state; currently the most recent edit state is used, something like a LRU; may want to cache recent edit states that spawned a FAR or group edit states like in the undo history currently)}

\subsection{Choosing the edit state to fill from}
\label{sec:far-past-edit-states}

Past edit states are stored in an undo history in Hazel, which allows the user to quickly return to previous edit states. The structure of the undo history is complicated and Hazel-specific, and thus not described here. We note that the result of evaluation is not stored alongside the evaluation result, but the evaluation function itself is memoized, so retrieving a previous edit state and re-evaluating the program is typically not expensive.

\todo{note why that memoization won't work well for us anymore}

There are a number of possible design decisions when searching for a valid hole fill. Firstly, one must decide the maximum number of edit states to search: should it be a fixed number of edit states, or should it be given a fixed time budget? Is it best to cache edit states that recently led to a fill operation (\`a la LRU cache)? Is the most recent edit state that leads to a valid fill usually the best candidate, or even a good candidate? Would it be best to allow for user-configurable settings, or perhaps even for the user to manually select the previous edit from which to fill?

\todo{automatic vs manual far detection}

\section{Mechanization of metatheorems and rules}
\label{sec:formalization}

\section{FAR for all edits}
\label{sec:far_all_edits}

% TODO: talk about structural diff-ing idea, go back to some past edit state
% in which all edits since then fall within some hole(s)

\section{Stateless and efficient notebook environment}
\label{sec:notebook_ui_future}

% TODO: talk about envisioning notebook-like ui

%%% Local Variables:
%%% mode: latex
%%% TeX-master: "main"
%%% End:
