\chapter{An overview of the Hazel programming environment}
\label{sec:hazel}

% TODO: define holes and the gap problem

Hazel is the experimental language that implements the Hazelnut bidirectionally-typed edit static semantics with holes and the Hazelnut Live dynamic semantics, and it is also the name of the reference implementation. It is intended to serve as a proof-of-concept of the semantics with holes that attempt to mitigate the gap problem; however, the implementation is becoming increasingly practical with additional research efforts. The reference implementation is an interpreter written in OCaml and transpiled to Javascript using the \mintinline{text}|js_of_ocaml| (JSOO) library \cite{vouillon2014bytecode} so that it may be run client-side in the browser. A screenshot of the reference implementation is shown in \Cref{fig:screenshot-hazel-ui} \cite{HazelDemo2022}. The source code may be found on GitHub \cite{Hazel2022}.

Hazel's syntax and semantics resembles languages in the ML (Meta Language) family of languages \cite{macqueen2020history} such as OCaml or SML/NJ, although Hazel does not support polymorphism at this time. Hazel can be characterized as a purely functional, statically-typed, bidirectionally-typed, strict-order evaluation, structured editor language. Hazel semantically differs most significantly from other ML languages in the last respect due to its theoretic foundations in solving the gap problem.

\section{Hazelnut static semantics}
\label{sec:statics}

\subsection{Expression and type holes}
\label{sec:holes}

\subsection{Bidirectional typing}
\label{sec:bidirectional_typing}

\subsection{Example of bidirectional type derivation}
\label{sec:typing_example}

\section{Hazelnut Live dynamic semantics}
\label{sec:dynamics}

\subsection{Example of elaboration}
\label{sec:elaboration_example}

\subsection{Example of evaluation}
\label{sec:evaluation_example}

\subsection{Example of hole instance numbering}
\label{sec:hole_instance_example}

\section{Hazel programming environment}
\label{sec:hazel_online}

\subsection{Explanation of interface}
\label{sec:hazel_interface}

\subsection{Implications of Hazel}
\label{sec:hazel_implications}

%%% Local Variables:
%%% mode: latex
%%% TeX-master: "main"
%%% End:
