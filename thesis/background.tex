\section{Background}
\label{sec:background}

\subsection{Functional programming}
\label{sec:fp}

\subsubsection{Recursion and mathematical induction}
\label{sec:recursion}

\subsubsection{The $\lambda$-calculus}
\label{sec:lambda_calculus}

\subsubsection{Purity and statefulness}
\label{sec:purity}

\subsubsection{Comparison to other programming paradigms}
\label{sec:fp_comp}

\subsection{Implementations for programming languages}
\label{sec:interpreters}

In order for a programming language to be practical, it must not only be defined as a set of syntax and semantics, but also have an \textit{implementation} to run programs in the language. Hazel is implemented as an interpreted language, whose runtime is transpiled to Javascript so that it may be run as a client-side web application in the browser.

It is important to note that the definition of a language (its syntax and semantics) are largely orthogonal to its implementation. In other words, a programming language does not dictate whether it requires a compiler or interpreter implementation, and languages sometimes have multiple implementations.

\subsubsection{Compiler vs. interpreter implementations}
\label{sec:comp_vs_interp}

% TODO: need citations for these definitions

There are two general classes of programming language implementations: \textit{interpreters} and \textit{compilers}. Both types of implementations share the function of taking a program as input, and should be able to produce the same result (assuming an equal and determinstic machine state, equal inputs, correct implementations, and no exceptional behavior due to differences in resource usage).

A compiler is a programming language implementation that converts the program to some low-level representation that is natively executable on the hardware architecture (e.g., x86-64 assembly for most modern personal computers, or the virtualized JVM architecture) before evaluation. This process typically comprises \textit{lexing} (breaking down into atomic tokens) the program text, \textit{parsing} the lexed tokens into a suitable \textit{intermediate representation} (IR) such as LLVM, performing optimization passes on the intermediate representation, and then generating the target bytecode (such as x86-64 assembly). The bytecode outputted from the compilation process is used for evaluation. Compiled implementations tend to produce better runtime efficiency, since the compilation steps are performed separate of the evaluation, and because there is little to no runtime overhead.

An interpreter is a programming language implementation that does not compile down to native bytecode, and thus requires an interpreter or \textit{runtime}, which performs the evaluation. Interpreters still require lexing and parsing, and may have any number of optimization stages, but do not generate bytecode for the native machine, instead evaluating the program directly.

% TODO: cite paper on SML elaboration

In certain contexts (especially in the ML spheres), the term \textit{elaboration} is used to the process of transforming the \textit{external language} (a well-formed, textual program) into the \textit{internal language} (IR). The interior language may include additional information not present in the external language, such as types generated by type inference (e.g., in SML/NJ) or bidirectional typing (e.g., in Hazel).

The distinction between compiled and interpreted languages is not a very clear line: some implementations feature just-in-time (JIT) compilation that allow ``on-the-fly'' compilation (e.g., the JVM or CLR), and some implementations may perform the lexing and parsing separately to generate a non-native bytecode representation to be later evaluated by a runtime. A general characterization of compiled vs. interpreted languages is the amount of runtime overhead required by the implementation.

Hazel is a purely interpreted language implementation, as optimizations for speed are not among its main concerns. However, performance is clearly one of the main concerns of this thesis project, but the gains will be algorithmic and use the nature of Hazel's structural editing and hole calculus to benefit performance, rather than changing the fundamental implementation. There is, however, a separate endeavor to write a compiled interpretation of Hazel\footnote{\url{https://github.com/hazelgrove/hazel/tree/hazelc}}, which is outside the scope of this project.

\subsubsection{The substitution and environment models of evaluation}
\label{sec:sub_vs_eval}

% TODO: move the non-Hazel related parts of this to the background section

% REF: https://cs.brown.edu/courses/cs173/2012/book/From_Substitution_to_Environments.html
% REF: https://www.cs.bham.ac.uk/research/projects/poplog/paradigms_lectures/lecture18.html
%   this introduces the issue with special forms like set! in a non-functional context

Evaluation in Hazel was performed using originally using a \textit{substitution model of evaluation}, which is a theoretically simpler model. In this model, variables that are bound by some construct are substituted into the construct's body. For example, the variable(s) bound using a \mintinline{ocaml}|let|-expression pattern are substituted in the \mintinline{ocaml}|let|-expression's body, and the variable(s) bound during a function application are substituted into the function's body, and then the body is evaluated.

% TODO: show example of this

In this formulation, variables are ``given meaning'' via substitution; once evaluation reaches an expression, all variables in scope (in the typing context) will have been replaced by their value by some containing binding expression. In other words, variables are never evaluated directly; they are substituted by their values when bound, and their values are evaluated. The substitution model is useful for teaching purposes because it is simple and close to its mathematical definition: a variable can be thought of as an equivalent stand-in for its value.

However, for the purpose of computational efficiency, a model in which values are lazily expanded (``looked-up'') only when needed is more efficient. This is called the \textit{environment model of evaluation}, and generally is more efficient because the runtime does not need to perform an extra substitution pass over subexpressions; untraversed (unevaluated) branches do not require substituting; and the runtime does not need to carry an expression-level IR of the language. The last point is due to the fact that the substitution model manipulates expressions, while evaluation does not; this means that the latter is more amenable for compilation, and is how compiled languages tend to be implemented: each frame of the theoretical stack frame is a de facto environment frame. While switching from the substitution to environment model is not an improvement in asymptotic efficency, these effects are useful especially for high-performance and compiled languages.

Note that the substitution model does not imply a lazy (i.e., normal-order, call-by-name, call-by-need) evaluation as in languages such as Haskell or Miranda, in which bound variables are (by default) not evaluated until their value is required. Laziness is conceptually tied to substitution, but the substitution model does not require laziness. Like most programming languages, Hazel only has strict (i.e., applicative-order, call-by-value) evaluation: the expressions bound to variables are evaluated at the time of binding.

% TODO: "Laziness is conceptually tied to substitution" -- but does it require substitution? Methinks so but not sure;
%   will find out more later due to this independent study

The implementation of evaluation with environments differs from that of evaluation with substitution primarily in that: an evaluation environment is required to look up bound variables as evaluation reaches them; binding constructs extend the evaluation environment rather than performing substitution; and $\lambda$ abstractions are bound with their evaluation environment at runtime to form (lexical) closures.

\subsection{Classifications of type systems}
\label{sec:ts_classifications}

\subsubsection{Primer on programming language semantics and reasoning}
\label{sec:pl_semantics}

\subsubsection{Static vs. dynamic typing}
\label{sec:staticness}

\subsubsection{Strong vs. weak typing}
\label{sec:strength}

\subsubsection{Inferred vs. manifold typing}
\label{sec:inferred_vs_manifold}

\subsubsection{Gradual typing}
\label{sec:gradual}

\subsubsection{Other classifications}
\label{sec:other_types}

\subsection{Approaches to programming interfaces}
\label{sec:prog_intf}

\subsubsection{Structure editors}
\label{sec:structure_editors}

\subsubsection{Graphical editors}
\label{sec:graphical_editors}

\subsubsection{Intentional, generative, and meta-programming}
\label{sec:intentional_programming}

\subsubsection{Applications to programming education}
\label{sec:programming_education}

\subsubsection{Drawbacks of non-textual editors}
\label{sec:textual_benefits}

%%% Local Variables:
%%% mode: latex
%%% TeX-master: "main"
%%% End:
