\chapter{Identifying hole closures by physical environment}
\label{sec:renumbering}

\section{Rationale behind hole instances and unique hole closures}
\label{sec:instance-illustration}

Consider the program displayed in \Cref{fig:instance-illustration}. The evaluation result of the program is \[
  [a\leftarrow[\varnothing]\hehole^1,x\leftarrow 3]\hehole^2
  + [a\leftarrow[\varnothing]\hehole^1,x\leftarrow 4]\hehole^2
\] Note that the two instances of \heh2 have different environments, and we thus distinguish between the two occurrences of \heh2 as separate \textit{instances} of a hole. However, note that while there are also two instances of the hole \heh1 in the result, these share the same (physically equal) environment. No matter what expression we fill hole \heh1 with (for example, using the fill-and-resume operation) the hole will evaluate to the same value. This differs from the hole \heh2, whose filling may cause different instances to evaluate to different values due to non-capture-avoiding substitution. For example, filling hole \heh2 with the expression $x+2$ will cause the instances to resolve to $5$ and $6$, respectively.

\begin{listing}
  \inputhminted{instance_illustration}
  \caption{Illustration of hole instances}
  \label{fig:instance-illustration}
\end{listing}

The current implementation assigns an identifier $i$ to each instance of a hole, and the instance number is unique between all instances of a hole. While this makes perfect sense for \heh2, the assignment of two separate holes to \heh1 may confuse Hazel users, since these hole instances are identical and filling them with any value will result in the same value. The solution is to unify all instances of a hole which share the same (physically equal) environment, and thus identify hole instances by hole number and environment. A set of hole instances that share the same environment will be called a \textit{unique hole closure}, or simply \textit{hole closure}\footnote{``Hole closure'' also is used to describe the generalized closure around hole expressions as described in \Cref{sec:env_model_evaluation}. Here we are referring to the set of instances of the same hole that share the same physical environment. Hence we call this interpretation ``unique hole closure'' to distinguish it from the former interpretation, but the interpretation should be clear from context.}.

To illustrate why physical equality is used to identify environments, consider the case shown in \Cref{fig:physical-equality-illustration}. This simpler program evaluates to \[
  [x\leftarrow 2]\hehole^1 + [x\leftarrow 2]\hehole^1
\] In this case, hole 1 has two instances with two environments with structurally equal bindings. If the argument to the second invocation of $f$ is changed to $3$, then the holes will have different environments and may thus fill to different values. This may be confusing to the Hazel user; what appears to be a single hole closure is actually two different hole closures which incidentally have the same values bound to its variables.

An intuitive way of understanding the use of physical equality is that separate \textit{instantiations} of the same hole should be distinguished. This is highly related to function applications. A hole may only appear multiple times in the result in two different ways: it may exist in the body of a function that is invoked multiple times (multiple hole instantiations), or it may appear in a hole that is referenced from other holes (shared hole instantiation). The former leads to multiple hole closures, while the latter leads to a single hole closure.

\begin{listing}
  \inputhminted{physical_equality_illustration}
  \caption{Illustration of physical equality for environment memoization}
  \label{fig:physical-equality-illustration}
\end{listing}

\section{The existing hole instance numbering algorithm}
\label{sec:existing-hole-numbering}

Hole numbering is a process that follows evaluation and operates on the evaluation result\footnote{The function in the existing codebase that performs hole renumbering is \mintinline{ocaml}|Program.renumber|. We may refer to it throughout this text as ``hole numbering,'' ``hole renumbering,'' or ``hole tracking.''}. It assigns a hole instance number to each hole. Hole instances are briefly motivated in Hazelnut Live, but the algorithm for hole numbering algorithm was not. We will provide a brief high-level description of it here, and then provide the inference rules for our own implementation. It is a breadth-first search of the result, recursing through holes. When a hole is encountered, it is assigned a unique hole instance number\footnote{I.e., each hole instance is uniquely identified by the pair of identifiers $(u,i)$. The hole instance number only has to be unique out of the hole instances for a particular hole $u$.} and added to a data structure \mintinline{ocaml}|HoleInstanceInfo.t| that keeps track of all hole instances. Each hole instance's hole number, hole instance number, hole closure environment, and path\footnote{The path of a hole is the recursive list of hole parents that must be traversed in order to reach a hole. In other words, this is the path to a hole if we envision the result expression as a tree, in which each hole is a node that fathers all of its variable binding expressions.} is stored in this data structure. The \mintinline{ocaml}|HoleInstanceInfo.t| is in turn stored in the \mintinline{ocaml}|Result.t| that stores all of the information about an evaluated program. The primary use of \mintinline{ocaml}|HoleInstanceInfo.t| is for the context inspector. With this data structure, users may easily iterate all instances of a selected hole, examine the hole path of a selected hole, examine the environment of a selected hole, or navigate to another hole instance.

\section{Issues with the current implementation}
\label{sec:current-problems}

Consider the program shown in \Cref{fig:hole_renumbering_problem}. A performance issue appears with the existing evaluator with the program shown in \Cref{fig:hole_renumbering_problem}\footnote{This was first brought to attention by a GitHub issue at \url{https://github.com/hazelgrove/hazel/issues/536}.}. As we increase the number of consecutive \mintinline{ocaml}|let| expressions, we get an exponential slowdown that makes evaluation impractical for $n>10$. The results of running this program for several values of $n$ is shown in tabular form in \Cref{tab:perf-hole-blowup} and graphically at \Cref{fig:perf-renum-dev}.

For now, let us consider the case when $n=3$. When evaluating with environments\footnote{When evaluating using the substitution model, evaluation also slows down exponentially, because the variables are eagerly substituted into the hole environments. We do not have a performance issue with evaluation with environments because of lazy variable lookups.}, the result is shown in \Cref{fig:hole-renumbering-solution-structure}.

The program slowdown happens in the hole numbering process. Recall from \Cref{sec:existing-hole-numbering} that the hole numbering process is a simple tree traversal algorithm. Thus, each time a hole (with the same environment) is encountered, it and all of its descendant holes will be given more hole instance numbers. This leads to the hole numbering shown in \Cref{fig:hole-renumbered-result}. We see that there are four instances of hole 1, two instances of hole 2, and one instance of hole 3. In sum, we see that there are eight total hole instances. The number of holes increases by powers of two. As $n$ increases, the total number of holes (including the instance of last hole) will be exactly $2^n$.

Clearly this is undesirable from an efficiency perspective. It is also undesirable from the perspective that there is only one instantiation of each of the holes. While there are multiple paths to each node, we would like to change the representation to match that of the unique hole closures or hole instantiations as described in \Cref{sec:instance-illustration}.

\begin{listing}
  \centering
  \inputhminted{holes_consecutive}
  \caption{A Hazel program that generates an exponential ($2^N$) number of total hole instances}
  \label{fig:hole_renumbering_problem}
\end{listing}

\begin{figure}
  \centering
  \begin{tikzpicture}
    \node[] (hole4) {$[\env^4]\hehole^4$};
    \node[below=of hole4] (hole3) {$[\env^3]\hehole^3$};
    \node[below=of hole3] (hole2) {$[\env^2]\hehole^2$};
    \node[below=of hole2] (hole1) {$[\varnothing]\hehole^1$};

    \draw[->] (hole4) -- node [midway,left] {$c$} (hole3);
    \draw[->] (hole4) to[bend left=60] node [midway,right] {$b$} (hole2);
    \draw[->] (hole4) to[bend left=80] node [midway,right] {$a$} (hole1);
    \draw[->] (hole3) -- node [midway,left] {$b$} (hole2);
    \draw[->] (hole3) to[bend right=60] node [midway,left] {$a$} (hole1);
    \draw[->] (hole2) -- node [midway,left] {$a$} (hole1);
  \end{tikzpicture}
  \caption{Structure of the result of the program in \Cref{fig:hole_renumbering_problem}}
  \label{fig:hole-renumbering-solution-structure}
\end{figure}

\begin{figure}
  \centering
  \begin{tikzpicture}
    \node[] (hole41) {$[\env^4]\hehole^{4:1}$};
    \node[below=of hole41] (hole31) {$[\env^3]\hehole^{3:1}$};
    \node[below=of hole31] (hole21) {$[\env^2]\hehole^{2:2}$};
    \node[below=of hole21] (hole11) {$[\varnothing]\hehole^{1:4}$};
    \node[left=of hole21] (hole12) {$[\varnothing]\hehole^{1:3}$};
    \node[left=of hole12] (hole13) {$[\varnothing]\hehole^{1:2}$};
    \node[above=of hole13] (hole22) {$[\env^2]\hehole^{2:1}$};
    \node[left=of hole22] (hole14) {$[\varnothing]\hehole^{1:1}$};

    \draw[->] (hole41) to[] node[left] {$c$} (hole31);
    \draw[->] (hole41) to[] node[below right] {$b$} (hole22);
    \draw[->] (hole41) to[] node[above left] {$a$} (hole14);
    \draw[->] (hole31) to[] node[left] {$b$} (hole21);
    \draw[->] (hole31) to[] node[above left] {$a$} (hole12);
    \draw[->] (hole22) to[] node[above left] {$a$} (hole13);
    \draw[->] (hole21) to[] node[above left] {$a$} (hole11);
  \end{tikzpicture}
  \caption{Numbered hole instances in the result of \Cref{fig:hole_renumbering_problem}}
  \label{fig:hole-renumbered-result}
\end{figure}

\subsection{Hole instance path versus hole closure parents}
\label{sec:closure-parents}

Visually, we would like to change the hole tracking to use a representation more similar to \Cref{fig:hole-renumbering-solution-structure} rather than that of \Cref{fig:hole-renumbered-result}. In the old representation, each hole instance is uniquely identified by a hole number and hole path.

In the new representation, hole instantiations are uniquely identified by hole number and environment, and are not uniquely identified by a path anymore. Thus, for each hole closure we instead keep track of a list of its parent holes.

We note that these two representations of a graph are equivalent\footnote{This structure is more specifically a join-semilattice.}, assuming that nodes sharing an environment are considered to be physically equal. The first describes the path to each node, while the latter is a well-known adjacency list. Either representation of a graph can be used to construct the other, but the latter is much more efficient in the case of a dense graph.

Changing the structure from using hole paths to hole parents forces a minor change to the Hazel UI. When a user selects a hole, rather than showing the path to the hole, the list of parents to the hole are shown instead.

\section{Algorithmic concerns and a two-stage approach}
\label{sec:two-stage-renumber}

To efficiently build the new hole-tracking data structure, we expect to have a fast lookup of hole numbers and environment identifiers. On the other hand, we want the interface of this data structure to be similar to the interface of \mintinline{ocaml}|HoleInstanceInfo.t|: the user should be able to efficiently look up environments by hole number and hole closure number.

To efficiently handle both of these desired properties, we require two different data structures. The first is an auxiliary data structure \mintinline{ocaml}|HoleClosureInfo_.t| that is a map $\hci:(u,\sigma)\mapsto(i,\pth)$ (where $\pth$ indicates the list of hole closure parents). The second is the data structure that will be used for the context inspector and hole closure lookups, \mintinline{ocaml}|HoleClosureInfo.t|, that is a map $\hci:(u,i)\mapsto(\sigma,\pth)$. The maps are implemented as binary search trees for efficient lookups and updates\footnote{Note that this is one of the few places where a hashtable implementation is appropriate in the context of this project, since we do not copy these data structures. However, there will likely not be a major performance benefit; the main benefit lies in memoizing environments.}. The first stage of this algorithm is to build the \mintinline{ocaml}|HoleClosureInfo_.t|; the second stage is to convert it to a \mintinline{ocaml}|HoleClosureInfo.t|\footnote{It is a good time to bring up whether algorithmic efficiency is a matter of concern at all }.

For convenience, we do not use two different symbols for these two data structures; the difference is purely an implementation detail regarding the construction of the data structure. The conversion from \mintinline{ocaml}|HoleClosureInfo_.t| to \mintinline{ocaml}|HoleClosureInfo.t| is trivial and will not be described here in detail. It simply involves looping over the unique hole closures and changing the mapping to be indexed by hole number and hole closure number.

\subsection{The hole numbering algorithm}
\label{sec:hole-numbering-algorithm}

The hole numbering algorithm is shown in \Cref{fig:big-step-renumber-new-rules}. Constants and variables are left unchanged by the hole numbering algorithm. For ordinary expressions with subexpressions, the algorithm recurses through subexpressions.

For hole expressions that have not been encountered before, the hole number and environment do not exist in $\hci$. A hole closure number $i=\hid(H,u)$ is generated to be unique out of hole closures for the hole $u$. We recursively postprocess the environment. Then the hole closure is inserted into $\hci$, along with the postprocessed environment.

For hole expressions that have been encountered before, the hole number and environment do exist in $\hci$. We can use this to look up the hole closure number $i$, postprocessed environment $\env'$, and list of parents $\{\pth_i\}$ for this hole closure. We update the list of parents to include the current parent $\pth$, and return the hole numbered with $i$. Note that the environment does not need to be re-postprocessed.

\begin{figure}
  \centering
  \begin{mdframed}
    \begin{singlespace}
      \judgbox{\hci,\pth\vdash d\ppn d'\dashv\hci'}{$d$ gets renumbered to $d'$}

\begin{mathpar}
  \Infer{\ppnl{}Const}{}{\hci,\pth\vdash c\ppn c\dashv\hci}
  \and
  \Infer{\ppnl{}Var}{}{\hci,\pth\vdash x\ppn x\dashv\hci}
  \and
  \Infer{\ppnl{}Lam}{
    \hci,\pth\vdash d\ppn d'\dashv\hci'
  }{\hci,\pth\vdash\lambda x:\tau.d\ppn\lambda x:\tau.d'\dashv\hci'}
  \and
  \Infer{\ppnl{}Ap}{
    \hci,\pth\vdash d_1\ppn d_1'\dashv\hci' \\
    \hci',\pth\vdash d_2\ppn d_2'\dashv\hci''
  }{\hci,\pth\vdash d_1\ d_2\ppn d_1'\ d_2'\dashv\hci'}
  \and
  \Infer{\ppnl{}Asc}{
    \hci,\pth\vdash d\ppn d'\dashv\hci'
  }{\hci,\pth\vdash d:\tau\ppn d':\tau\dashv\hci'}
  \and
  \Infer{\ppnl{}EHoleNew}{
    (u,e)\not\in\hci \\
    i=\hid(H,u) \\
    \hci,(u,i)\vdash\env\ppn\env'\dashv\hci' \\
    \hci''=\hci,(u,e)\leftarrow(i,\{\pth\},\env')
  }{\hci,\pth\vdash[\env^e]\hehole^u\ppn[\env'^e]\hehole^{u:i}\dashv\hci''}
  \and
  \Infer{\ppnl{}EHoleFound}{
    \hci=\hci',(u,e)\leftarrow(i,\{\pth_i\},\env'^e) \\
    \hci''=\hci,(u,e)\leftarrow(i,\{\pth_i\}\cup\{\pth\},\env'^e)
  }{\hci,\pth\vdash[\env^e]\hehole^u\ppn[\env'^e]\hehole^{u:i}\dashv\hci''}
  \and
  \Infer{\ppnl{}NEHoleNew}{
    (u,e)\not\in\hci \\
    i=\hid(H,u) \\
    \hci,(u,e),\vdash\env\ppn\env'\dashv\hci' \\
    \hci''=\hci,(u,e)\leftarrow(i,\{\pth\},\env') \\
    \hci'',\pth\vdash d\ppn d'\dashv\hci'''
  }{\hci,\pth\vdash[\env^e]\hhole{d}^u\ppn[\env'^e]\hhole{d'}^{u:i}\dashv\hci'''}
  \and
  \Infer{\ppnl{}NEHoleFound}{
    \hci=\hci',(u,e)\leftarrow(i,\{\pth_i\},\env'^e) \\
    \hci''=\hci,(u,e)\leftarrow(i,\{\pth_i\}\cup\{\pth\},\env'^e) \\
    \hci'',\pth\vdash d\ppn d'\dashv\hci'''
  }{\hci,\pth\vdash[\env^e]\hhole{d}^u\ppn[\env'^e]\hhole{d'}^{u:i}\dashv\hci'''}
\end{mathpar}

\judgbox{\hci,\pth\vdash\env\ppn\env'\dashv\hci'}{$\env$ gets renumbered to $\env'$}

\begin{mathpar}
  \Infer{\ppnl{}EnvTriv}{}{\hci,\pth\vdash\varnothing\ppn\varnothing\dashv\hci}
  \and
  \Infer{\ppnl{}Env}{
    \hci,\pth\vdash\env\ppn\env'\dashv\hci' \\
    \hci',\pth\vdash d\ppn d'\dashv\hci''
  }{\hci,\pth\vdash\env,x\leftarrow d\ppn\env',x\leftarrow d'\dashv\hci''}
\end{mathpar}

%%% Local Variables:
%%% mode: latex
%%% TeX-master: "main"
%%% End:

    \end{singlespace}
  \end{mdframed}
  \caption{Hole closure numbering postprocessing semantics}
  \label{fig:big-step-renumber-new-rules}
\end{figure}

\subsection{Hole closure numbering order}
\label{sec:numbering-order}

The order of the numbers assigned to hole closures is not specified in the algorithm shown in \Cref{fig:big-step-renumber-new-rules}, but it is a consideration in the implementation. In the existing implementation, the evaluation result is traversed in a breadth-first search (BFS) order. On the other hand, our implementation is more simply implemented using a depth-first search (DFS) order. This changes the order that hole closures are encountered and numbered. While the hole closure number is not specified explicitly ordered value, certain orderings may be more intuitive to the user. A choice of explicit hole numbering order is left to future work.

\subsection{Unification with $\lambda$-conversion post-processing}
\label{sec:unification-postprocessing}

The overall postprocessing operation is shown in \Cref{fig:big-step-postprocessing-rules}. The postprocessing is the composition of the $\lambda$-conversion postprocessing and the hole numbering postprocessing.

We can combine the two postprocessing steps into a single pass. Since the renumbering step leaves most expressions unchanged, it is convenient to incorporating the hole numbering rules into the closure and hole rules for $\lambda$-conversion postprocessing. The reference implementation uses this single-pass postprocessing.

There is a nuance here: the hole numbering step limits the use of memoization of environments in the postprocessing process. In particular, a complexity arises because we wish to update the list of parents of a hole each time a hole is encountered, forcing us to re-postprocess environments. Thus we require that closure environments are always ``shallowly'' postprocessed. This means that environments are postprocessed whenever a hole is encountered, rather than whenever a closure is encountered. Lastly, since the $\lambda$-closure postprocessing requires that the environment be postprocessed before performing a variable substitution, environments must also be postprocessed whenever a (bound) variable is encountered. In summary, the \pplcl{}Closure rule (outside the evaluation boundary) does not postprocess the closure's environment, and the \pplcl{}Var and \pplcl{}(N)EHole rules (inside the evaluation boundary) postprocess the environment.

This ``fix'' is not ideal because of the added complexity and confusion, but it does mostly memoize the postprocessing process. We leave a more elegant implementation of the postprocessing step is left for future work. However, we note that we do not note any performance issues with postprocessing when we perform our empirical performance evaluation in \Cref{sec:evaluation-renumbering}.

\begin{figure}
  \centering
  \begin{mdframed}
    \begin{singlespace}
      \judgbox{d\pp(\hci,d')}{$d$ postprocess-evaluates to $d'$ with hole closure info $\hci$}

\begin{mathpar}
  \Infer{PP-Result}{
    d\pplc d' \\
    d'\ppn (\hci,d'')
  }{d\pp(\hci,d'')}
\end{mathpar}

%%% Local Variables:
%%% mode: latex
%%% TeX-master: "main"
%%% End:
    \end{singlespace}
  \end{mdframed}
  \caption{Overall postprocessing judgment}
  \label{fig:big-step-postprocessing-rules}
\end{figure}

\section{Fast structural equality checking}
\label{sec:fast-equals}

After the hole instance numbering is solved, there is an additional performance issue that is related to a recursive traversal of the evaluation result. After evaluation and hole numbering, there is an additional step (located in \mintinline{ocaml}{Model.update_program}) that compares two evaluation results (\mintinline{ocaml}|Result.t|) using a structural equality check\footnote{This step in \mintinline{ocaml}|Model.update_program| is used to check if the program result changes. There may be more efficient heuristics to detect a change in program output, such as comparing the programs' external expressions or detecting the type of edit action. However, we are simply concerned here with maintaining the original intent of comparing structural equality. Moreover, this memoized structural equality check may be useful whenever a structural equality check is required on expressions with environments.}.

This step is also very slow if repeated environments are re-traversed, so we memoize it by environments. We implement a manual structural checking algorithm, \mintinline{ocaml}|DHExp.fast_equals|. For any leaf node (node with no subexpressions), the value of the node is compared for equality. For branch nodes (nodes with subexpressions), the nodes are equal if subexpressions are equal and if the node's properties are equal. Importantly, the equality check for environments is simply to check if the environment identifiers are equal.

This step makes the assumption that checking the equality of environment identifiers is equivalent to checking the physical equality of two environments. \Cref{thm:env-id} states that this is true for two environments in the same program evaluation, but this may not be the case anymore for comparing environments across separate evaluations. Thus we also need to structurally check environments. Luckily, this is easily memoized so that we only compare each environment exactly once.

%%% Local Variables:
%%% mode: latex
%%% TeX-master: "main"
%%% End:
