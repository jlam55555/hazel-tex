\section{Memoizing hole instance numbering using environments}
\label{sec:renumbering}

\begin{singlespace}
  % see: https://tex.stackexchange.com/a/110156
  \begin{figure}
    \centering
    \begin{subfigure}{\textwidth}
      \begin{mdframed}[bottomline=false]
        \judgbox{\hci\vdash d\ppnd(\hci',d')}{
  Hole instance numbering in expression $d$ with hole instance info $\hci$
}
\begin{mathpar}
  \Infer{\ppndl-Value}{
    d\textsf{ value} \\
    d\ne\lambda x.d
  }{\hci\vdash d\ppnd(\hci,d)}
  \and
  \Infer{\ppndl-Var}{}{\hci\vdash x\ppnd(\hci,x)}
  \and
  \Infer{\ppndl-Lam}{
    \hci\vdash d\ppnd(\hci',d')
  }{\hci\vdash\lambda x.d\ppnd(\hci',d')}
  \and
  \Infer{\ppndl-Ap}{
    \hci\vdash d_1\ppnd(\hci',d_1') \\
    \hci'\vdash d_2\ppnd(\hci'',d_2')
  }{\hci\vdash\lambda d_1(d_2)\ppnd (\hci'',d_1'(d_2'))}
  \and
  \Infer{\ppndl-Op}{
    \hci\vdash d_1\ppnd(\hci',d_1') \\
    \hci'\vdash d_2\ppnd(\hci'',d_2')
  }{\hci\vdash\lambda d_1+d_2\ppnd (\hci'',d_1'+d_2')} \\
  \and
  \Infer{\ppndl-EHole}{
    \hid(\hci,u)=i \\
    \hci'=\hci,(u,i,\_,p)
  }{\hci\vdash\hehole_\env^u\ppnd(\hci',\hehole_\env^{u:i})}
  \and
  \Infer{\ppndl-NEHole}{
    \hid(\hci,u)=i \\
    \hci'=\hci,(u,i,\_,p) \\
    \hci'\vdash d\ppnd(\hci'',d') \\
  }{\hci\vdash\hhole{d}_\env^u\ppnd(\hci'',\hhole{d'}_\env^{u:i})}
\end{mathpar}

\judgbox{\hci\vdash d\ppns(\hci',d')}{
  Hole instance numbering in hole envs in $d$ with hole instance info $\hci$
}
\begin{mathpar}
  \Infer{\ppnsl-Value}{
    d\textsf{ value} \\
    d\ne\lambda x.d
  }{\hci\vdash d\ppns(\hci,d)}
  \and
  \Infer{\ppnsl-Var}{}{\hci\vdash x\ppns(\hci,x)}
  \and
  \Infer{\ppnsl-Lam}{
    \hci\vdash d\ppns(\hci',d')
  }{\hci\vdash\lambda x.d\ppns(\hci',d')}
  \and
  \Infer{\ppnsl-Ap}{
    \hci\vdash d_1\ppns(\hci',d_1') \\
    \hci'\vdash d_2\ppns(\hci'',d_2')
  }{\hci\vdash\lambda d_1(d_2)\ppns (\hci'',d_1'(d_2'))}
  \and
  \Infer{\ppnsl-Op}{
    \hci\vdash d_1\ppns(\hci',d_1') \\
    \hci'\vdash d_2\ppns(\hci'',d_2')
  }{\hci\vdash\lambda d_1+d_2\ppns (\hci'',d_1'+d_2')} \\
  \and
  \Infer{\ppnsl-EHole}{
    \hci\vdash\env\ppnd(\hci',\env') \\
    \hci'\vdash\env'\ppns(\hci'',\env'') \\
  }{\hci\vdash\hehole_\env^{u:i}\ppns (\hci'',\hehole_{\env''}^{u:i})}
  \and
  \Infer{\ppnsl-NEHole}{
    \hci\vdash d\ppns(\hci',d') \\
    \hci'\vdash\env\ppnd(\hci'',\env') \\
    \hci''\vdash\env'\ppns(\hci''',\env'')
  }{\hci\vdash\hhole{d}_\env^{u:i}\ppns (\hci''',\hhole{d'}_{\env''}^{u:i})}
\end{mathpar}

%%% Local Variables:
%%% mode: latex
%%% TeX-master: "main"
%%% End:
      \end{mdframed}
    \end{subfigure}
  \end{figure}
  \begin{figure}
    \ContinuedFloat
    \begin{subfigure}{\textwidth}
      \begin{mdframed}[topline=false]
        \judgbox{\hci\vdash\env\ppnd(\hci',\env')}{
  Hole instance numbering in hole environment $\sigma$ with hole instance info $\hci$
}
\begin{mathpar}
  \Infer{\ppndl-TrivEnv}{}{\hci\vdash\varnothing\ppnd(\hci,\varnothing)}
  \and
  \Infer{\ppndl-Env}{
    \hci\vdash\env\ppnd(\hci',\env') \\
    \hci'\vdash d\ppnd(\hci'',d') \\
  }{\hci\vdash\env,x\leftarrow d\ppnd(\hci'',(\env',x\leftarrow d'))}
\end{mathpar}

\judgbox{\hci\vdash\env\ppns(\hci',\env')}{
  Hole instance numbering in hole environment $\sigma$ with hole instance info $\hci$
}
\begin{mathpar}
  \Infer{\ppnsl-TrivEnv}{}{\hci\vdash\varnothing\ppns(\hci,\varnothing)}
  \and
  \Infer{\ppnsl-Env}{
    \hci\vdash\env\ppns(\hci',\env') \\
    \hci'\vdash d\ppns(\hci'',d') \\
  }{\hci\vdash\env,x\leftarrow d\ppns(\hci'',(\env',x\leftarrow d'))}
\end{mathpar}

\judgbox{d\ppn(\hci',\env')}{
  Hole instance numbering in expression $d$ and subexpressions
}
\begin{mathpar}
  \Infer{PP$_i$-Root}{
    \varnothing\vdash d\ppnd(\hci,d') \\
    \hci\vdash d'\ppnd(\hci',d'')
  }{d\ppn(\hci',d'')}
\end{mathpar}


%%% Local Variables:
%%% mode: latex
%%% TeX-master: "main"
%%% End:

      \end{mdframed}
    \end{subfigure}
    \caption{Big-step semantics for the previous hole instance numbering algorithm}
    \label{fig:small-step-formal}
  \end{figure}
\end{singlespace}

\subsection{Issues with the current implementation}
\label{sec:current_problems}

Consider the program shown in \Cref{fig:sample_hazel_program}.

\begin{listing}
  \centering
  \begin{hminted}
let a = /\heh1/ in
let b = /\lbd/ x . { a + x + /\heh2/ } in
let c = /\heh3/ in
/\heh4/ + b 1 + f /\heh5/
  \end{hminted}
  \caption{A seemingly innocuous Hazel program}
  \label{fig:sample_hazel_program}
\end{listing}

A performance issue appears with the existing evaluator with the program shown in \Cref{fig:hole_renumbering_problem}.

% TODO: describe performance issue

% TODO: describe the existing hole renumbering implementation

\subsection{Hole instances and closures}
\label{sec:hole_instances_and_closures}

\begin{listing}
  \centering
  \begin{hminted}
let a = /\heh1/ in
let b = /\heh2/ in
let c = /\heh3/ in
let d = /\heh4/ in
let e = /\heh5/ in
let f = /\heh6/ in
let g = /\heh7/ in
/\dots/
let x = /\heh n/ in
/\heh{n+1}/
  \end{hminted}
  \caption{A Hazel program that generates an exponential ($2^N$) number of total hole instances}
  \label{fig:hole_renumbering_problem}
\end{listing}

\subsection{Algorithmic concerns and a two-stage approach}
\label{sec:two-stage-renumber}

\subsection{Memoization and unification with closure post-processing}
\label{sec:renumbering_memoization}

\subsection{Differences in the hole instance numbering}
\label{sec:differences_numbering}

% two-stage to one-stage approach

%%% Local Variables:
%%% mode: latex
%%% TeX-master: "main"
%%% End:
