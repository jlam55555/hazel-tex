\chapter{Conclusions and recommendations}
\label{sec:concl}

This thesis explores several advancements to the evaluation model in Hazel, an implementation of a live programming editor with typed holes. We develop rules and a metatheory for the propesd changes as well as a working implementation for most of the rules provided.

The first proposed change is the switch from using variable substitution at binding time to storing variables in an environment at lookup time. This implementation leads to the introduction of closures to the internal language of Hazel, as well as a postprocessing step to restore the same result that would have been achieved by evaluation using the substitution model. Initially, we begin with the conventional function closures and the hole closures introduced in Hazelnut. Later, we introduce generalized closures, which subsume function and hole closures, and represent any paused evaluation. Generalized closures play a critical role in FAR.

The second major change is the use of the environment model to memoize operations that occur on shared environments. Environments are uniquely numbered to allow for lookup and memoization. Memoization is applied to the hole renumbering process (in the process changing the useful interpretation from hole instances to hole closures), and to speed up the structural equality checking. Memoization may also be helpful with the re-evaluation with filled expressions and environments during the FAR re-evaluation process, but this has not been fully implemented yet.

The third major contribution is a set of practical considerations to implementing the fill-and-resume (FAR) optimization, as originally proposed in Hazelnut Live. FAR promotes resuming evaluation from a previous evaluation result when an edit action is performed, as opposed to restarting evaluation from the beginning on every edit action.  Firstly, we describe a structural diffing algorithm for detection of a valid fill operation. This algorithm is intuitive and robust to work between an arbitrary past edit state and the current edit state (a $n$-step fill operation). We also provide the basic semantics for the fill (preprocessing) and resume (re-evaluation) operations. A 1-step FAR operation is implemented as a proof of concept.

We evaluate the performance of these methods empirically, via a series of benchmarks of sample programs. We compare the difference between the current main development branch on the \texttt{dev} branch to an updated evaluation model implementing the changes proposed in this thesis, in the \texttt{eval-environment} and \texttt{fill-and-resume-backend} branches. The results qualitatively match the expectation. Evaluation with environments is beneficial for performance when lazy variable lookups are reduced and the environment size is small. Substitution may be beneficial for performance when the number of substitution passes is small. The memoization of environments solves the performance issue in the program that motivated the memoization of environments in the hole numbering and structural equality checking processes. FAR provides a great improvement in efficiency when there is a valid fill operation and expensive re-computations can be avoided. However, FAR may sometimes be more expensive than regular evaluation from the beginning, due to the recursive re-evaluation of environments. Future work on this environment memoization and the implementation of the $n$-step fill operation should further improve the performance benefit of FAR.

We do not prove the correctness of the implementation. We instead provide a metatheory governing the implementation and provide a logical intuition for the correctness of the proposed metatheorems. We leave formal proofs of the metatheory and proofs of the completenes of the metatheorems to future work.

The primary goal of this work is to inform future development on Hazel or related research efforts, and the explanations and motivations have been written in enough detail to allow for others to independently reproduce this work. The implementation of the rules in this work are intended to act as a reference implementation and not necessarily be incorporated directly into the main development branches; the theoretical discussion is the more useful part of this work. We discuss practical tradeoffs of implementing evaluation with environments. Evaluation using environments leads to some improvement in performance in many programs where lazy variable lookup is beneficial, and leads to a major improvement in performance in some programs due to the effect of memoizing environments, but comes at the expense of a great deal of extra complexity that may obscure Hazel's original goal in approaching the gap problem. We also describe a possible implementation of FAR and entrypoint for the FAR algorithm, with possibilities for further memoizing re-evaluation. The empirical results that are presented may serves as a guideline for performance benefits, but it will be useful to collect user editing and program statistics to better evaluate the average or expected performance benefit of the proposed changes. Along the way, we introduce several novel generalizations that both simplify implementation and give nice theoretical interpretations, such as generalized closures and the presentation of $n$-step FAR as a generalization of evaluation.

%%% Local Variables:
%%% mode: latex
%%% TeX-master: "main"
%%% End:
