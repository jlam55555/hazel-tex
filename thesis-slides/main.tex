\documentclass{beamer}

\usetheme{AnnArbor}

\def\OPTIONConf{1}
\usepackage{joshuadunfield}
\usepackage{llproof}
\input{../latex-includes/macros.tex}
\newcommand{\vs}{\vspace{20pt}}

\newcommand{\pt}[1]{\MakeUppercase{
    \centering{}
    \large{}
    #1 \\
    \vs{}
}}

%%% Local Variables:
%%% mode: latex
%%% TeX-master: "../thesis/main"
%%% End:


\newtheorem{mtheorem}{Metatheorem}[section]

\title[Hazel evaluation improvements]{Practical performance enhancements to the evaluation model of the Hazel programming environment}

\author[Lam]
{
  Jonathan~Lam\inst{1} \and Prof. Fred Fontaine, Advisor\inst{1} \\
  \and Prof. Robert Marano, Co-advisor\inst{1} \and Prof. Cyrus Omar\inst{2}
}

\institute[Cooper Union]
{
  \inst{1}%
  Electrical Engineering\\
  The Cooper Union for the Advancement of Science and Art
  \and
  \inst{2}%
  Electrical Engineering and Computer Science\\
  Future of Programming Lab (FPLab), University of Michigan
}

\date[Spring 2022]{2022/04/29}

% https://www.overleaf.com/learn/latex/Beamer
\AtBeginSection[]
{
  \begin{frame}
    \frametitle{Table of Contents}
    \tableofcontents[currentsection]
  \end{frame}
}

\begin{document}

\frame{\titlepage}

\begin{frame}[allowframebreaks]
  \frametitle{Overview}

  Project context
  \begin{description}
  \item[Implementation-based] Mostly practically-driven
  \item[Functional programming] Context for PL theory
  \item[Hazel live programming environment] An experimental editor with typed holes aimed at solving the ``gap problem,'' developed at UM
  \end{description}

  \vspace{12em} Project scope
  \begin{description}
  \item[Evaluation with environments] Lazy variable lookup for performance
  \item[Hole instances to hole closures] Redefining hole instances for performance
  \item[Implementing fill-and-resume (FAR)] Efficiently resume evaluation
  \end{description}

  \vspace{4em}Project evaluation
  \begin{description}
  \item[Empirical evaluation] Measure performance gain of motivating cases
  \item[Informal metatheory] State metatheorems and provide proof sketches
  \end{description}
\end{frame}

\section{Primer on PL theory}

\begin{frame}
  \frametitle{A programming language is a specification}

  \begin{description}
  \item[Syntax] is the grammar of a valid program
  \item[Semantics] describes the behavior of a syntactically valid program
  \end{description}

  \begin{figure}
    \centering
    \begin{align*}
  \tau &::= \tau\to\tau
         \mid b
         \mid \tehole \\
  e &::= c
      \mid x
      \mid\lambda x:\tau.e
      \mid e\ e
      \mid e:\tau
      \mid \hehole
      \mid \hhole{e}
\end{align*}

    \caption{Hazelnut grammar}
    \label{fig:hazelnut-grammar}
  \end{figure}

\end{frame}

\begin{frame}
  \frametitle{Static and dynamic semantics}

  \begin{description}
  \item[Statics] Edit actions, type-checking, elaboration (``compile-time'')
  \item[Dynamics] Evaluation (``run-time'')
  \end{description}

  \begin{figure}
    \centering
    \begin{mathpar}
      \Infer{EAp}{\
        e_1\Downarrow \lambda x.e_1' \\
        e_2\Downarrow e_2' \\
        [e_2'/x]e_1'=e
      }{e_1\ e_2\Downarrow e}
    \end{mathpar}
    \caption{Evaluation rule for function application using a big-step semantics}
    \label{fig:inference-rules}
  \end{figure}
\end{frame}

\begin{frame}
  \frametitle{A brief primer on the $\lambda$-calculus}

  \begin{description}
  \item[Untyped $\lambda$-calculus] Simple universal model of computation by Church
  \item[Simply-typed $\lambda$-calculus] Extension of the ULC with static type-checking
  \item[Gradually-typed $\lambda$-calculus] Optionally-typed, with ``pay-as-you-go'' benefits of static typing
  \end{description}

  \begin{figure}
    \centering
    \begin{subfigure}[b]{.2\textwidth}
        \begin{align*}
          e &::= x\\
            &\mid \lambda x.e\\
            &\mid e\ e
        \end{align*}
      \caption{Grammar}
    \end{subfigure}\qquad
    \begin{subfigure}[b]{.7\textwidth}
      \begin{mathpar}
  \Infer{\ulc{}-ELam}{}{\lambda x.e\Downarrow\lambda x.e}
  \and
  \Infer{\ulc{}-EAp}{
    e_1\Downarrow\lambda x.e_1' \\
    [e_2/x]e_1'\Downarrow e
  }{e_1\ e_2\Downarrow e}
\end{mathpar}

      \caption{Dynamic semantics}
    \end{subfigure}
    \caption{The untyped $\lambda$-calculus}
    \label{fig:ulc}
  \end{figure}
\end{frame}

\section{The Hazel live programming environment}

\begin{frame}
  \frametitle{The Hazel programming language and environment}

  \begin{description}
  \item[Live programming] Rapid static and dynamic feedback (``gap problem'')
  \item[Structured editor] Elimination of syntax errors
  \item[Bidirectionally typed] Simple type inference
  \item[Gradually typed] Hole type and cast-calculus based on Siek et al. \cite{Siek06gradualtyping,siek2015refined}
  \item[Purely functional] Avoids side-effects and promotes commutativity
  \end{description}

  \begin{figure}
    \centering
    \begin{subfigure}[b]{0.5\textwidth}
      \centering
      \includegraphics[height=5em]{thesis/img/hazelgrove.png}
      \caption{The Hazelgrove organization}
    \end{subfigure}%
    \begin{subfigure}[b]{0.5\textwidth}
      \centering
      \includegraphics[height=5em]{thesis/img/reasonml.png}
      \caption{Implemented in ReasonML and JSOO}
    \end{subfigure}
    \caption{Hazel implementation}
  \end{figure}
\end{frame}

\begin{frame}
  \frametitle{The Hazel programming interface}

  \begin{figure}
    \centering
    \includegraphics[width=.9\linewidth]{thesis/img/hazel_ui_annot.png}
    \caption{The Hazel interface}
    \label{fig:hazel-ui}
  \end{figure}

\end{frame}

\begin{frame}
  \frametitle{Hazelnut: A bidirectionally-typed static semantics}

\end{frame}

\begin{frame}
  \frametitle{Hazelnut Live: A bidirectionally-typed dynamic semantics}

\end{frame}

\section{Evaluation using the environment model}

\begin{frame}
  \frametitle{Evaluation using environments vs. substitution}
\end{frame}

\begin{frame}
  \frametitle{Updated evaluation rules}

\end{frame}

\begin{frame}
  \frametitle{Handling recursion}

\end{frame}

\begin{frame}
  \frametitle{Matching the result from evaluation using substitution}

\end{frame}

\begin{frame}
  \frametitle{Memoizing by environments for substitution and equality checking}

\end{frame}

\begin{frame}
  \frametitle{Generalized closures}

\end{frame}

\section{Identifying hole instances by physical environment}

\begin{frame}
  \frametitle{Motivating example}
\end{frame}

\begin{frame}
  \frametitle{Hole instances vs. hole closures/instantiations}
\end{frame}

\begin{frame}
  \frametitle{Hole instance parent vs. hole closure parents}
\end{frame}

\begin{frame}
  \frametitle{The hole numbering algorithm}
\end{frame}

\begin{frame}
  \frametitle{A unified postprocessing algorithm}
\end{frame}

\section{The fill-and-resume (FAR) optimization}

\begin{frame}
  \frametitle{Motivating example}
\end{frame}

\begin{frame}
  \frametitle{The FAR process}
\end{frame}

\begin{frame}
  \frametitle{1-step vs. $n$-step FAR}
\end{frame}

\begin{frame}
  \frametitle{Detecting a valid fill operation}
\end{frame}

\begin{frame}
  \frametitle{The fill operation}
\end{frame}

\begin{frame}
  \frametitle{The resume operation}
\end{frame}

\begin{frame}
  \frametitle{The postprocessing operation}
\end{frame}

\section{Empirical results}

\begin{frame}
  \frametitle{Evaluation with environments}
\end{frame}

\begin{frame}
  \frametitle{Hole numbering motivating example}
\end{frame}

\begin{frame}
  \frametitle{FAR motivating example}
\end{frame}

\section{Theoretical results/innovations}

\begin{frame}
  \frametitle{Generalized closures}
\end{frame}

\begin{frame}
  \frametitle{Unique hole closures}
\end{frame}

\begin{frame}
  \frametitle{FAR as a generalization of evaluation}
\end{frame}

\section{Future work/conclusions}

\begin{frame}
  \frametitle{Future work}

  \begin{description}
  \item[$n$-step FAR] Integrate edit history into FAR
  \item[Generalized memoization] Unify notation and metatheory of memoization
  \item[Formal evaluation of metatheory] Check coverage and correctness of metatheorems using Agda
  \end{description}
\end{frame}


\begin{frame}
  \frametitle{Conclusions}

  \begin{description}
  \item[Evaluation with environments] Expected performance gains, implementation remains pure
  \item[Generalized closures] Simplify many parts of the implementation
  \end{description}
\end{frame}

\begin{frame}[allowframebreaks]
  \frametitle{References}

  \footnotesize
  \bibliographystyle{unsrt}
  \bibliography{thesis/refs}
\end{frame}

\end{document}

%%% Local Variables:
%%% mode: latex
%%% TeX-master: t
%%% End:
