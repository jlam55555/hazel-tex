\documentclass{beamer}

\usetheme{AnnArbor}

\def\OPTIONConf{1}
\usepackage{joshuadunfield}
\usepackage{llproof}
\input{../latex-includes/macros.tex}
\newcommand{\vs}{\vspace{20pt}}

\newcommand{\pt}[1]{\MakeUppercase{
    \centering{}
    \large{}
    #1 \\
    \vs{}
}}

%%% Local Variables:
%%% mode: latex
%%% TeX-master: "../thesis/main"
%%% End:


\newtheorem{mtheorem}{Metatheorem}[section]

\title[Hazel evaluation improvements]{Practical performance enhancements to the evaluation model of the Hazel programming environment}

\author[Lam]
{
  Jonathan~Lam\inst{1} \and Prof. Fred Fontaine, Advisor\inst{1} \\
  \and Prof. Robert Marano, Co-advisor\inst{1} \and Prof. Cyrus Omar\inst{2}
}

\institute[Cooper Union]
{
  \inst{1}%
  Electrical Engineering\\
  The Cooper Union for the Advancement of Science and Art
  \and
  \inst{2}%
  Electrical Engineering and Computer Science\\
  Future of Programming Lab (FPLab), University of Michigan
}

\date[Spring 2022]{2022/04/29}

% https://www.overleaf.com/learn/latex/Beamer
\AtBeginSection[]
{
  \begin{frame}
    \frametitle{Table of Contents}
    \tableofcontents[currentsection]
  \end{frame}
}

\begin{document}

\frame{\titlepage}

\section{Primer on PL theory}

\begin{frame}
  \frametitle{A programming language is a specification}

  \todo{semantics}

  \todo{syntax}

\end{frame}

\begin{frame}
  \frametitle{A brief primer on the $\lambda$-calculus}

\end{frame}

\section{The Hazel live programming environment}

\begin{frame}
  \frametitle{The Hazel programming environment}

\end{frame}

\begin{frame}
  \frametitle{Hazelnut: A bidirectionally-typed static semantics}

\end{frame}

\begin{frame}
  \frametitle{Hazelnut Live: A bidirectionally-typed dynamic semantics}

\end{frame}

\section{Evaluation using the environment model}

\begin{frame}
  \frametitle{Evaluation using environments vs. substitution}
\end{frame}

\begin{frame}
  \frametitle{Updated evaluation rules}

\end{frame}

\begin{frame}
  \frametitle{Handling recursion}

\end{frame}

\begin{frame}
  \frametitle{Matching the result from evaluation using substitution}

\end{frame}

\begin{frame}
  \frametitle{Memoizing by environments for substitution and equality checking}

\end{frame}

\begin{frame}
  \frametitle{Generalized closures}

\end{frame}

\section{Identifying hole instances by physical environment}

\begin{frame}
  \frametitle{Motivating example}
\end{frame}

\begin{frame}
  \frametitle{Hole instances vs. hole closures/instantiations}
\end{frame}

\begin{frame}
  \frametitle{Hole instance parent vs. hole closure parents}
\end{frame}

\begin{frame}
  \frametitle{The hole numbering algorithm}
\end{frame}

\begin{frame}
  \frametitle{A unified postprocessing algorithm}
\end{frame}

\section{The fill-and-resume (FAR) optimization}

\begin{frame}
  \frametitle{Motivating example}
\end{frame}

\begin{frame}
  \frametitle{The FAR process}
\end{frame}

\begin{frame}
  \frametitle{1-step vs. $n$-step FAR}
\end{frame}

\begin{frame}
  \frametitle{Detecting a valid fill operation}
\end{frame}

\begin{frame}
  \frametitle{The fill operation}
\end{frame}

\begin{frame}
  \frametitle{The resume operation}
\end{frame}

\begin{frame}
  \frametitle{The postprocessing operation}
\end{frame}

\section{Empirical results}

\begin{frame}
  \frametitle{Evaluation with environments}
\end{frame}

\begin{frame}
  \frametitle{Hole numbering motivating example}
\end{frame}

\begin{frame}
  \frametitle{FAR motivating example}
\end{frame}

\section{Theoretical results/innovations}

\begin{frame}
  \frametitle{Generalized closures}
\end{frame}

\begin{frame}
  \frametitle{Unique hole closures}
\end{frame}

\begin{frame}
  \frametitle{FAR as a generalization of evaluation}
\end{frame}

\section{Future work/conclusions}

\begin{frame}
  \frametitle{Completion of $n$-step FAR}
\end{frame}

\begin{frame}
  \frametitle{Generalized memoization}
\end{frame}

\begin{frame}
  \frametitle{Formal evaluation of metatheory}
\end{frame}

\begin{frame}
  \frametitle{Conclusions}
\end{frame}

\begin{frame}
  \frametitle{References}

  \todo{include metatheory}
\end{frame}

\end{document}

%%% Local Variables:
%%% mode: latex
%%% TeX-master: t
%%% End:
