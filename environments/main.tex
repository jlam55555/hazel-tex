\documentclass{article}

% see readme
\def\OPTIONConf{1}%
\usepackage{joshuadunfield}
\usepackage{llproof}
\input{../latex-includes/macros.tex}

\usepackage{amsmath}
\usepackage{stmaryrd}
\usepackage{hyperref}

\newtheorem{theorem}{Theorem}

\title{Evaluation with environments}
\author{Jonathan Lam}
\date{2022/01/03}

\begin{document}

\maketitle{}

\section{Motivation}

Evaluation with substitution is not efficient because it forces the re-evaluation of the substituted expression every time it is encountered. A more efficient involves an environment model, where variable values are evaluated and stored in an environment when bound and looked-up when encountered.

(Is evaluation with substitution considered normal order evaluation? This seems similar to normal/applicative order evaluation described in \href{https://mitpress.mit.edu/sites/default/files/sicp/full-text/book/book-Z-H-10.html#%_sec_1.1.5}{SICP 1.1.5}.)

\section*{Big step semantics}

The irreducible judgment (for internal expressions) in Hazel is not $d\textsf{ val}$, but rather $d\textsf{ final}$. Thus, final expressions evaluate to themselves. Variables evaluate to the final value that they are bound to (assuming they are bound; otherwise they are free and thus final). Lambdas evaluate to a closure type. The evaluation of a let-expression or function application extends the current environment with the newly-bound variable. For function applications, the current environment is first extended with the closure environment before binding the new variable. When extending an environment ($E::E'$ or $E,x\leftarrow d$), bindings on the right overwrite bindings on the left. \\

\judgbox{d\Downarrow d'}{Internal expression $d$ evaluates to $d'$}

\begin{mathpar}
  \Infer{Eval-Final}{d\text{ final}}{E\vdash d\Downarrow d}
  \and
  \Infer{Eval-Var}{}{E,x\leftarrow d\vdash x\Downarrow d}
  \and
  \Infer{Eval-Lam}{}{E\vdash(\lambda x:\tau.d)\Downarrow [E](\lambda x:\tau.d)}
  \and
  \Infer{Eval-Ap}{
    d_2\Downarrow d_2' \\
    E::E',x\leftarrow d_2'\vdash d_1\Downarrow d
  }{E\vdash[E'](\lambda x:\tau.d_1)(d_2)\Downarrow d}
  \and
  \Infer{Eval-Let}{
    d_2\Downarrow d_2' \\
    E,x\leftarrow d_2'\vdash d_1\Downarrow d
  }{E\vdash\text{let }x=d_2\text{ in }d_1\Downarrow d}
\end{mathpar}
The following metatheorem states that environments only include final terms.
\begin{theorem}
  If the variable binding $x\leftarrow d$ exists in $E$, then $d\textsf{ final}$.
\end{theorem}

\end{document}

%%% Local Variables:
%%% mode: latex
%%% TeX-master: t
%%% End:
