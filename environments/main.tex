\documentclass{article}

% see readme
\def\OPTIONConf{1}%
\usepackage{joshuadunfield}
\usepackage{llproof}
% !TEX root = hazelnut-popl17.tex

% Violet hotdogs; highlight color helps distinguish them
\newcommand{\llparenthesiscolor}{\textcolor{violet}{\llparenthesis}}
\newcommand{\rrparenthesiscolor}{\textcolor{violet}{\rrparenthesis}}

% HTyp and HExp
\newcommand{\hcomplete}[1]{#1~\mathsf{complete}}

% HTyp
\newcommand{\htau}{\dot{\tau}}
\newcommand{\tarr}[2]{\inparens{#1 \rightarrow #2}}
\newcommand{\tarrnp}[2]{#1 \rightarrow #2}
\newcommand{\tnum}{\mathtt{num}}
\newcommand{\tehole}{\llparenthesiscolor\rrparenthesiscolor}
\newcommand{\tsum}[2]{\inparens{{#1} + {#2}}}

\newcommand{\tcompat}[2]{#1 \sim #2}
\newcommand{\tincompat}[2]{#1 \nsim #2}

% HExp
\newcommand{\hexp}{\dot{e}}
\newcommand{\hlam}[2]{\inparens{\lambda #1.#2}}
\newcommand{\hap}[2]{#1(#2)}
\newcommand{\hapP}[2]{(#1)~(#2)} % Extra paren around function term
\newcommand{\hnum}[1]{\underline{#1}}
\newcommand{\hadd}[2]{\inparens{#1 + #2}}
\newcommand{\hehole}{\llparenthesiscolor\rrparenthesiscolor}
\newcommand{\hhole}[1]{\llparenthesiscolor#1\rrparenthesiscolor}
\newcommand{\hindet}[1]{\lceil#1\rceil}
\newcommand{\hinj}[2]{\mathtt{inj}_{#1}({#2})}
\newcommand{\hcase}[5]{\mathtt{case}({#1},{#2}.{#3},{#4}.{#5})}

\newcommand{\hGamma}{\dot{\Gamma}}
\newcommand{\domof}[1]{\text{dom}(#1)}
\newcommand{\hsyn}[3]{#1 \vdash #2 \Rightarrow #3}
\newcommand{\hana}[3]{#1 \vdash #2 \Leftarrow #3}

% ZTyp and ZExp
\newcommand{\zlsel}[1]{{\bowtie}{#1}}
\newcommand{\zrsel}[1]{{#1}{\bowtie}}
\newcommand{\zwsel}[1]{
  \setlength{\fboxsep}{0pt}
  \colorbox{green!10!white!100}{
    \ensuremath{{{\textcolor{Green}{{\hspace{-2px}\triangleright}}}}{#1}{\textcolor{Green}{\triangleleft{\vphantom{\tehole}}}}}}
}

\newcommand{\removeSel}[1]{#1^{\diamond}}

% ZTyp
\newcommand{\ztau}{\hat{\tau}}

% ZExp
\newcommand{\zexp}{\hat{e}}

% Direction
\newcommand{\dParent}{\mathtt{parent}}
\newcommand{\dChildn}[1]{\mathtt{child}~\mathtt{{#1}}}
\newcommand{\dChildnm}[1]{\mathtt{child}~{#1}}

% Action
\newcommand{\aMove}[1]{\mathtt{move}~#1}
	\newcommand{\zrightmost}[1]{\mathsf{rightmost}(#1)}
	\newcommand{\zleftmost}[1]{\mathsf{leftmost}(#1)}
\newcommand{\aSelect}[1]{\mathtt{sel}~#1}
\newcommand{\aDel}{\mathtt{del}}
\newcommand{\aReplace}[1]{\mathtt{replace}~#1}
\newcommand{\aConstruct}[1]{\mathtt{construct}~#1}
\newcommand{\aConstructx}[1]{#1}
\newcommand{\aFinish}{\mathtt{finish}}

\newcommand{\performAna}[5]{#1 \vdash #2 \xlongrightarrow{#4} #5 \Leftarrow #3}
\newcommand{\performAnaI}[5]{#1 \vdash #2 \xlongrightarrow{#4}\hspace{-3px}{}^{*}~ #5 \Leftarrow #3}
\newcommand{\performSyn}[6]{#1 \vdash #2 \Rightarrow #3 \xlongrightarrow{#4} #5 \Rightarrow #6}
\newcommand{\performSynI}[6]{#1 \vdash #2 \Rightarrow #3 \xlongrightarrow{#4}\hspace{-3px}{}^{*}~ #5 \Rightarrow #6}
\newcommand{\performTyp}[3]{#1 \xlongrightarrow{#2} #3}
\newcommand{\performTypI}[3]{#1 \xlongrightarrow{#2}\hspace{-3px}{}^{*}~#3}

\newcommand{\performMove}[3]{#1 \xlongrightarrow{#2} #3}
\newcommand{\performDel}[2]{#1 \xlongrightarrow{\aDel} #2}

% Form
\newcommand{\farr}{\mathtt{arrow}}
\newcommand{\fnum}{\mathtt{num}}
\newcommand{\fsum}{\mathtt{sum}}

\newcommand{\fasc}{\mathtt{asc}}
\newcommand{\fvar}[1]{\mathtt{var}~#1}
\newcommand{\flam}[1]{\mathtt{lam}~#1}
\newcommand{\fap}{\mathtt{ap}}
% \newcommand{\farg}{\mathtt{arg}}
\newcommand{\fnumlit}[1]{\mathtt{lit}~#1}
\newcommand{\fplus}{\mathtt{plus}}
\newcommand{\fhole}{\mathtt{hole}}
\newcommand{\fnehole}{\mathtt{nehole}}

\newcommand{\finj}[1]{\mathtt{inj}~#1}
\newcommand{\fcase}[2]{\mathtt{case}~#1~#2}

% Talk about formal rules in example
\newcommand{\refrule}[1]{\textrm{Rule~(#1)}}

\newcommand{\herase}[1]{\left|#1\right|_\textsf{erase}}

\newcommand{\arrmatch}[2]{#1 \blacktriangleright_{\rightarrow} #2}


\newcommand{\TABperformAna}[5]{#1 \vdash & #2                & \xlongrightarrow{#4} & #5 & \Leftarrow #3}
\newcommand{\TABperformSyn}[6]{#1 \vdash & #2 \Rightarrow #3 & \xlongrightarrow{#4} & #5 \Rightarrow #6}
\newcommand{\TABperformTyp}[3]{& #1 & \xlongrightarrow{#2} & #3}

\newcommand{\TABperformMove}[3]{#1 & \xlongrightarrow{#2} & #3}
\newcommand{\TABperformDel}[2]{#1 \xlongrightarrow{\aDel} #2}

\newcommand{\sumhasmatched}[2]{#1 \mathrel{\textcolor{black}{\blacktriangleright_{+}}} #2}

\newcommand{\subminsyn}[1]{\mathsf{submin}_{\Rightarrow}(#1)}
\newcommand{\subminana}[1]{\mathsf{submin}_{\Leftarrow}(#1)}


\newcommand{\inparens}[1]{{\color{gray}(}#1{\color{gray})}}

%% rule names for appendix
\newcommand{\rname}[1]{\textsc{#1}}
\newcommand{\gap}{\vspace{7pt}}


\usepackage{amsmath}
\usepackage{stmaryrd}
\usepackage{hyperref}

\newtheorem{theorem}{Theorem}

\title{Evaluation with environments}
\author{Jonathan Lam}
\date{2022/01/03}

\begin{document}

\maketitle{}

\section{Motivation}

Evaluation with substitution is not efficient because it forces the re-evaluation of the substituted expression every time it is encountered. A more efficient involves an environment model, where variable values are evaluated and stored in an environment when bound and looked-up when encountered.

(Is evaluation with substitution considered normal order evaluation? This seems similar to normal/applicative order evaluation described in \href{https://mitpress.mit.edu/sites/default/files/sicp/full-text/book/book-Z-H-10.html#%_sec_1.1.5}{SICP 1.1.5}.)

\section*{Big step semantics}

The irreducible judgment (for internal expressions) in Hazel is not $d\textsf{ val}$, but rather $d\textsf{ final}$. Thus, final expressions evaluate to themselves. Variables evaluate to the final value that they are bound to (assuming they are bound; otherwise they are free and thus final). Lambdas evaluate to a closure type. The evaluation of a let-expression or function application extends the current environment with the newly-bound variable. For function applications, the current environment is first extended with the closure environment before binding the new variable. When extending an environment ($E::E'$ or $E,x\leftarrow d$), bindings on the right overwrite bindings on the left. \\

\judgbox{d\Downarrow d'}{Internal expression $d$ evaluates to $d'$}

\begin{mathpar}
  \Infer{Eval-Final}{d\text{ final}}{E\vdash d\Downarrow d}
  \and
  \Infer{Eval-Var}{}{E,x\leftarrow d\vdash x\Downarrow d}
  \and
  \Infer{Eval-Lam}{}{E\vdash(\lambda x:\tau.d)\Downarrow [E](\lambda x:\tau.d)}
  \and
  \Infer{Eval-Ap}{
    d_2\Downarrow d_2' \\
    E::E',x\leftarrow d_2'\vdash d_1\Downarrow d
  }{E\vdash[E'](\lambda x:\tau.d_1)(d_2)\Downarrow d}
  \and
  \Infer{Eval-Let}{
    d_2\Downarrow d_2' \\
    E,x\leftarrow d_2'\vdash d_1\Downarrow d
  }{E\vdash\text{let }x=d_2\text{ in }d_1\Downarrow d}
\end{mathpar}
The following metatheorem states that environments only include final terms.
\begin{theorem}
  If the variable binding $x\leftarrow d$ exists in $E$, then $d\textsf{ final}$.
\end{theorem}

\clearpage

%% Cyrus: We took a look at the dynamics. Overall it seems like the
%% right idea. We noticed:

%% Done: 1) The stepping rules are non-deterministic (i.e. you can
%% step the right or left of e1 + e2 in any order). Might be useful to
%% make them deterministic.

%% Done (mostly): 2) The premises that have disjunctions in them could
%% be broken out into two rules -- this would take a little more space
%% but follows the usual conventions more closely.

%% Done (mostly): 3) We need to add the ``ceil'' forms to the grammar
%% of \dot{e} (you just used e) and give them a static semantics.

%% TODO: 4) We need to figure out (the analogs of) canonical forms,
%% preservation and progress -- I guess we had decided on defining a
%% declarative statics to do that. Ian has started to prove the
%% correspondence (sans the ceil forms).

%% #4 seems like the most important next step.

\begin{figure}[htbp]
  \centering

  \judgbox{\hexp~\textsf{value}}{H-Expression $\hexp$ is a closed value}
  \begin{mathpar}
    \Infer{V-num}
          { }
          {\hnum{n}~\textsf{value}}
    \and
    \Infer{V-lam}
        {}
        {\hlam{x}{\hexp}~\textsf{value}}
  \end{mathpar}
  
  \caption{Value forms}
  \label{fig:judg-value}
\end{figure}

\begin{figure}[htbp]
  \centering

  \judgbox{\hexp~\textsf{final}}{H-Expression $\hexp$ is final}
  \begin{mathpar}
    \Infer{F-val}
          {\hexp~\textsf{value}}
          {\hexp~\textsf{final}}
    \and
    \Infer{F-filled}
          {\hexp~\textsf{final}}
          {\hhole{\hexp}~\textsf{final}}
    \and
    \Infer{F-unfilled}
          { }
          {\hhole{}~\textsf{final}}
    \and
    \Infer{F-indet}
        {\hexp~\textsf{indet}}
        {\hindet{\hexp}~\textsf{final}}
  \end{mathpar}
  
  \caption{Final forms}
  \label{fig:judg-value}
\end{figure}

\begin{figure}[htbp]
  \centering

  \judgbox{\hexp~\textsf{indet}}{H-Expression~$\hexp$ is indeterminate}
  \begin{mathpar}
    \Infer{I-plus$_1$}
          { \hexp_1~\textsf{final} \\
            \hexp_2~\textsf{final} \\
            \hexp_1 \ne \hnum{n_1}
          }
          {\hadd{e_1}{e_2}~\textsf{indet}}
    \and
    \Infer{I-plus$_2$}
          { \hexp_1~\textsf{final} \\
            \hexp_2~\textsf{final} \\
            \hexp_2 \ne \hnum{n_2}
          }
          {\hadd{\hexp_1}{\hexp_2}~\textsf{indet}}
    \and
    \Infer{I-app}
          { \hexp_1~\textsf{final} \\
            \hexp_2~\textsf{final} \\
            \hexp_1 \ne \hlam{x}{\hexp_1'}
          }
          {\hap{\hexp_1}{\hexp_2}~\textsf{indet}}
    \and
  \end{mathpar}
  
  \caption{Indeterminate forms}
  \label{fig:judg-value}
\end{figure}

\begin{figure}[htbp]
  \centering

  \judgbox{\hexp_1 \longrightarrow \hexp_2}{H-Expression $\hexp_1$ steps to~$\hexp_2$}
  \begin{mathpar}
    \Infer{S-plus$_1$}
          { \hexp_1 \longrightarrow \hexp_1' }
          { \hadd{\hexp_1}{\hexp_2} \longrightarrow \hadd{\hexp_1'}{\hexp_2} }

    \Infer{S-plus$_2$}
          { \hexp_1~\textsf{final}
            \\
            \hexp_2 \longrightarrow \hexp_2' }
          { \hadd{\hexp_1}{\hexp_2} \longrightarrow \hadd{\hexp_1}{\hexp_2'} }

    \Infer{S-plus$_3$}
          { n_1 + n_2 = n_3 }
          { \hadd{\hnum{n_1}}{\hnum{n_2}} \longrightarrow \hnum{n_3} }

   \Infer{S-plus$_4$}
          { \hexp_1~\textsf{final} ~~ ~~ ~~
            \hexp_2~\textsf{final} 
            \\\\
            \left(\hexp_1 \ne \hnum{n_1} \vee \hexp_2 \ne \hnum{n_2}\right)
          }
          { \hadd{\hexp_1}{\hexp_2} \longrightarrow \hindet{\hadd{\hexp_1}{\hexp_2}} }

    \Infer{S-ap$_1$}
          { \hexp_1 \longrightarrow \hexp_1' }
          { \hap{\hexp_1}{\hexp_2} \longrightarrow \hap{\hexp_1'}{\hexp_2} }

    \Infer{S-ap$_2$}
          { \hexp_2 \longrightarrow \hexp_2'
            \\
            \hexp_1~\textsf{final}
          }
          { \hap{\hexp_1}{\hexp_2} \longrightarrow \hap{\hexp_1}{\hexp_2'} }

    \Infer{S-ap$_3$}
          { \hexp_2~\textsf{final} }
          { \hap{\hlam{x}{\hexp_1}}{\hexp_2} \longrightarrow [\hexp_2/x]\hexp_1 }

   \Infer{S-ap$_4$}
          { \hexp_1~\textsf{final} ~~ ~~ ~~
            \hexp_2~\textsf{final}
            \\\\
            \hexp_1 \ne \hlam{x}{\hexp_1'} ~~ ~~ ~~ ~~ ~~ ~~ ~~ { }
          }
          { \hap{\hexp_1}{\hexp_2} \longrightarrow \hindet{\hap{\hexp_1}{\hexp_2}} }

  \end{mathpar}
  
  \caption{Small-step operational semantics.}
  \label{fig:judg-value}
\end{figure}

% Q: Are we running programs with ascriptions?
% No. We erase the ascriptions before running. Ascriptions below are
%
% Q: Are the ascriptions needed to type-check the programs?
% No. The ascriptions are needed to check the programs algorithmically, not to derive typing derivations.
% (Q: How do ascriptions, holes, and typing derivations interact?)
%
% Q: Does it ever make sense to run open programs? (i.e., run programs with ``free variables''?)
%  -- Yes, but only in the sense that we can bind holes to variables,
%     ascribe them types, and (attempt to) compute with them.
%
% Example: (where '?' means "empty hole")
%  let x = ? : int -> int in
%  let y = ? : int in
%  x y : int
% ==erase==>
%  let x = ? in
%  let y = ? in
%  x y
% -->
%  let y = ? in
%  ? y
% -->
%  ? ?
% --> 
%  (indet ? ?)

We extend the syntax of H-Expressions as follows:
\[
\hexp ::= \cdots~|~\hindet{\hexp}
\]

Here are some things that we want to prove:

\textbf{Def}~(Ascription erasure). 
\\
$\herase{\hexp}$ is the same as $\hexp$, but without its type ascriptions.
%
All cases are congruences, except for the ascription case, where $\herase{\hexp : \htau} = \hexp$.

\textbf{Def}~(Declarative typing).
\\
The judgement $\hGamma \vdash \hexp : \htau$ is a type assignment
system for erased terms.  It is declarative, and unlike the
bidirectional rules, is not algorithmic.
%
We employ it to relate bidirectionally-typed terms to (erased) terms
that enjoy type soundness with respect to the dynamics.

\textbf{Conjecture}~(Bidirectional implies declarative).
\\
(i) If $\hsyn{\hGamma}{\hexp}{\htau}$ then $\hGamma \vdash \herase{\hexp} : \htau$.
\\
(ii) If $\hana{\hGamma}{\hexp}{\htau}$ then $\hGamma \vdash \herase{\hexp} : \htau$.

\textbf{Conjecture}~(Substitution).
\\
If $\hGamma, x : \tau_x \vdash \hexp : \htau$
\\
and $\hexp'~\textsf{final}$~~~(do we actually need this condition?)
\\
and $\hGamma \vdash \hexp' : \htau_x$
\\
then $\hGamma \vdash \hexp[\hexp' / x] : \htau$

% Q: Do we need to enforce that hexp' is final?

\textbf{Conjecture}~(Canonical forms)
\\
If $\cdot \vdash \hexp : \htau$
\\
and $\hexp~\textsf{final}$ then
\begin{itemize}
\item if $\hexp = \hindet{\hexp'}$ then $\hexp'~\textsf{indet}$ and $\cdot \vdash \hexp' : \htau$
%\marginnote{Since $\hexp'~\textsf{indet}$, the conjecture applies to $\hexp'$.}
\item else:
\begin{itemize}
\item if $\htau = \tnum$ then exists $\hnum{n}$ such that $\hexp = \hnum{n}$.
\item else if $\htau = \tarr{\htau_1}{\htau_2}$ then exists $x$ and $\hexp'$ such that $\hexp = \hlam{x}{\hexp'}$
\item else if $\htau = \tehole$ then either:
\begin{itemize}
\item $\hexp = \hehole$, or
\item exists $\htau'$ and $\hexp'$ such that $\hexp = \hhole{\hexp'}$, $\hexp'~\textsf{final}$ and $\cdot \vdash \hexp' : \htau'$
\end{itemize}
\end{itemize}
\end{itemize}

\textbf{Conjecture}~(Progresss).
\\
If $\cdot \vdash \hexp_1 : \htau$
\\
then either $\hexp_1~\textsf{final}$
\\
or exists $\hexp_2$ such that $\hexp_1 \longrightarrow \hexp_2$.

\textbf{Conjecture}~(Preservation).
\\
If $\cdot\vdash \hexp_1 : \htau$ 
\\
and $\hexp_1 \longrightarrow \hexp_2$
\\
then $\cdot \vdash \hexp_2 : \htau$


\end{document}

%%% Local Variables:
%%% mode: latex
%%% TeX-master: t
%%% End:
