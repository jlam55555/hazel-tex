\documentclass{article}

% see readme
\def\OPTIONConf{1}%
\usepackage{joshuadunfield}
\usepackage{llproof}
% !TEX root = hazelnut-popl17.tex

% Violet hotdogs; highlight color helps distinguish them
\newcommand{\llparenthesiscolor}{\textcolor{violet}{\llparenthesis}}
\newcommand{\rrparenthesiscolor}{\textcolor{violet}{\rrparenthesis}}

% HTyp and HExp
\newcommand{\hcomplete}[1]{#1~\mathsf{complete}}

% HTyp
\newcommand{\htau}{\dot{\tau}}
\newcommand{\tarr}[2]{\inparens{#1 \rightarrow #2}}
\newcommand{\tarrnp}[2]{#1 \rightarrow #2}
\newcommand{\tnum}{\mathtt{num}}
\newcommand{\tehole}{\llparenthesiscolor\rrparenthesiscolor}
\newcommand{\tsum}[2]{\inparens{{#1} + {#2}}}

\newcommand{\tcompat}[2]{#1 \sim #2}
\newcommand{\tincompat}[2]{#1 \nsim #2}

% HExp
\newcommand{\hexp}{\dot{e}}
\newcommand{\hlam}[2]{\inparens{\lambda #1.#2}}
\newcommand{\hap}[2]{#1(#2)}
\newcommand{\hapP}[2]{(#1)~(#2)} % Extra paren around function term
\newcommand{\hnum}[1]{\underline{#1}}
\newcommand{\hadd}[2]{\inparens{#1 + #2}}
\newcommand{\hehole}{\llparenthesiscolor\rrparenthesiscolor}
\newcommand{\hhole}[1]{\llparenthesiscolor#1\rrparenthesiscolor}
\newcommand{\hindet}[1]{\lceil#1\rceil}
\newcommand{\hinj}[2]{\mathtt{inj}_{#1}({#2})}
\newcommand{\hcase}[5]{\mathtt{case}({#1},{#2}.{#3},{#4}.{#5})}

\newcommand{\hGamma}{\dot{\Gamma}}
\newcommand{\domof}[1]{\text{dom}(#1)}
\newcommand{\hsyn}[3]{#1 \vdash #2 \Rightarrow #3}
\newcommand{\hana}[3]{#1 \vdash #2 \Leftarrow #3}

% ZTyp and ZExp
\newcommand{\zlsel}[1]{{\bowtie}{#1}}
\newcommand{\zrsel}[1]{{#1}{\bowtie}}
\newcommand{\zwsel}[1]{
  \setlength{\fboxsep}{0pt}
  \colorbox{green!10!white!100}{
    \ensuremath{{{\textcolor{Green}{{\hspace{-2px}\triangleright}}}}{#1}{\textcolor{Green}{\triangleleft{\vphantom{\tehole}}}}}}
}

\newcommand{\removeSel}[1]{#1^{\diamond}}

% ZTyp
\newcommand{\ztau}{\hat{\tau}}

% ZExp
\newcommand{\zexp}{\hat{e}}

% Direction
\newcommand{\dParent}{\mathtt{parent}}
\newcommand{\dChildn}[1]{\mathtt{child}~\mathtt{{#1}}}
\newcommand{\dChildnm}[1]{\mathtt{child}~{#1}}

% Action
\newcommand{\aMove}[1]{\mathtt{move}~#1}
	\newcommand{\zrightmost}[1]{\mathsf{rightmost}(#1)}
	\newcommand{\zleftmost}[1]{\mathsf{leftmost}(#1)}
\newcommand{\aSelect}[1]{\mathtt{sel}~#1}
\newcommand{\aDel}{\mathtt{del}}
\newcommand{\aReplace}[1]{\mathtt{replace}~#1}
\newcommand{\aConstruct}[1]{\mathtt{construct}~#1}
\newcommand{\aConstructx}[1]{#1}
\newcommand{\aFinish}{\mathtt{finish}}

\newcommand{\performAna}[5]{#1 \vdash #2 \xlongrightarrow{#4} #5 \Leftarrow #3}
\newcommand{\performAnaI}[5]{#1 \vdash #2 \xlongrightarrow{#4}\hspace{-3px}{}^{*}~ #5 \Leftarrow #3}
\newcommand{\performSyn}[6]{#1 \vdash #2 \Rightarrow #3 \xlongrightarrow{#4} #5 \Rightarrow #6}
\newcommand{\performSynI}[6]{#1 \vdash #2 \Rightarrow #3 \xlongrightarrow{#4}\hspace{-3px}{}^{*}~ #5 \Rightarrow #6}
\newcommand{\performTyp}[3]{#1 \xlongrightarrow{#2} #3}
\newcommand{\performTypI}[3]{#1 \xlongrightarrow{#2}\hspace{-3px}{}^{*}~#3}

\newcommand{\performMove}[3]{#1 \xlongrightarrow{#2} #3}
\newcommand{\performDel}[2]{#1 \xlongrightarrow{\aDel} #2}

% Form
\newcommand{\farr}{\mathtt{arrow}}
\newcommand{\fnum}{\mathtt{num}}
\newcommand{\fsum}{\mathtt{sum}}

\newcommand{\fasc}{\mathtt{asc}}
\newcommand{\fvar}[1]{\mathtt{var}~#1}
\newcommand{\flam}[1]{\mathtt{lam}~#1}
\newcommand{\fap}{\mathtt{ap}}
% \newcommand{\farg}{\mathtt{arg}}
\newcommand{\fnumlit}[1]{\mathtt{lit}~#1}
\newcommand{\fplus}{\mathtt{plus}}
\newcommand{\fhole}{\mathtt{hole}}
\newcommand{\fnehole}{\mathtt{nehole}}

\newcommand{\finj}[1]{\mathtt{inj}~#1}
\newcommand{\fcase}[2]{\mathtt{case}~#1~#2}

% Talk about formal rules in example
\newcommand{\refrule}[1]{\textrm{Rule~(#1)}}

\newcommand{\herase}[1]{\left|#1\right|_\textsf{erase}}

\newcommand{\arrmatch}[2]{#1 \blacktriangleright_{\rightarrow} #2}


\newcommand{\TABperformAna}[5]{#1 \vdash & #2                & \xlongrightarrow{#4} & #5 & \Leftarrow #3}
\newcommand{\TABperformSyn}[6]{#1 \vdash & #2 \Rightarrow #3 & \xlongrightarrow{#4} & #5 \Rightarrow #6}
\newcommand{\TABperformTyp}[3]{& #1 & \xlongrightarrow{#2} & #3}

\newcommand{\TABperformMove}[3]{#1 & \xlongrightarrow{#2} & #3}
\newcommand{\TABperformDel}[2]{#1 \xlongrightarrow{\aDel} #2}

\newcommand{\sumhasmatched}[2]{#1 \mathrel{\textcolor{black}{\blacktriangleright_{+}}} #2}

\newcommand{\subminsyn}[1]{\mathsf{submin}_{\Rightarrow}(#1)}
\newcommand{\subminana}[1]{\mathsf{submin}_{\Leftarrow}(#1)}


\newcommand{\inparens}[1]{{\color{gray}(}#1{\color{gray})}}

%% rule names for appendix
\newcommand{\rname}[1]{\textsc{#1}}
\newcommand{\gap}{\vspace{7pt}}


\usepackage{amsmath}
\usepackage{stmaryrd}
\usepackage{hyperref}

\newtheorem{theorem}{Theorem}

\title{Evaluation with environments}
\author{Jonathan Lam}
\date{2022/01/03}

\begin{document}

\maketitle{}

\section{Motivation}

Evaluation with substitution is not efficient because it forces the re-evaluation of the substituted expression every time it is encountered. A more efficient involves an environment model, where variable values are evaluated and stored in an environment when bound and looked-up when encountered.

(Is evaluation with substitution considered normal order evaluation? This seems similar to normal/applicative order evaluation described in \href{https://mitpress.mit.edu/sites/default/files/sicp/full-text/book/book-Z-H-10.html#%_sec_1.1.5}{SICP 1.1.5}.)

\section*{Big step semantics}

The irreducible judgment (for internal expressions) in Hazel is not $d\textsf{ val}$, but rather $d\textsf{ final}$. Thus, final expressions evaluate to themselves. Variables evaluate to the final value that they are bound to (assuming they are bound; otherwise they are free and thus final). Lambdas evaluate to a closure type. The evaluation of a let-expression or function application extends the current environment with the newly-bound variable. For function applications, the current environment is first extended with the closure environment before binding the new variable. When extending an environment ($E::E'$ or $E,x\leftarrow d$), bindings on the right overwrite bindings on the left. \\

\judgbox{d\Downarrow d'}{Internal expression $d$ evaluates to $d'$}

\begin{mathpar}
  \Infer{Eval-Final}{d\text{ final}}{E\vdash d\Downarrow d}
  \and
  \Infer{Eval-Var}{}{E,x\leftarrow d\vdash x\Downarrow d}
  \and
  \Infer{Eval-Lam}{}{E\vdash(\lambda x:\tau.d)\Downarrow [E](\lambda x:\tau.d)}
  \and
  \Infer{Eval-Ap}{
    d_2\Downarrow d_2' \\
    E::E',x\leftarrow d_2'\vdash d_1\Downarrow d
  }{E\vdash[E'](\lambda x:\tau.d_1)(d_2)\Downarrow d}
  \and
  \Infer{Eval-Let}{
    d_2\Downarrow d_2' \\
    E,x\leftarrow d_2'\vdash d_1\Downarrow d
  }{E\vdash\text{let }x=d_2\text{ in }d_1\Downarrow d}
\end{mathpar}
The following metatheorem states that environments only include final terms.
\begin{theorem}
  If the variable binding $x\leftarrow d$ exists in $E$, then $d\textsf{ final}$.
\end{theorem}

\end{document}

%%% Local Variables:
%%% mode: latex
%%% TeX-master: t
%%% End:
